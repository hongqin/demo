



<!DOCTYPE html>
<html lang="en" class=" is-copy-enabled emoji-size-boost is-u2f-enabled">
  <head prefix="og: http://ogp.me/ns# fb: http://ogp.me/ns/fb# object: http://ogp.me/ns/object# article: http://ogp.me/ns/article# profile: http://ogp.me/ns/profile#">
    <meta charset='utf-8'>
    

    <link crossorigin="anonymous" href="https://assets-cdn.github.com/assets/frameworks-7f2a605da6435efc2ee84e59714b38adf17b327656cf5739a7f65a0029fbe2d5.css" integrity="sha256-fypgXaZDXvwu6E5ZcUs4rfF7MnZWz1c5p/ZaACn74tU=" media="all" rel="stylesheet" />
    <link crossorigin="anonymous" href="https://assets-cdn.github.com/assets/github-7c76c3ab0cbc30b4ff801613d7be2e7c7970a2c515e0ed5924a1f413e46916a3.css" integrity="sha256-fHbDqwy8MLT/gBYT174ufHlwosUV4O1ZJKH0E+RpFqM=" media="all" rel="stylesheet" />
    
    
    
    

    <meta http-equiv="X-UA-Compatible" content="IE=edge">
    <meta http-equiv="Content-Language" content="en">
    <meta name="viewport" content="width=device-width">
    
    <title>NOISeq/NOISeq.Rnw at master · Bioconductor-mirror/NOISeq</title>
    <link rel="search" type="application/opensearchdescription+xml" href="/opensearch.xml" title="GitHub">
    <link rel="fluid-icon" href="https://github.com/fluidicon.png" title="GitHub">
    <link rel="apple-touch-icon" href="/apple-touch-icon.png">
    <link rel="apple-touch-icon" sizes="57x57" href="/apple-touch-icon-57x57.png">
    <link rel="apple-touch-icon" sizes="60x60" href="/apple-touch-icon-60x60.png">
    <link rel="apple-touch-icon" sizes="72x72" href="/apple-touch-icon-72x72.png">
    <link rel="apple-touch-icon" sizes="76x76" href="/apple-touch-icon-76x76.png">
    <link rel="apple-touch-icon" sizes="114x114" href="/apple-touch-icon-114x114.png">
    <link rel="apple-touch-icon" sizes="120x120" href="/apple-touch-icon-120x120.png">
    <link rel="apple-touch-icon" sizes="144x144" href="/apple-touch-icon-144x144.png">
    <link rel="apple-touch-icon" sizes="152x152" href="/apple-touch-icon-152x152.png">
    <link rel="apple-touch-icon" sizes="180x180" href="/apple-touch-icon-180x180.png">
    <meta property="fb:app_id" content="1401488693436528">

      <meta content="https://avatars2.githubusercontent.com/u/11679331?v=3&amp;s=400" name="twitter:image:src" /><meta content="@github" name="twitter:site" /><meta content="summary" name="twitter:card" /><meta content="Bioconductor-mirror/NOISeq" name="twitter:title" /><meta content="This is a read-only mirror of the Bioconductor SVN repository. Package Homepage: http://bioconductor.org/packages/devel/bioc/html/NOISeq.html Bug Reports: https://support.bioconductor.org/p/new/pos..." name="twitter:description" />
      <meta content="https://avatars2.githubusercontent.com/u/11679331?v=3&amp;s=400" property="og:image" /><meta content="GitHub" property="og:site_name" /><meta content="object" property="og:type" /><meta content="Bioconductor-mirror/NOISeq" property="og:title" /><meta content="https://github.com/Bioconductor-mirror/NOISeq" property="og:url" /><meta content="This is a read-only mirror of the Bioconductor SVN repository. Package Homepage: http://bioconductor.org/packages/devel/bioc/html/NOISeq.html Bug Reports: https://support.bioconductor.org/p/new/pos..." property="og:description" />
      <meta name="browser-stats-url" content="https://api.github.com/_private/browser/stats">
    <meta name="browser-errors-url" content="https://api.github.com/_private/browser/errors">
    <link rel="assets" href="https://assets-cdn.github.com/">
    <link rel="web-socket" href="wss://live.github.com/_sockets/MzE1MTk2Mzo1NGE1YjI1MTg3NTg1MDRiN2YzYzRiZTZhZmQxMjkxNzowYjJlOTg4NmI5N2JhMmNjMDY3YTdlMjMwMWE3NjE0NmIwYWNiNDZmNjMxYzkyNjUzMTE1NmFlOTNiMjIyNmMz--e3ac620910447ca9f2acf2961c034c176f412577">
    <meta name="pjax-timeout" content="1000">
    <link rel="sudo-modal" href="/sessions/sudo_modal">
    <meta name="request-id" content="C01ED0F2:6284:4249333:584EC9B8" data-pjax-transient>

    <meta name="msapplication-TileImage" content="/windows-tile.png">
    <meta name="msapplication-TileColor" content="#ffffff">
    <meta name="selected-link" value="repo_source" data-pjax-transient>

    <meta name="google-site-verification" content="KT5gs8h0wvaagLKAVWq8bbeNwnZZK1r1XQysX3xurLU">
<meta name="google-site-verification" content="ZzhVyEFwb7w3e0-uOTltm8Jsck2F5StVihD0exw2fsA">
    <meta name="google-analytics" content="UA-3769691-2">

<meta content="collector.githubapp.com" name="octolytics-host" /><meta content="github" name="octolytics-app-id" /><meta content="C01ED0F2:6284:4249333:584EC9B8" name="octolytics-dimension-request_id" /><meta content="3151963" name="octolytics-actor-id" /><meta content="hongqin" name="octolytics-actor-login" /><meta content="fd8da17da4201019bfe50bc2a325f21319ff709f01cf426aee2c14f8fbb42d92" name="octolytics-actor-hash" />
<meta content="/&lt;user-name&gt;/&lt;repo-name&gt;/blob/show" data-pjax-transient="true" name="analytics-location" />



  <meta class="js-ga-set" name="dimension1" content="Logged In">



        <meta name="hostname" content="github.com">
    <meta name="user-login" content="hongqin">

        <meta name="expected-hostname" content="github.com">
      <meta name="js-proxy-site-detection-payload" content="MzFjZTlkYmQyZmQ1NjA3OWNhN2YzNTk3N2JkNmQ2YTQ2MGViOWQyMDMzYTZjNGQ2ZjA5ZjQzMTg0ZTIyMTNhOHx7InJlbW90ZV9hZGRyZXNzIjoiMTkyLjMwLjIwOC4yNDIiLCJyZXF1ZXN0X2lkIjoiQzAxRUQwRjI6NjI4NDo0MjQ5MzMzOjU4NEVDOUI4IiwidGltZXN0YW1wIjoxNDgxNTU4NDU3LCJob3N0IjoiZ2l0aHViLmNvbSJ9">


      <link rel="mask-icon" href="https://assets-cdn.github.com/pinned-octocat.svg" color="#000000">
      <link rel="icon" type="image/x-icon" href="https://assets-cdn.github.com/favicon.ico">

    <meta name="html-safe-nonce" content="4d3c209d9f79744ce861799947845628e9a3bff5">
    <meta content="f316d99f31dced27cd3d76033417f957c977fe36" name="form-nonce" />

    <meta http-equiv="x-pjax-version" content="339e4a7c6efc07cb8134de61f966c2c7">
    

      
  <meta name="description" content="This is a read-only mirror of the Bioconductor SVN repository. Package Homepage: http://bioconductor.org/packages/devel/bioc/html/NOISeq.html Bug Reports: https://support.bioconductor.org/p/new/post/?tag_val=NOISeq.">
  <meta name="go-import" content="github.com/Bioconductor-mirror/NOISeq git https://github.com/Bioconductor-mirror/NOISeq.git">

  <meta content="11679331" name="octolytics-dimension-user_id" /><meta content="Bioconductor-mirror" name="octolytics-dimension-user_login" /><meta content="32989734" name="octolytics-dimension-repository_id" /><meta content="Bioconductor-mirror/NOISeq" name="octolytics-dimension-repository_nwo" /><meta content="true" name="octolytics-dimension-repository_public" /><meta content="false" name="octolytics-dimension-repository_is_fork" /><meta content="32989734" name="octolytics-dimension-repository_network_root_id" /><meta content="Bioconductor-mirror/NOISeq" name="octolytics-dimension-repository_network_root_nwo" />
  <link href="https://github.com/Bioconductor-mirror/NOISeq/commits/master.atom" rel="alternate" title="Recent Commits to NOISeq:master" type="application/atom+xml">


      <link rel="canonical" href="https://github.com/Bioconductor-mirror/NOISeq/blob/master/vignettes/NOISeq.Rnw" data-pjax-transient>
  </head>


  <body class="logged-in  env-production macintosh vis-public page-blob">
    <div id="js-pjax-loader-bar" class="pjax-loader-bar"><div class="progress"></div></div>
    <a href="#start-of-content" tabindex="1" class="accessibility-aid js-skip-to-content">Skip to content</a>

    
    
    



        <div class="header header-logged-in true" role="banner">
  <div class="container clearfix">

    <a class="header-logo-invertocat" href="https://github.com/" data-hotkey="g d" aria-label="Homepage" data-ga-click="Header, go to dashboard, icon:logo">
  <svg aria-hidden="true" class="octicon octicon-mark-github" height="28" version="1.1" viewBox="0 0 16 16" width="28"><path fill-rule="evenodd" d="M8 0C3.58 0 0 3.58 0 8c0 3.54 2.29 6.53 5.47 7.59.4.07.55-.17.55-.38 0-.19-.01-.82-.01-1.49-2.01.37-2.53-.49-2.69-.94-.09-.23-.48-.94-.82-1.13-.28-.15-.68-.52-.01-.53.63-.01 1.08.58 1.23.82.72 1.21 1.87.87 2.33.66.07-.52.28-.87.51-1.07-1.78-.2-3.64-.89-3.64-3.95 0-.87.31-1.59.82-2.15-.08-.2-.36-1.02.08-2.12 0 0 .67-.21 2.2.82.64-.18 1.32-.27 2-.27.68 0 1.36.09 2 .27 1.53-1.04 2.2-.82 2.2-.82.44 1.1.16 1.92.08 2.12.51.56.82 1.27.82 2.15 0 3.07-1.87 3.75-3.65 3.95.29.25.54.73.54 1.48 0 1.07-.01 1.93-.01 2.2 0 .21.15.46.55.38A8.013 8.013 0 0 0 16 8c0-4.42-3.58-8-8-8z"/></svg>
</a>


        <div class="header-search scoped-search site-scoped-search js-site-search" role="search">
  <!-- '"` --><!-- </textarea></xmp> --></option></form><form accept-charset="UTF-8" action="/Bioconductor-mirror/NOISeq/search" class="js-site-search-form" data-scoped-search-url="/Bioconductor-mirror/NOISeq/search" data-unscoped-search-url="/search" method="get"><div style="margin:0;padding:0;display:inline"><input name="utf8" type="hidden" value="&#x2713;" /></div>
    <label class="form-control header-search-wrapper js-chromeless-input-container">
      <div class="header-search-scope">This repository</div>
      <input type="text"
        class="form-control header-search-input js-site-search-focus js-site-search-field is-clearable"
        data-hotkey="s"
        name="q"
        placeholder="Search"
        aria-label="Search this repository"
        data-unscoped-placeholder="Search GitHub"
        data-scoped-placeholder="Search"
        autocapitalize="off">
    </label>
</form></div>


      <ul class="header-nav float-left" role="navigation">
        <li class="header-nav-item">
          <a href="/pulls" aria-label="Pull requests you created" class="js-selected-navigation-item header-nav-link" data-ga-click="Header, click, Nav menu - item:pulls context:user" data-hotkey="g p" data-selected-links="/pulls /pulls/assigned /pulls/mentioned /pulls">
            Pull requests
</a>        </li>
        <li class="header-nav-item">
          <a href="/issues" aria-label="Issues you created" class="js-selected-navigation-item header-nav-link" data-ga-click="Header, click, Nav menu - item:issues context:user" data-hotkey="g i" data-selected-links="/issues /issues/assigned /issues/mentioned /issues">
            Issues
</a>        </li>
          <li class="header-nav-item">
            <a class="header-nav-link" href="https://gist.github.com/" data-ga-click="Header, go to gist, text:gist">Gist</a>
          </li>
      </ul>

    
<ul class="header-nav user-nav float-right" id="user-links">
  <li class="header-nav-item">
    
    <a href="/notifications" aria-label="You have unread notifications" class="header-nav-link notification-indicator tooltipped tooltipped-s js-socket-channel js-notification-indicator" data-channel="tenant:1:notification-changed:3151963" data-ga-click="Header, go to notifications, icon:unread" data-hotkey="g n">
        <span class="mail-status unread"></span>
        <svg aria-hidden="true" class="octicon octicon-bell" height="16" version="1.1" viewBox="0 0 14 16" width="14"><path fill-rule="evenodd" d="M14 12v1H0v-1l.73-.58c.77-.77.81-2.55 1.19-4.42C2.69 3.23 6 2 6 2c0-.55.45-1 1-1s1 .45 1 1c0 0 3.39 1.23 4.16 5 .38 1.88.42 3.66 1.19 4.42l.66.58H14zm-7 4c1.11 0 2-.89 2-2H5c0 1.11.89 2 2 2z"/></svg>
</a>
  </li>

  <li class="header-nav-item dropdown js-menu-container">
    <a class="header-nav-link tooltipped tooltipped-s js-menu-target" href="/new"
       aria-label="Create new…"
       data-ga-click="Header, create new, icon:add">
      <svg aria-hidden="true" class="octicon octicon-plus float-left" height="16" version="1.1" viewBox="0 0 12 16" width="12"><path fill-rule="evenodd" d="M12 9H7v5H5V9H0V7h5V2h2v5h5z"/></svg>
      <span class="dropdown-caret"></span>
    </a>

    <div class="dropdown-menu-content js-menu-content">
      <ul class="dropdown-menu dropdown-menu-sw">
        
<a class="dropdown-item" href="/new" data-ga-click="Header, create new repository">
  New repository
</a>

  <a class="dropdown-item" href="/new/import" data-ga-click="Header, import a repository">
    Import repository
  </a>

<a class="dropdown-item" href="https://gist.github.com/" data-ga-click="Header, create new gist">
  New gist
</a>

  <a class="dropdown-item" href="/organizations/new" data-ga-click="Header, create new organization">
    New organization
  </a>




      </ul>
    </div>
  </li>

  <li class="header-nav-item dropdown js-menu-container">
    <a class="header-nav-link name tooltipped tooltipped-sw js-menu-target" href="/hongqin"
       aria-label="View profile and more"
       data-ga-click="Header, show menu, icon:avatar">
      <img alt="@hongqin" class="avatar" height="20" src="https://avatars0.githubusercontent.com/u/3151963?v=3&amp;s=40" width="20" />
      <span class="dropdown-caret"></span>
    </a>

    <div class="dropdown-menu-content js-menu-content">
      <div class="dropdown-menu dropdown-menu-sw">
        <div class="dropdown-header header-nav-current-user css-truncate">
          Signed in as <strong class="css-truncate-target">hongqin</strong>
        </div>

        <div class="dropdown-divider"></div>

        <a class="dropdown-item" href="/hongqin" data-ga-click="Header, go to profile, text:your profile">
          Your profile
        </a>
        <a class="dropdown-item" href="/hongqin?tab=stars" data-ga-click="Header, go to starred repos, text:your stars">
          Your stars
        </a>
        <a class="dropdown-item" href="/explore" data-ga-click="Header, go to explore, text:explore">
          Explore
        </a>
          <a class="dropdown-item" href="/integrations" data-ga-click="Header, go to integrations, text:integrations">
            Integrations
          </a>
        <a class="dropdown-item" href="https://help.github.com" data-ga-click="Header, go to help, text:help">
          Help
        </a>

        <div class="dropdown-divider"></div>

        <a class="dropdown-item" href="/settings/profile" data-ga-click="Header, go to settings, icon:settings">
          Settings
        </a>

        <!-- '"` --><!-- </textarea></xmp> --></option></form><form accept-charset="UTF-8" action="/logout" class="logout-form" data-form-nonce="f316d99f31dced27cd3d76033417f957c977fe36" method="post"><div style="margin:0;padding:0;display:inline"><input name="utf8" type="hidden" value="&#x2713;" /><input name="authenticity_token" type="hidden" value="tE0atH6aTXtcPt+vkLGunMmxe8J1MDlm5vmg37ZNm3bDwfk+27y/9PIxMRP1J/iQujtGyC3SxT9lrJQgU/tttQ==" /></div>
          <button type="submit" class="dropdown-item dropdown-signout" data-ga-click="Header, sign out, icon:logout">
            Sign out
          </button>
</form>      </div>
    </div>
  </li>
</ul>


    
  </div>
</div>


      


    <div id="start-of-content" class="accessibility-aid"></div>

      <div id="js-flash-container">
</div>


    <div role="main">
        <div itemscope itemtype="http://schema.org/SoftwareSourceCode">
    <div id="js-repo-pjax-container" data-pjax-container>
      
<div class="pagehead repohead instapaper_ignore readability-menu experiment-repo-nav">
  <div class="container repohead-details-container">

    

<ul class="pagehead-actions">

  <li>
        <!-- '"` --><!-- </textarea></xmp> --></option></form><form accept-charset="UTF-8" action="/notifications/subscribe" class="js-social-container" data-autosubmit="true" data-form-nonce="f316d99f31dced27cd3d76033417f957c977fe36" data-remote="true" method="post"><div style="margin:0;padding:0;display:inline"><input name="utf8" type="hidden" value="&#x2713;" /><input name="authenticity_token" type="hidden" value="YmY8n2HVUTkqarCDXeWC/fm550jFYMUNgfMUQnQGS/EV6t8VxPOjtoRlXj84c9TxijPaQp2COVQCpiC9kbC9Mg==" /></div>      <input class="form-control" id="repository_id" name="repository_id" type="hidden" value="32989734" />

        <div class="select-menu js-menu-container js-select-menu">
          <a href="/Bioconductor-mirror/NOISeq/subscription"
            class="btn btn-sm btn-with-count select-menu-button js-menu-target" role="button" tabindex="0" aria-haspopup="true"
            data-ga-click="Repository, click Watch settings, action:blob#show">
            <span class="js-select-button">
              <svg aria-hidden="true" class="octicon octicon-eye" height="16" version="1.1" viewBox="0 0 16 16" width="16"><path fill-rule="evenodd" d="M8.06 2C3 2 0 8 0 8s3 6 8.06 6C13 14 16 8 16 8s-3-6-7.94-6zM8 12c-2.2 0-4-1.78-4-4 0-2.2 1.8-4 4-4 2.22 0 4 1.8 4 4 0 2.22-1.78 4-4 4zm2-4c0 1.11-.89 2-2 2-1.11 0-2-.89-2-2 0-1.11.89-2 2-2 1.11 0 2 .89 2 2z"/></svg>
              Watch
            </span>
          </a>
          <a class="social-count js-social-count"
            href="/Bioconductor-mirror/NOISeq/watchers"
            aria-label="3 users are watching this repository">
            3
          </a>

        <div class="select-menu-modal-holder">
          <div class="select-menu-modal subscription-menu-modal js-menu-content" aria-hidden="true">
            <div class="select-menu-header js-navigation-enable" tabindex="-1">
              <svg aria-label="Close" class="octicon octicon-x js-menu-close" height="16" role="img" version="1.1" viewBox="0 0 12 16" width="12"><path fill-rule="evenodd" d="M7.48 8l3.75 3.75-1.48 1.48L6 9.48l-3.75 3.75-1.48-1.48L4.52 8 .77 4.25l1.48-1.48L6 6.52l3.75-3.75 1.48 1.48z"/></svg>
              <span class="select-menu-title">Notifications</span>
            </div>

              <div class="select-menu-list js-navigation-container" role="menu">

                <div class="select-menu-item js-navigation-item selected" role="menuitem" tabindex="0">
                  <svg aria-hidden="true" class="octicon octicon-check select-menu-item-icon" height="16" version="1.1" viewBox="0 0 12 16" width="12"><path fill-rule="evenodd" d="M12 5l-8 8-4-4 1.5-1.5L4 10l6.5-6.5z"/></svg>
                  <div class="select-menu-item-text">
                    <input checked="checked" id="do_included" name="do" type="radio" value="included" />
                    <span class="select-menu-item-heading">Not watching</span>
                    <span class="description">Be notified when participating or @mentioned.</span>
                    <span class="js-select-button-text hidden-select-button-text">
                      <svg aria-hidden="true" class="octicon octicon-eye" height="16" version="1.1" viewBox="0 0 16 16" width="16"><path fill-rule="evenodd" d="M8.06 2C3 2 0 8 0 8s3 6 8.06 6C13 14 16 8 16 8s-3-6-7.94-6zM8 12c-2.2 0-4-1.78-4-4 0-2.2 1.8-4 4-4 2.22 0 4 1.8 4 4 0 2.22-1.78 4-4 4zm2-4c0 1.11-.89 2-2 2-1.11 0-2-.89-2-2 0-1.11.89-2 2-2 1.11 0 2 .89 2 2z"/></svg>
                      Watch
                    </span>
                  </div>
                </div>

                <div class="select-menu-item js-navigation-item " role="menuitem" tabindex="0">
                  <svg aria-hidden="true" class="octicon octicon-check select-menu-item-icon" height="16" version="1.1" viewBox="0 0 12 16" width="12"><path fill-rule="evenodd" d="M12 5l-8 8-4-4 1.5-1.5L4 10l6.5-6.5z"/></svg>
                  <div class="select-menu-item-text">
                    <input id="do_subscribed" name="do" type="radio" value="subscribed" />
                    <span class="select-menu-item-heading">Watching</span>
                    <span class="description">Be notified of all conversations.</span>
                    <span class="js-select-button-text hidden-select-button-text">
                      <svg aria-hidden="true" class="octicon octicon-eye" height="16" version="1.1" viewBox="0 0 16 16" width="16"><path fill-rule="evenodd" d="M8.06 2C3 2 0 8 0 8s3 6 8.06 6C13 14 16 8 16 8s-3-6-7.94-6zM8 12c-2.2 0-4-1.78-4-4 0-2.2 1.8-4 4-4 2.22 0 4 1.8 4 4 0 2.22-1.78 4-4 4zm2-4c0 1.11-.89 2-2 2-1.11 0-2-.89-2-2 0-1.11.89-2 2-2 1.11 0 2 .89 2 2z"/></svg>
                      Unwatch
                    </span>
                  </div>
                </div>

                <div class="select-menu-item js-navigation-item " role="menuitem" tabindex="0">
                  <svg aria-hidden="true" class="octicon octicon-check select-menu-item-icon" height="16" version="1.1" viewBox="0 0 12 16" width="12"><path fill-rule="evenodd" d="M12 5l-8 8-4-4 1.5-1.5L4 10l6.5-6.5z"/></svg>
                  <div class="select-menu-item-text">
                    <input id="do_ignore" name="do" type="radio" value="ignore" />
                    <span class="select-menu-item-heading">Ignoring</span>
                    <span class="description">Never be notified.</span>
                    <span class="js-select-button-text hidden-select-button-text">
                      <svg aria-hidden="true" class="octicon octicon-mute" height="16" version="1.1" viewBox="0 0 16 16" width="16"><path fill-rule="evenodd" d="M8 2.81v10.38c0 .67-.81 1-1.28.53L3 10H1c-.55 0-1-.45-1-1V7c0-.55.45-1 1-1h2l3.72-3.72C7.19 1.81 8 2.14 8 2.81zm7.53 3.22l-1.06-1.06-1.97 1.97-1.97-1.97-1.06 1.06L11.44 8 9.47 9.97l1.06 1.06 1.97-1.97 1.97 1.97 1.06-1.06L13.56 8l1.97-1.97z"/></svg>
                      Stop ignoring
                    </span>
                  </div>
                </div>

              </div>

            </div>
          </div>
        </div>
</form>
  </li>

  <li>
    
  <div class="js-toggler-container js-social-container starring-container ">

    <!-- '"` --><!-- </textarea></xmp> --></option></form><form accept-charset="UTF-8" action="/Bioconductor-mirror/NOISeq/unstar" class="starred" data-form-nonce="f316d99f31dced27cd3d76033417f957c977fe36" data-remote="true" method="post"><div style="margin:0;padding:0;display:inline"><input name="utf8" type="hidden" value="&#x2713;" /><input name="authenticity_token" type="hidden" value="OkNAQzXR4A1I4/ZH4FI75cjCHzqdaW1tfrKmVfL//YhNz6PJkPcSgubsGPuFxG3pu0giMMWLkTT955KqF0kLSw==" /></div>
      <button
        type="submit"
        class="btn btn-sm btn-with-count js-toggler-target"
        aria-label="Unstar this repository" title="Unstar Bioconductor-mirror/NOISeq"
        data-ga-click="Repository, click unstar button, action:blob#show; text:Unstar">
        <svg aria-hidden="true" class="octicon octicon-star" height="16" version="1.1" viewBox="0 0 14 16" width="14"><path fill-rule="evenodd" d="M14 6l-4.9-.64L7 1 4.9 5.36 0 6l3.6 3.26L2.67 14 7 11.67 11.33 14l-.93-4.74z"/></svg>
        Unstar
      </button>
        <a class="social-count js-social-count" href="/Bioconductor-mirror/NOISeq/stargazers"
           aria-label="0 users starred this repository">
          0
        </a>
</form>
    <!-- '"` --><!-- </textarea></xmp> --></option></form><form accept-charset="UTF-8" action="/Bioconductor-mirror/NOISeq/star" class="unstarred" data-form-nonce="f316d99f31dced27cd3d76033417f957c977fe36" data-remote="true" method="post"><div style="margin:0;padding:0;display:inline"><input name="utf8" type="hidden" value="&#x2713;" /><input name="authenticity_token" type="hidden" value="9Kw1piOpEm+an8w1Keoqsufl9vEUaxHXdUnQgzNMEYCDINYsho/g4DSQIolMfHy+lG/L+0yJ7Y72HOR81vrnQw==" /></div>
      <button
        type="submit"
        class="btn btn-sm btn-with-count js-toggler-target"
        aria-label="Star this repository" title="Star Bioconductor-mirror/NOISeq"
        data-ga-click="Repository, click star button, action:blob#show; text:Star">
        <svg aria-hidden="true" class="octicon octicon-star" height="16" version="1.1" viewBox="0 0 14 16" width="14"><path fill-rule="evenodd" d="M14 6l-4.9-.64L7 1 4.9 5.36 0 6l3.6 3.26L2.67 14 7 11.67 11.33 14l-.93-4.74z"/></svg>
        Star
      </button>
        <a class="social-count js-social-count" href="/Bioconductor-mirror/NOISeq/stargazers"
           aria-label="0 users starred this repository">
          0
        </a>
</form>  </div>

  </li>

  <li>
          <a href="#fork-destination-box" class="btn btn-sm btn-with-count"
              title="Fork your own copy of Bioconductor-mirror/NOISeq to your account"
              aria-label="Fork your own copy of Bioconductor-mirror/NOISeq to your account"
              rel="facebox"
              data-ga-click="Repository, show fork modal, action:blob#show; text:Fork">
              <svg aria-hidden="true" class="octicon octicon-repo-forked" height="16" version="1.1" viewBox="0 0 10 16" width="10"><path fill-rule="evenodd" d="M8 1a1.993 1.993 0 0 0-1 3.72V6L5 8 3 6V4.72A1.993 1.993 0 0 0 2 1a1.993 1.993 0 0 0-1 3.72V6.5l3 3v1.78A1.993 1.993 0 0 0 5 15a1.993 1.993 0 0 0 1-3.72V9.5l3-3V4.72A1.993 1.993 0 0 0 8 1zM2 4.2C1.34 4.2.8 3.65.8 3c0-.65.55-1.2 1.2-1.2.65 0 1.2.55 1.2 1.2 0 .65-.55 1.2-1.2 1.2zm3 10c-.66 0-1.2-.55-1.2-1.2 0-.65.55-1.2 1.2-1.2.65 0 1.2.55 1.2 1.2 0 .65-.55 1.2-1.2 1.2zm3-10c-.66 0-1.2-.55-1.2-1.2 0-.65.55-1.2 1.2-1.2.65 0 1.2.55 1.2 1.2 0 .65-.55 1.2-1.2 1.2z"/></svg>
            Fork
          </a>

          <div id="fork-destination-box" style="display: none;">
            <h2 class="facebox-header" data-facebox-id="facebox-header">Where should we fork this repository?</h2>
            <include-fragment src=""
                class="js-fork-select-fragment fork-select-fragment"
                data-url="/Bioconductor-mirror/NOISeq/fork?fragment=1">
              <img alt="Loading" height="64" src="https://assets-cdn.github.com/images/spinners/octocat-spinner-128.gif" width="64" />
            </include-fragment>
          </div>

    <a href="/Bioconductor-mirror/NOISeq/network" class="social-count"
       aria-label="0 users forked this repository">
      0
    </a>
  </li>
</ul>

    <h1 class="public ">
  <svg aria-hidden="true" class="octicon octicon-repo" height="16" version="1.1" viewBox="0 0 12 16" width="12"><path fill-rule="evenodd" d="M4 9H3V8h1v1zm0-3H3v1h1V6zm0-2H3v1h1V4zm0-2H3v1h1V2zm8-1v12c0 .55-.45 1-1 1H6v2l-1.5-1.5L3 16v-2H1c-.55 0-1-.45-1-1V1c0-.55.45-1 1-1h10c.55 0 1 .45 1 1zm-1 10H1v2h2v-1h3v1h5v-2zm0-10H2v9h9V1z"/></svg>
  <span class="author" itemprop="author"><a href="/Bioconductor-mirror" class="url fn" rel="author">Bioconductor-mirror</a></span><!--
--><span class="path-divider">/</span><!--
--><strong itemprop="name"><a href="/Bioconductor-mirror/NOISeq" data-pjax="#js-repo-pjax-container">NOISeq</a></strong>

</h1>

  </div>
  <div class="container">
    
<nav class="reponav js-repo-nav js-sidenav-container-pjax"
     itemscope
     itemtype="http://schema.org/BreadcrumbList"
     role="navigation"
     data-pjax="#js-repo-pjax-container">

  <span itemscope itemtype="http://schema.org/ListItem" itemprop="itemListElement">
    <a href="/Bioconductor-mirror/NOISeq" class="js-selected-navigation-item selected reponav-item" data-hotkey="g c" data-selected-links="repo_source repo_downloads repo_commits repo_releases repo_tags repo_branches /Bioconductor-mirror/NOISeq" itemprop="url">
      <svg aria-hidden="true" class="octicon octicon-code" height="16" version="1.1" viewBox="0 0 14 16" width="14"><path fill-rule="evenodd" d="M9.5 3L8 4.5 11.5 8 8 11.5 9.5 13 14 8 9.5 3zm-5 0L0 8l4.5 5L6 11.5 2.5 8 6 4.5 4.5 3z"/></svg>
      <span itemprop="name">Code</span>
      <meta itemprop="position" content="1">
</a>  </span>


  <span itemscope itemtype="http://schema.org/ListItem" itemprop="itemListElement">
    <a href="/Bioconductor-mirror/NOISeq/pulls" class="js-selected-navigation-item reponav-item" data-hotkey="g p" data-selected-links="repo_pulls /Bioconductor-mirror/NOISeq/pulls" itemprop="url">
      <svg aria-hidden="true" class="octicon octicon-git-pull-request" height="16" version="1.1" viewBox="0 0 12 16" width="12"><path fill-rule="evenodd" d="M11 11.28V5c-.03-.78-.34-1.47-.94-2.06C9.46 2.35 8.78 2.03 8 2H7V0L4 3l3 3V4h1c.27.02.48.11.69.31.21.2.3.42.31.69v6.28A1.993 1.993 0 0 0 10 15a1.993 1.993 0 0 0 1-3.72zm-1 2.92c-.66 0-1.2-.55-1.2-1.2 0-.65.55-1.2 1.2-1.2.65 0 1.2.55 1.2 1.2 0 .65-.55 1.2-1.2 1.2zM4 3c0-1.11-.89-2-2-2a1.993 1.993 0 0 0-1 3.72v6.56A1.993 1.993 0 0 0 2 15a1.993 1.993 0 0 0 1-3.72V4.72c.59-.34 1-.98 1-1.72zm-.8 10c0 .66-.55 1.2-1.2 1.2-.65 0-1.2-.55-1.2-1.2 0-.65.55-1.2 1.2-1.2.65 0 1.2.55 1.2 1.2zM2 4.2C1.34 4.2.8 3.65.8 3c0-.65.55-1.2 1.2-1.2.65 0 1.2.55 1.2 1.2 0 .65-.55 1.2-1.2 1.2z"/></svg>
      <span itemprop="name">Pull requests</span>
      <span class="counter">0</span>
      <meta itemprop="position" content="3">
</a>  </span>

  <a href="/Bioconductor-mirror/NOISeq/projects" class="js-selected-navigation-item reponav-item" data-selected-links="repo_projects new_repo_project repo_project /Bioconductor-mirror/NOISeq/projects">
    <svg aria-hidden="true" class="octicon octicon-project" height="16" version="1.1" viewBox="0 0 15 16" width="15"><path fill-rule="evenodd" d="M10 12h3V2h-3v10zm-4-2h3V2H6v8zm-4 4h3V2H2v12zm-1 1h13V1H1v14zM14 0H1a1 1 0 0 0-1 1v14a1 1 0 0 0 1 1h13a1 1 0 0 0 1-1V1a1 1 0 0 0-1-1z"/></svg>
    Projects
    <span class="counter">0</span>
</a>


  <a href="/Bioconductor-mirror/NOISeq/pulse" class="js-selected-navigation-item reponav-item" data-selected-links="pulse /Bioconductor-mirror/NOISeq/pulse">
    <svg aria-hidden="true" class="octicon octicon-pulse" height="16" version="1.1" viewBox="0 0 14 16" width="14"><path fill-rule="evenodd" d="M11.5 8L8.8 5.4 6.6 8.5 5.5 1.6 2.38 8H0v2h3.6l.9-1.8.9 5.4L9 8.5l1.6 1.5H14V8z"/></svg>
    Pulse
</a>
  <a href="/Bioconductor-mirror/NOISeq/graphs" class="js-selected-navigation-item reponav-item" data-selected-links="repo_graphs repo_contributors /Bioconductor-mirror/NOISeq/graphs">
    <svg aria-hidden="true" class="octicon octicon-graph" height="16" version="1.1" viewBox="0 0 16 16" width="16"><path fill-rule="evenodd" d="M16 14v1H0V0h1v14h15zM5 13H3V8h2v5zm4 0H7V3h2v10zm4 0h-2V6h2v7z"/></svg>
    Graphs
</a>

</nav>

  </div>
</div>

<div class="container new-discussion-timeline experiment-repo-nav">
  <div class="repository-content">

    

<a href="/Bioconductor-mirror/NOISeq/blob/d2220661b752045969ba82e142c30de62adb5854/vignettes/NOISeq.Rnw" class="d-none js-permalink-shortcut" data-hotkey="y">Permalink</a>

<!-- blob contrib key: blob_contributors:v21:89bfc344bef4d2796152bcea5d5ab644 -->

<div class="file-navigation js-zeroclipboard-container">
  
<div class="select-menu branch-select-menu js-menu-container js-select-menu float-left">
  <button class="btn btn-sm select-menu-button js-menu-target css-truncate" data-hotkey="w"
    
    type="button" aria-label="Switch branches or tags" tabindex="0" aria-haspopup="true">
    <i>Branch:</i>
    <span class="js-select-button css-truncate-target">master</span>
  </button>

  <div class="select-menu-modal-holder js-menu-content js-navigation-container" data-pjax aria-hidden="true">

    <div class="select-menu-modal">
      <div class="select-menu-header">
        <svg aria-label="Close" class="octicon octicon-x js-menu-close" height="16" role="img" version="1.1" viewBox="0 0 12 16" width="12"><path fill-rule="evenodd" d="M7.48 8l3.75 3.75-1.48 1.48L6 9.48l-3.75 3.75-1.48-1.48L4.52 8 .77 4.25l1.48-1.48L6 6.52l3.75-3.75 1.48 1.48z"/></svg>
        <span class="select-menu-title">Switch branches/tags</span>
      </div>

      <div class="select-menu-filters">
        <div class="select-menu-text-filter">
          <input type="text" aria-label="Filter branches/tags" id="context-commitish-filter-field" class="form-control js-filterable-field js-navigation-enable" placeholder="Filter branches/tags">
        </div>
        <div class="select-menu-tabs">
          <ul>
            <li class="select-menu-tab">
              <a href="#" data-tab-filter="branches" data-filter-placeholder="Filter branches/tags" class="js-select-menu-tab" role="tab">Branches</a>
            </li>
            <li class="select-menu-tab">
              <a href="#" data-tab-filter="tags" data-filter-placeholder="Find a tag…" class="js-select-menu-tab" role="tab">Tags</a>
            </li>
          </ul>
        </div>
      </div>

      <div class="select-menu-list select-menu-tab-bucket js-select-menu-tab-bucket" data-tab-filter="branches" role="menu">

        <div data-filterable-for="context-commitish-filter-field" data-filterable-type="substring">


            <a class="select-menu-item js-navigation-item js-navigation-open selected"
               href="/Bioconductor-mirror/NOISeq/blob/master/vignettes/NOISeq.Rnw"
               data-name="master"
               data-skip-pjax="true"
               rel="nofollow">
              <svg aria-hidden="true" class="octicon octicon-check select-menu-item-icon" height="16" version="1.1" viewBox="0 0 12 16" width="12"><path fill-rule="evenodd" d="M12 5l-8 8-4-4 1.5-1.5L4 10l6.5-6.5z"/></svg>
              <span class="select-menu-item-text css-truncate-target js-select-menu-filter-text">
                master
              </span>
            </a>
            <a class="select-menu-item js-navigation-item js-navigation-open "
               href="/Bioconductor-mirror/NOISeq/blob/release-3.0/vignettes/NOISeq.Rnw"
               data-name="release-3.0"
               data-skip-pjax="true"
               rel="nofollow">
              <svg aria-hidden="true" class="octicon octicon-check select-menu-item-icon" height="16" version="1.1" viewBox="0 0 12 16" width="12"><path fill-rule="evenodd" d="M12 5l-8 8-4-4 1.5-1.5L4 10l6.5-6.5z"/></svg>
              <span class="select-menu-item-text css-truncate-target js-select-menu-filter-text">
                release-3.0
              </span>
            </a>
            <a class="select-menu-item js-navigation-item js-navigation-open "
               href="/Bioconductor-mirror/NOISeq/blob/release-3.1/vignettes/NOISeq.Rnw"
               data-name="release-3.1"
               data-skip-pjax="true"
               rel="nofollow">
              <svg aria-hidden="true" class="octicon octicon-check select-menu-item-icon" height="16" version="1.1" viewBox="0 0 12 16" width="12"><path fill-rule="evenodd" d="M12 5l-8 8-4-4 1.5-1.5L4 10l6.5-6.5z"/></svg>
              <span class="select-menu-item-text css-truncate-target js-select-menu-filter-text">
                release-3.1
              </span>
            </a>
            <a class="select-menu-item js-navigation-item js-navigation-open "
               href="/Bioconductor-mirror/NOISeq/blob/release-3.2/vignettes/NOISeq.Rnw"
               data-name="release-3.2"
               data-skip-pjax="true"
               rel="nofollow">
              <svg aria-hidden="true" class="octicon octicon-check select-menu-item-icon" height="16" version="1.1" viewBox="0 0 12 16" width="12"><path fill-rule="evenodd" d="M12 5l-8 8-4-4 1.5-1.5L4 10l6.5-6.5z"/></svg>
              <span class="select-menu-item-text css-truncate-target js-select-menu-filter-text">
                release-3.2
              </span>
            </a>
            <a class="select-menu-item js-navigation-item js-navigation-open "
               href="/Bioconductor-mirror/NOISeq/blob/release-3.3/vignettes/NOISeq.Rnw"
               data-name="release-3.3"
               data-skip-pjax="true"
               rel="nofollow">
              <svg aria-hidden="true" class="octicon octicon-check select-menu-item-icon" height="16" version="1.1" viewBox="0 0 12 16" width="12"><path fill-rule="evenodd" d="M12 5l-8 8-4-4 1.5-1.5L4 10l6.5-6.5z"/></svg>
              <span class="select-menu-item-text css-truncate-target js-select-menu-filter-text">
                release-3.3
              </span>
            </a>
            <a class="select-menu-item js-navigation-item js-navigation-open "
               href="/Bioconductor-mirror/NOISeq/blob/release-3.4/vignettes/NOISeq.Rnw"
               data-name="release-3.4"
               data-skip-pjax="true"
               rel="nofollow">
              <svg aria-hidden="true" class="octicon octicon-check select-menu-item-icon" height="16" version="1.1" viewBox="0 0 12 16" width="12"><path fill-rule="evenodd" d="M12 5l-8 8-4-4 1.5-1.5L4 10l6.5-6.5z"/></svg>
              <span class="select-menu-item-text css-truncate-target js-select-menu-filter-text">
                release-3.4
              </span>
            </a>
        </div>

          <div class="select-menu-no-results">Nothing to show</div>
      </div>

      <div class="select-menu-list select-menu-tab-bucket js-select-menu-tab-bucket" data-tab-filter="tags">
        <div data-filterable-for="context-commitish-filter-field" data-filterable-type="substring">


        </div>

        <div class="select-menu-no-results">Nothing to show</div>
      </div>

    </div>
  </div>
</div>

  <div class="BtnGroup float-right">
    <a href="/Bioconductor-mirror/NOISeq/find/master"
          class="js-pjax-capture-input btn btn-sm BtnGroup-item"
          data-pjax
          data-hotkey="t">
      Find file
    </a>
    <button aria-label="Copy file path to clipboard" class="js-zeroclipboard btn btn-sm BtnGroup-item tooltipped tooltipped-s" data-copied-hint="Copied!" type="button">Copy path</button>
  </div>
  <div class="breadcrumb js-zeroclipboard-target">
    <span class="repo-root js-repo-root"><span class="js-path-segment"><a href="/Bioconductor-mirror/NOISeq"><span>NOISeq</span></a></span></span><span class="separator">/</span><span class="js-path-segment"><a href="/Bioconductor-mirror/NOISeq/tree/master/vignettes"><span>vignettes</span></a></span><span class="separator">/</span><strong class="final-path">NOISeq.Rnw</strong>
  </div>
</div>


  <div class="commit-tease">
      <span class="float-right">
        <a class="commit-tease-sha" href="/Bioconductor-mirror/NOISeq/commit/78e4f6b9d5669058fb9b8263829fbf2ef6f51882" data-pjax>
          78e4f6b
        </a>
        <relative-time datetime="2016-02-11T11:59:38Z">Feb 11, 2016</relative-time>
      </span>
      <div>
        <img alt="" class="avatar" data-canonical-src="https://1.gravatar.com/avatar/40f09d2f16a4bf3f380c3864b55300db?d=https%3A%2F%2Fassets-cdn.github.com%2Fimages%2Fgravatars%2Fgravatar-user-420.png&amp;r=x&amp;s=140" height="20" src="https://camo.githubusercontent.com/92f861bf53d9c6c4ce5ffae0e6f6cebb75d647bf/68747470733a2f2f312e67726176617461722e636f6d2f6176617461722f34306630396432663136613462663366333830633338363462353533303064623f643d68747470732533412532462532466173736574732d63646e2e6769746875622e636f6d253246696d6167657325324667726176617461727325324667726176617461722d757365722d3432302e706e6726723d7826733d313430" width="20" />
        <span class="user-mention">s.tarazona</span>
          <a href="/Bioconductor-mirror/NOISeq/commit/78e4f6b9d5669058fb9b8263829fbf2ef6f51882" class="message" data-pjax="true" title="Major fixes applied for version 2.16.0

git-svn-id: https://hedgehog.fhcrc.org/bioconductor/trunk/madman/Rpacks/NOISeq@113521 bc3139a8-67e5-0310-9ffc-ced21a209358">Major fixes applied for version 2.16.0</a>
      </div>

    <div class="commit-tease-contributors">
      <button type="button" class="btn-link muted-link contributors-toggle" data-facebox="#blob_contributors_box">
        <strong>0</strong>
         contributors
      </button>
      
    </div>

    <div id="blob_contributors_box" style="display:none">
      <h2 class="facebox-header" data-facebox-id="facebox-header">Users who have contributed to this file</h2>
      <ul class="facebox-user-list" data-facebox-id="facebox-description">
      </ul>
    </div>
  </div>


<div class="file">
  <div class="file-header">
  <div class="file-actions">

    <div class="BtnGroup">
      <a href="/Bioconductor-mirror/NOISeq/raw/master/vignettes/NOISeq.Rnw" class="btn btn-sm BtnGroup-item" id="raw-url">Raw</a>
        <a href="/Bioconductor-mirror/NOISeq/blame/master/vignettes/NOISeq.Rnw" class="btn btn-sm js-update-url-with-hash BtnGroup-item">Blame</a>
      <a href="/Bioconductor-mirror/NOISeq/commits/master/vignettes/NOISeq.Rnw" class="btn btn-sm BtnGroup-item" rel="nofollow">History</a>
    </div>

        <a class="btn-octicon tooltipped tooltipped-nw"
           href="github-mac://openRepo/https://github.com/Bioconductor-mirror/NOISeq?branch=master&amp;filepath=vignettes%2FNOISeq.Rnw"
           aria-label="Open this file in GitHub Desktop"
           data-ga-click="Repository, open with desktop, type:mac">
            <svg aria-hidden="true" class="octicon octicon-device-desktop" height="16" version="1.1" viewBox="0 0 16 16" width="16"><path fill-rule="evenodd" d="M15 2H1c-.55 0-1 .45-1 1v9c0 .55.45 1 1 1h5.34c-.25.61-.86 1.39-2.34 2h8c-1.48-.61-2.09-1.39-2.34-2H15c.55 0 1-.45 1-1V3c0-.55-.45-1-1-1zm0 9H1V3h14v8z"/></svg>
        </a>

        <!-- '"` --><!-- </textarea></xmp> --></option></form><form accept-charset="UTF-8" action="/Bioconductor-mirror/NOISeq/edit/master/vignettes/NOISeq.Rnw" class="inline-form js-update-url-with-hash" data-form-nonce="f316d99f31dced27cd3d76033417f957c977fe36" method="post"><div style="margin:0;padding:0;display:inline"><input name="utf8" type="hidden" value="&#x2713;" /><input name="authenticity_token" type="hidden" value="K9/0fQ7mGkmKeXR93Jl2F63AX7hsdf5iqGKSGon1uUBcUxf3q8DoxiR2msG5DyAb3kpisjSXAjsrN6blbENPgw==" /></div>
          <button class="btn-octicon tooltipped tooltipped-nw" type="submit"
            aria-label="Fork this project and edit the file" data-hotkey="e" data-disable-with>
            <svg aria-hidden="true" class="octicon octicon-pencil" height="16" version="1.1" viewBox="0 0 14 16" width="14"><path fill-rule="evenodd" d="M0 12v3h3l8-8-3-3-8 8zm3 2H1v-2h1v1h1v1zm10.3-9.3L12 6 9 3l1.3-1.3a.996.996 0 0 1 1.41 0l1.59 1.59c.39.39.39 1.02 0 1.41z"/></svg>
          </button>
</form>        <!-- '"` --><!-- </textarea></xmp> --></option></form><form accept-charset="UTF-8" action="/Bioconductor-mirror/NOISeq/delete/master/vignettes/NOISeq.Rnw" class="inline-form" data-form-nonce="f316d99f31dced27cd3d76033417f957c977fe36" method="post"><div style="margin:0;padding:0;display:inline"><input name="utf8" type="hidden" value="&#x2713;" /><input name="authenticity_token" type="hidden" value="NvZjKi5Ju8M1TfjGB38nhpV5j8xDqHL5u++GhN/a/1NBeoCgi29JTJtCFnpi6XGK5vOyxhtKjqA4urJ7OmwJkA==" /></div>
          <button class="btn-octicon btn-octicon-danger tooltipped tooltipped-nw" type="submit"
            aria-label="Fork this project and delete the file" data-disable-with>
            <svg aria-hidden="true" class="octicon octicon-trashcan" height="16" version="1.1" viewBox="0 0 12 16" width="12"><path fill-rule="evenodd" d="M11 2H9c0-.55-.45-1-1-1H5c-.55 0-1 .45-1 1H2c-.55 0-1 .45-1 1v1c0 .55.45 1 1 1v9c0 .55.45 1 1 1h7c.55 0 1-.45 1-1V5c.55 0 1-.45 1-1V3c0-.55-.45-1-1-1zm-1 12H3V5h1v8h1V5h1v8h1V5h1v8h1V5h1v9zm1-10H2V3h9v1z"/></svg>
          </button>
</form>  </div>

  <div class="file-info">
      <span class="file-mode" title="File mode">executable file</span>
      <span class="file-info-divider"></span>
      1148 lines (818 sloc)
      <span class="file-info-divider"></span>
    65.5 KB
  </div>
</div>

  

  <div itemprop="text" class="blob-wrapper data type-text">
      <table class="highlight tab-size js-file-line-container" data-tab-size="8">
      <tr>
        <td id="L1" class="blob-num js-line-number" data-line-number="1"></td>
        <td id="LC1" class="blob-code blob-code-inner js-file-line">\documentclass[10pt]{article}</td>
      </tr>
      <tr>
        <td id="L2" class="blob-num js-line-number" data-line-number="2"></td>
        <td id="LC2" class="blob-code blob-code-inner js-file-line">\usepackage[a4paper,left=1.9cm,top=1.9cm,bottom=2.5cm,right=1.9cm,ignoreheadfoot]{geometry}</td>
      </tr>
      <tr>
        <td id="L3" class="blob-num js-line-number" data-line-number="3"></td>
        <td id="LC3" class="blob-code blob-code-inner js-file-line">\usepackage{cite}</td>
      </tr>
      <tr>
        <td id="L4" class="blob-num js-line-number" data-line-number="4"></td>
        <td id="LC4" class="blob-code blob-code-inner js-file-line">%\topmargin 0in</td>
      </tr>
      <tr>
        <td id="L5" class="blob-num js-line-number" data-line-number="5"></td>
        <td id="LC5" class="blob-code blob-code-inner js-file-line">%\headheight 0in</td>
      </tr>
      <tr>
        <td id="L6" class="blob-num js-line-number" data-line-number="6"></td>
        <td id="LC6" class="blob-code blob-code-inner js-file-line">%\headsep 0in</td>
      </tr>
      <tr>
        <td id="L7" class="blob-num js-line-number" data-line-number="7"></td>
        <td id="LC7" class="blob-code blob-code-inner js-file-line">%\oddsidemargin 0in</td>
      </tr>
      <tr>
        <td id="L8" class="blob-num js-line-number" data-line-number="8"></td>
        <td id="LC8" class="blob-code blob-code-inner js-file-line">%\evensidemargin 0in</td>
      </tr>
      <tr>
        <td id="L9" class="blob-num js-line-number" data-line-number="9"></td>
        <td id="LC9" class="blob-code blob-code-inner js-file-line">%\textwidth 176mm</td>
      </tr>
      <tr>
        <td id="L10" class="blob-num js-line-number" data-line-number="10"></td>
        <td id="LC10" class="blob-code blob-code-inner js-file-line">%\textheight 215mm</td>
      </tr>
      <tr>
        <td id="L11" class="blob-num js-line-number" data-line-number="11"></td>
        <td id="LC11" class="blob-code blob-code-inner js-file-line">\usepackage[numbers]{natbib}</td>
      </tr>
      <tr>
        <td id="L12" class="blob-num js-line-number" data-line-number="12"></td>
        <td id="LC12" class="blob-code blob-code-inner js-file-line">\usepackage{amsmath}</td>
      </tr>
      <tr>
        <td id="L13" class="blob-num js-line-number" data-line-number="13"></td>
        <td id="LC13" class="blob-code blob-code-inner js-file-line">\usepackage{amssymb}</td>
      </tr>
      <tr>
        <td id="L14" class="blob-num js-line-number" data-line-number="14"></td>
        <td id="LC14" class="blob-code blob-code-inner js-file-line">\usepackage{Sweave}</td>
      </tr>
      <tr>
        <td id="L15" class="blob-num js-line-number" data-line-number="15"></td>
        <td id="LC15" class="blob-code blob-code-inner js-file-line">\SweaveOpts{keep.source=FALSE,eps=FALSE,pdf=TRUE,png=FALSE,include=FALSE,concordance=TRUE}</td>
      </tr>
      <tr>
        <td id="L16" class="blob-num js-line-number" data-line-number="16"></td>
        <td id="LC16" class="blob-code blob-code-inner js-file-line">\usepackage{url}</td>
      </tr>
      <tr>
        <td id="L17" class="blob-num js-line-number" data-line-number="17"></td>
        <td id="LC17" class="blob-code blob-code-inner js-file-line">\usepackage[utf8]{inputenc}</td>
      </tr>
      <tr>
        <td id="L18" class="blob-num js-line-number" data-line-number="18"></td>
        <td id="LC18" class="blob-code blob-code-inner js-file-line">%\DeclareGraphicsRule{.tif}{png}{.png}{`convert #1 `dirname #1`/`basename #1 .tif`.png}</td>
      </tr>
      <tr>
        <td id="L19" class="blob-num js-line-number" data-line-number="19"></td>
        <td id="LC19" class="blob-code blob-code-inner js-file-line">\newcommand{\noiseq}{\textsf{NOISeq}}</td>
      </tr>
      <tr>
        <td id="L20" class="blob-num js-line-number" data-line-number="20"></td>
        <td id="LC20" class="blob-code blob-code-inner js-file-line">\newcommand{\noiseqbio}{\textsf{NOISeqBIO}}</td>
      </tr>
      <tr>
        <td id="L21" class="blob-num js-line-number" data-line-number="21"></td>
        <td id="LC21" class="blob-code blob-code-inner js-file-line">\newcommand{\code}[1]{{\small\texttt{#1}}}</td>
      </tr>
      <tr>
        <td id="L22" class="blob-num js-line-number" data-line-number="22"></td>
        <td id="LC22" class="blob-code blob-code-inner js-file-line">\newcommand{\R}{\textsf{R}}</td>
      </tr>
      <tr>
        <td id="L23" class="blob-num js-line-number" data-line-number="23"></td>
        <td id="LC23" class="blob-code blob-code-inner js-file-line">
</td>
      </tr>
      <tr>
        <td id="L24" class="blob-num js-line-number" data-line-number="24"></td>
        <td id="LC24" class="blob-code blob-code-inner js-file-line">\begin{document}</td>
\Sconcordance{concordance:NOISeq.tex:NOISeq.rnw:%
1 5661 1}

      </tr>
      <tr>
        <td id="L25" class="blob-num js-line-number" data-line-number="25"></td>
        <td id="LC25" class="blob-code blob-code-inner js-file-line">
</td>
      </tr>
      <tr>
        <td id="L26" class="blob-num js-line-number" data-line-number="26"></td>
        <td id="LC26" class="blob-code blob-code-inner js-file-line">%\VignetteIndexEntry{NOISeq User&#39;s Guide}</td>
      </tr>
      <tr>
        <td id="L27" class="blob-num js-line-number" data-line-number="27"></td>
        <td id="LC27" class="blob-code blob-code-inner js-file-line">
</td>
      </tr>
      <tr>
        <td id="L28" class="blob-num js-line-number" data-line-number="28"></td>
        <td id="LC28" class="blob-code blob-code-inner js-file-line">
</td>
      </tr>
      <tr>
        <td id="L29" class="blob-num js-line-number" data-line-number="29"></td>
        <td id="LC29" class="blob-code blob-code-inner js-file-line">\title{\noiseq: Differential Expression in \textsf{RNA-seq}}</td>
      </tr>
      <tr>
        <td id="L30" class="blob-num js-line-number" data-line-number="30"></td>
        <td id="LC30" class="blob-code blob-code-inner js-file-line">\author{Sonia Tarazona (\texttt{starazona@cipf.es})\\Pedro Furi\&#39;{o}-Tar\&#39;{i} (\texttt{pfurio@cipf.es})\\</td>
      </tr>
      <tr>
        <td id="L31" class="blob-num js-line-number" data-line-number="31"></td>
        <td id="LC31" class="blob-code blob-code-inner js-file-line">Mar\&#39;{i}a Jos\&#39;{e} Nueda (\texttt{mj.nueda@ua.es})\\Alberto Ferrer (\texttt{aferrer@eio.upv.es})\\Ana Conesa (\texttt{aconesa@cipf.es})}</td>
      </tr>
      <tr>
        <td id="L32" class="blob-num js-line-number" data-line-number="32"></td>
        <td id="LC32" class="blob-code blob-code-inner js-file-line">% Please increment date when working on this document, so that</td>
      </tr>
      <tr>
        <td id="L33" class="blob-num js-line-number" data-line-number="33"></td>
        <td id="LC33" class="blob-code blob-code-inner js-file-line">% date shows genuine change date, not merely date of compile.</td>
      </tr>
      <tr>
        <td id="L34" class="blob-num js-line-number" data-line-number="34"></td>
        <td id="LC34" class="blob-code blob-code-inner js-file-line">\date{11 February 2016 \\(Version 2.16.0)}</td>
      </tr>
      <tr>
        <td id="L35" class="blob-num js-line-number" data-line-number="35"></td>
        <td id="LC35" class="blob-code blob-code-inner js-file-line">\maketitle</td>
      </tr>
      <tr>
        <td id="L36" class="blob-num js-line-number" data-line-number="36"></td>
        <td id="LC36" class="blob-code blob-code-inner js-file-line">
</td>
      </tr>
      <tr>
        <td id="L37" class="blob-num js-line-number" data-line-number="37"></td>
        <td id="LC37" class="blob-code blob-code-inner js-file-line">
</td>
      </tr>
      <tr>
        <td id="L38" class="blob-num js-line-number" data-line-number="38"></td>
        <td id="LC38" class="blob-code blob-code-inner js-file-line">\tableofcontents</td>
      </tr>
      <tr>
        <td id="L39" class="blob-num js-line-number" data-line-number="39"></td>
        <td id="LC39" class="blob-code blob-code-inner js-file-line">
</td>
      </tr>
      <tr>
        <td id="L40" class="blob-num js-line-number" data-line-number="40"></td>
        <td id="LC40" class="blob-code blob-code-inner js-file-line">\clearpage</td>
      </tr>
      <tr>
        <td id="L41" class="blob-num js-line-number" data-line-number="41"></td>
        <td id="LC41" class="blob-code blob-code-inner js-file-line">
</td>
      </tr>
      <tr>
        <td id="L42" class="blob-num js-line-number" data-line-number="42"></td>
        <td id="LC42" class="blob-code blob-code-inner js-file-line">&lt;&lt;options,results=hide,echo=FALSE&gt;&gt;=</td>
      </tr>
      <tr>
        <td id="L43" class="blob-num js-line-number" data-line-number="43"></td>
        <td id="LC43" class="blob-code blob-code-inner js-file-line">options(digits=3, width=95)</td>
      </tr>
      <tr>
        <td id="L44" class="blob-num js-line-number" data-line-number="44"></td>
        <td id="LC44" class="blob-code blob-code-inner js-file-line">@</td>
      </tr>
      <tr>
        <td id="L45" class="blob-num js-line-number" data-line-number="45"></td>
        <td id="LC45" class="blob-code blob-code-inner js-file-line">
</td>
      </tr>
      <tr>
        <td id="L46" class="blob-num js-line-number" data-line-number="46"></td>
        <td id="LC46" class="blob-code blob-code-inner js-file-line">\section{Introduction}</td>
      </tr>
      <tr>
        <td id="L47" class="blob-num js-line-number" data-line-number="47"></td>
        <td id="LC47" class="blob-code blob-code-inner js-file-line">
</td>
      </tr>
      <tr>
        <td id="L48" class="blob-num js-line-number" data-line-number="48"></td>
        <td id="LC48" class="blob-code blob-code-inner js-file-line">This document will guide you through to the use of the \R{} Bioconductor package \noiseq{},</td>
      </tr>
      <tr>
        <td id="L49" class="blob-num js-line-number" data-line-number="49"></td>
        <td id="LC49" class="blob-code blob-code-inner js-file-line">for analyzing count data coming from next generation sequencing technologies. </td>
      </tr>
      <tr>
        <td id="L50" class="blob-num js-line-number" data-line-number="50"></td>
        <td id="LC50" class="blob-code blob-code-inner js-file-line">\noiseq{} package consists of three modules: (1) Quality control of count data; (2) Normalization and low-count filtering; and (3) Differential expression analysis.</td>
      </tr>
      <tr>
        <td id="L51" class="blob-num js-line-number" data-line-number="51"></td>
        <td id="LC51" class="blob-code blob-code-inner js-file-line">
</td>
      </tr>
      <tr>
        <td id="L52" class="blob-num js-line-number" data-line-number="52"></td>
        <td id="LC52" class="blob-code blob-code-inner js-file-line">First, we describe the input data format. Next, we illustrate the utilities to explore the quality of the count data: saturation, biases, </td>
      </tr>
      <tr>
        <td id="L53" class="blob-num js-line-number" data-line-number="53"></td>
        <td id="LC53" class="blob-code blob-code-inner js-file-line">contamination, etc. and show the normalization, filtering and batch correction methods included in the package. Finally, we explain how to compute differential </td>
      </tr>
      <tr>
        <td id="L54" class="blob-num js-line-number" data-line-number="54"></td>
        <td id="LC54" class="blob-code blob-code-inner js-file-line">expression between two experimental conditions. The differential expression method \noiseq{} and some of the plots included in the package were displayed in \cite{tarazona2011,tarazona2015}.The new version of \noiseq{} for biological replicates (\noiseqbio{}) is also implemented in the package.</td>
      </tr>
      <tr>
        <td id="L55" class="blob-num js-line-number" data-line-number="55"></td>
        <td id="LC55" class="blob-code blob-code-inner js-file-line">
</td>
      </tr>
      <tr>
        <td id="L56" class="blob-num js-line-number" data-line-number="56"></td>
        <td id="LC56" class="blob-code blob-code-inner js-file-line">The \noiseq{} and \noiseqbio{} methods are data-adaptive and nonparametric. </td>
      </tr>
      <tr>
        <td id="L57" class="blob-num js-line-number" data-line-number="57"></td>
        <td id="LC57" class="blob-code blob-code-inner js-file-line">Therefore, no distributional assumptions need to be done for the data</td>
      </tr>
      <tr>
        <td id="L58" class="blob-num js-line-number" data-line-number="58"></td>
        <td id="LC58" class="blob-code blob-code-inner js-file-line">and differential expression analysis may be carried on for both raw counts or previously normalized or transformed datasets. </td>
      </tr>
      <tr>
        <td id="L59" class="blob-num js-line-number" data-line-number="59"></td>
        <td id="LC59" class="blob-code blob-code-inner js-file-line">
</td>
      </tr>
      <tr>
        <td id="L60" class="blob-num js-line-number" data-line-number="60"></td>
        <td id="LC60" class="blob-code blob-code-inner js-file-line">We will use the ``reduced&#39;&#39; Marioni&#39;s dataset \cite{marioni2008} as an example throughout this document. In Marioni&#39;s experiment, </td>
      </tr>
      <tr>
        <td id="L61" class="blob-num js-line-number" data-line-number="61"></td>
        <td id="LC61" class="blob-code blob-code-inner js-file-line">human kidney and liver RNA-seq samples were sequenced. There are 5 technical replicates per tissue, </td>
      </tr>
      <tr>
        <td id="L62" class="blob-num js-line-number" data-line-number="62"></td>
        <td id="LC62" class="blob-code blob-code-inner js-file-line">and samples were sequenced in two different runs. </td>
      </tr>
      <tr>
        <td id="L63" class="blob-num js-line-number" data-line-number="63"></td>
        <td id="LC63" class="blob-code blob-code-inner js-file-line">We selected chromosomes I to IV from the original data and removed genes with 0 counts in all samples and with no length information available. </td>
      </tr>
      <tr>
        <td id="L64" class="blob-num js-line-number" data-line-number="64"></td>
        <td id="LC64" class="blob-code blob-code-inner js-file-line">Note that this reduced dataset is only used to decrease the computing time while testing the examples. We strongly recommend to use the whole set of features (e.g. the whole genome) in real analysis.</td>
      </tr>
      <tr>
        <td id="L65" class="blob-num js-line-number" data-line-number="65"></td>
        <td id="LC65" class="blob-code blob-code-inner js-file-line">
</td>
      </tr>
      <tr>
        <td id="L66" class="blob-num js-line-number" data-line-number="66"></td>
        <td id="LC66" class="blob-code blob-code-inner js-file-line">The example dataset can be obtained by typing:</td>
      </tr>
      <tr>
        <td id="L67" class="blob-num js-line-number" data-line-number="67"></td>
        <td id="LC67" class="blob-code blob-code-inner js-file-line">
</td>
      </tr>
      <tr>
        <td id="L68" class="blob-num js-line-number" data-line-number="68"></td>
        <td id="LC68" class="blob-code blob-code-inner js-file-line">&lt;&lt;data&gt;&gt;=</td>
      </tr>
      <tr>
        <td id="L69" class="blob-num js-line-number" data-line-number="69"></td>
        <td id="LC69" class="blob-code blob-code-inner js-file-line">library(NOISeq)</td>
      </tr>
      <tr>
        <td id="L70" class="blob-num js-line-number" data-line-number="70"></td>
        <td id="LC70" class="blob-code blob-code-inner js-file-line">data(Marioni)</td>
      </tr>
      <tr>
        <td id="L71" class="blob-num js-line-number" data-line-number="71"></td>
        <td id="LC71" class="blob-code blob-code-inner js-file-line">@  </td>
      </tr>
      <tr>
        <td id="L72" class="blob-num js-line-number" data-line-number="72"></td>
        <td id="LC72" class="blob-code blob-code-inner js-file-line">
</td>
      </tr>
      <tr>
        <td id="L73" class="blob-num js-line-number" data-line-number="73"></td>
        <td id="LC73" class="blob-code blob-code-inner js-file-line">
</td>
      </tr>
      <tr>
        <td id="L74" class="blob-num js-line-number" data-line-number="74"></td>
        <td id="LC74" class="blob-code blob-code-inner js-file-line">
</td>
      </tr>
      <tr>
        <td id="L75" class="blob-num js-line-number" data-line-number="75"></td>
        <td id="LC75" class="blob-code blob-code-inner js-file-line">\vspace{1cm}</td>
      </tr>
      <tr>
        <td id="L76" class="blob-num js-line-number" data-line-number="76"></td>
        <td id="LC76" class="blob-code blob-code-inner js-file-line">
</td>
      </tr>
      <tr>
        <td id="L77" class="blob-num js-line-number" data-line-number="77"></td>
        <td id="LC77" class="blob-code blob-code-inner js-file-line">\section{Input data}</td>
      </tr>
      <tr>
        <td id="L78" class="blob-num js-line-number" data-line-number="78"></td>
        <td id="LC78" class="blob-code blob-code-inner js-file-line">
</td>
      </tr>
      <tr>
        <td id="L79" class="blob-num js-line-number" data-line-number="79"></td>
        <td id="LC79" class="blob-code blob-code-inner js-file-line">\noiseq{} requires two pieces of information to work that must be provided to the \code{readData} function: the expression data (\texttt{data}) </td>
      </tr>
      <tr>
        <td id="L80" class="blob-num js-line-number" data-line-number="80"></td>
        <td id="LC80" class="blob-code blob-code-inner js-file-line">and the factors defining the experimental groups to be studied or compared (\texttt{factors}). However, in order to perform the quality control of the data or normalize them, other additional annotations need to be provided such as the feature length, </td>
      </tr>
      <tr>
        <td id="L81" class="blob-num js-line-number" data-line-number="81"></td>
        <td id="LC81" class="blob-code blob-code-inner js-file-line">the GC content, the biological classification of the features (e.g. Ensembl biotypes), or the chromosome position of each feature.</td>
      </tr>
      <tr>
        <td id="L82" class="blob-num js-line-number" data-line-number="82"></td>
        <td id="LC82" class="blob-code blob-code-inner js-file-line">
</td>
      </tr>
      <tr>
        <td id="L83" class="blob-num js-line-number" data-line-number="83"></td>
        <td id="LC83" class="blob-code blob-code-inner js-file-line">
</td>
      </tr>
      <tr>
        <td id="L84" class="blob-num js-line-number" data-line-number="84"></td>
        <td id="LC84" class="blob-code blob-code-inner js-file-line">\subsection{Expression data}</td>
      </tr>
      <tr>
        <td id="L85" class="blob-num js-line-number" data-line-number="85"></td>
        <td id="LC85" class="blob-code blob-code-inner js-file-line">
</td>
      </tr>
      <tr>
        <td id="L86" class="blob-num js-line-number" data-line-number="86"></td>
        <td id="LC86" class="blob-code blob-code-inner js-file-line">The expression data must be provided in a matrix or a data.frame R object, having as many rows as the number of features to be studied and as many columns as the number of</td>
      </tr>
      <tr>
        <td id="L87" class="blob-num js-line-number" data-line-number="87"></td>
        <td id="LC87" class="blob-code blob-code-inner js-file-line">samples in the experiment. The following example shows part of the count data for Marioni&#39;s dataset:</td>
      </tr>
      <tr>
        <td id="L88" class="blob-num js-line-number" data-line-number="88"></td>
        <td id="LC88" class="blob-code blob-code-inner js-file-line">&lt;&lt;&gt;&gt;=</td>
      </tr>
      <tr>
        <td id="L89" class="blob-num js-line-number" data-line-number="89"></td>
        <td id="LC89" class="blob-code blob-code-inner js-file-line">head(mycounts)</td>
      </tr>
      <tr>
        <td id="L90" class="blob-num js-line-number" data-line-number="90"></td>
        <td id="LC90" class="blob-code blob-code-inner js-file-line">@ </td>
      </tr>
      <tr>
        <td id="L91" class="blob-num js-line-number" data-line-number="91"></td>
        <td id="LC91" class="blob-code blob-code-inner js-file-line">
</td>
      </tr>
      <tr>
        <td id="L92" class="blob-num js-line-number" data-line-number="92"></td>
        <td id="LC92" class="blob-code blob-code-inner js-file-line">The expression data can be both read counts or normalized expression data such as RPKM values, and also any other normalized expression values.</td>
      </tr>
      <tr>
        <td id="L93" class="blob-num js-line-number" data-line-number="93"></td>
        <td id="LC93" class="blob-code blob-code-inner js-file-line">
</td>
      </tr>
      <tr>
        <td id="L94" class="blob-num js-line-number" data-line-number="94"></td>
        <td id="LC94" class="blob-code blob-code-inner js-file-line">
</td>
      </tr>
      <tr>
        <td id="L95" class="blob-num js-line-number" data-line-number="95"></td>
        <td id="LC95" class="blob-code blob-code-inner js-file-line">\subsection{Factors}</td>
      </tr>
      <tr>
        <td id="L96" class="blob-num js-line-number" data-line-number="96"></td>
        <td id="LC96" class="blob-code blob-code-inner js-file-line">Factors are the variables indicating the experimental group for each sample. They must be given to the \code{readData} function in a data frame </td>
      </tr>
      <tr>
        <td id="L97" class="blob-num js-line-number" data-line-number="97"></td>
        <td id="LC97" class="blob-code blob-code-inner js-file-line">object. This data frame must have as many rows as samples (columns in data object) and as many columns or factors as different sample annotations the user wants to use. For instance, in Marioni&#39;s data, we have the factor ``Tissue&#39;&#39;, but we can also define another factors (``Run&#39;&#39; or ``TissueRun&#39;&#39;). The levels of the factor ``Tissue&#39;&#39; are ``Kidney&#39;&#39; and ``Liver&#39;&#39;. The factor ``Run&#39;&#39; has two levels: ``R1&#39;&#39; and ``R2&#39;&#39;. The factor ``TissueRun&#39;&#39; combines the sequencing run with the tissue and hence has four levels: ``Kidney\_1&#39;&#39;, ``Liver\_1&#39;&#39;, ``Kidney\_2&#39;&#39; and ``Liver\_2&#39;&#39;. </td>
      </tr>
      <tr>
        <td id="L98" class="blob-num js-line-number" data-line-number="98"></td>
        <td id="LC98" class="blob-code blob-code-inner js-file-line">
</td>
      </tr>
      <tr>
        <td id="L99" class="blob-num js-line-number" data-line-number="99"></td>
        <td id="LC99" class="blob-code blob-code-inner js-file-line">Be careful here, the order of the elements of the factor must coincide with the order of the samples (columns) in the expression data file provided.</td>
      </tr>
      <tr>
        <td id="L100" class="blob-num js-line-number" data-line-number="100"></td>
        <td id="LC100" class="blob-code blob-code-inner js-file-line">
</td>
      </tr>
      <tr>
        <td id="L101" class="blob-num js-line-number" data-line-number="101"></td>
        <td id="LC101" class="blob-code blob-code-inner js-file-line">&lt;&lt;factors&gt;&gt;=</td>
      </tr>
      <tr>
        <td id="L102" class="blob-num js-line-number" data-line-number="102"></td>
        <td id="LC102" class="blob-code blob-code-inner js-file-line">myfactors = data.frame(Tissue=c(&quot;Kidney&quot;,&quot;Liver&quot;,&quot;Kidney&quot;,&quot;Liver&quot;,&quot;Liver&quot;,&quot;Kidney&quot;,&quot;Liver&quot;,</td>
      </tr>
      <tr>
        <td id="L103" class="blob-num js-line-number" data-line-number="103"></td>
        <td id="LC103" class="blob-code blob-code-inner js-file-line">                                &quot;Kidney&quot;,&quot;Liver&quot;,&quot;Kidney&quot;),</td>
      </tr>
      <tr>
        <td id="L104" class="blob-num js-line-number" data-line-number="104"></td>
        <td id="LC104" class="blob-code blob-code-inner js-file-line">                       TissueRun = c(&quot;Kidney_1&quot;,&quot;Liver_1&quot;,&quot;Kidney_1&quot;,&quot;Liver_1&quot;,&quot;Liver_1&quot;,</td>
      </tr>
      <tr>
        <td id="L105" class="blob-num js-line-number" data-line-number="105"></td>
        <td id="LC105" class="blob-code blob-code-inner js-file-line">                                     &quot;Kidney_1&quot;,&quot;Liver_1&quot;,&quot;Kidney_2&quot;,&quot;Liver_2&quot;,&quot;Kidney_2&quot;),</td>
      </tr>
      <tr>
        <td id="L106" class="blob-num js-line-number" data-line-number="106"></td>
        <td id="LC106" class="blob-code blob-code-inner js-file-line">                       Run = c(rep(&quot;R1&quot;, 7), rep(&quot;R2&quot;, 3)))</td>
      </tr>
      <tr>
        <td id="L107" class="blob-num js-line-number" data-line-number="107"></td>
        <td id="LC107" class="blob-code blob-code-inner js-file-line">myfactors</td>
      </tr>
      <tr>
        <td id="L108" class="blob-num js-line-number" data-line-number="108"></td>
        <td id="LC108" class="blob-code blob-code-inner js-file-line">@ </td>
      </tr>
      <tr>
        <td id="L109" class="blob-num js-line-number" data-line-number="109"></td>
        <td id="LC109" class="blob-code blob-code-inner js-file-line">
</td>
      </tr>
      <tr>
        <td id="L110" class="blob-num js-line-number" data-line-number="110"></td>
        <td id="LC110" class="blob-code blob-code-inner js-file-line">
</td>
      </tr>
      <tr>
        <td id="L111" class="blob-num js-line-number" data-line-number="111"></td>
        <td id="LC111" class="blob-code blob-code-inner js-file-line">\subsection{Additional biological annotation}</td>
      </tr>
      <tr>
        <td id="L112" class="blob-num js-line-number" data-line-number="112"></td>
        <td id="LC112" class="blob-code blob-code-inner js-file-line">
</td>
      </tr>
      <tr>
        <td id="L113" class="blob-num js-line-number" data-line-number="113"></td>
        <td id="LC113" class="blob-code blob-code-inner js-file-line">Some of the exploratory plots in \noiseq{} package require additional biological information such as feature length, GC content, biological </td>
      </tr>
      <tr>
        <td id="L114" class="blob-num js-line-number" data-line-number="114"></td>
        <td id="LC114" class="blob-code blob-code-inner js-file-line">classification of features, or chromosome position. You need to provide at least part of this information if you want to either generate </td>
      </tr>
      <tr>
        <td id="L115" class="blob-num js-line-number" data-line-number="115"></td>
        <td id="LC115" class="blob-code blob-code-inner js-file-line">the corresponding plots or apply a normalization method that corrects by length.</td>
      </tr>
      <tr>
        <td id="L116" class="blob-num js-line-number" data-line-number="116"></td>
        <td id="LC116" class="blob-code blob-code-inner js-file-line">
</td>
      </tr>
      <tr>
        <td id="L117" class="blob-num js-line-number" data-line-number="117"></td>
        <td id="LC117" class="blob-code blob-code-inner js-file-line">The following code show how the R objects containing such information should look like:</td>
      </tr>
      <tr>
        <td id="L118" class="blob-num js-line-number" data-line-number="118"></td>
        <td id="LC118" class="blob-code blob-code-inner js-file-line">
</td>
      </tr>
      <tr>
        <td id="L119" class="blob-num js-line-number" data-line-number="119"></td>
        <td id="LC119" class="blob-code blob-code-inner js-file-line">&lt;&lt;&gt;&gt;=</td>
      </tr>
      <tr>
        <td id="L120" class="blob-num js-line-number" data-line-number="120"></td>
        <td id="LC120" class="blob-code blob-code-inner js-file-line">head(mylength)</td>
      </tr>
      <tr>
        <td id="L121" class="blob-num js-line-number" data-line-number="121"></td>
        <td id="LC121" class="blob-code blob-code-inner js-file-line">head(mygc)</td>
      </tr>
      <tr>
        <td id="L122" class="blob-num js-line-number" data-line-number="122"></td>
        <td id="LC122" class="blob-code blob-code-inner js-file-line">head(mybiotypes)</td>
      </tr>
      <tr>
        <td id="L123" class="blob-num js-line-number" data-line-number="123"></td>
        <td id="LC123" class="blob-code blob-code-inner js-file-line">head(mychroms)</td>
      </tr>
      <tr>
        <td id="L124" class="blob-num js-line-number" data-line-number="124"></td>
        <td id="LC124" class="blob-code blob-code-inner js-file-line">@ </td>
      </tr>
      <tr>
        <td id="L125" class="blob-num js-line-number" data-line-number="125"></td>
        <td id="LC125" class="blob-code blob-code-inner js-file-line">
</td>
      </tr>
      <tr>
        <td id="L126" class="blob-num js-line-number" data-line-number="126"></td>
        <td id="LC126" class="blob-code blob-code-inner js-file-line">Please note, that these objects might contain a different number of features and in different order than the expression data. </td>
      </tr>
      <tr>
        <td id="L127" class="blob-num js-line-number" data-line-number="127"></td>
        <td id="LC127" class="blob-code blob-code-inner js-file-line">However, it is important to specify the names or IDs of the features in each case so the package can properly match all this information. </td>
      </tr>
      <tr>
        <td id="L128" class="blob-num js-line-number" data-line-number="128"></td>
        <td id="LC128" class="blob-code blob-code-inner js-file-line">The length, GC content or biological groups (e.g. biotypes), could be vectors, matrices or data.frames. </td>
      </tr>
      <tr>
        <td id="L129" class="blob-num js-line-number" data-line-number="129"></td>
        <td id="LC129" class="blob-code blob-code-inner js-file-line">If they are vectors, the names of the vector must be the feature names or</td>
      </tr>
      <tr>
        <td id="L130" class="blob-num js-line-number" data-line-number="130"></td>
        <td id="LC130" class="blob-code blob-code-inner js-file-line">IDs. If they are matrices or data.frame objects, the feature names or IDs must be in the row names of the object. </td>
      </tr>
      <tr>
        <td id="L131" class="blob-num js-line-number" data-line-number="131"></td>
        <td id="LC131" class="blob-code blob-code-inner js-file-line">The same applies for chromosome position, which is also a matrix or data.frame.</td>
      </tr>
      <tr>
        <td id="L132" class="blob-num js-line-number" data-line-number="132"></td>
        <td id="LC132" class="blob-code blob-code-inner js-file-line">
</td>
      </tr>
      <tr>
        <td id="L133" class="blob-num js-line-number" data-line-number="133"></td>
        <td id="LC133" class="blob-code blob-code-inner js-file-line">Ensembl Biomart data base provides these annotations for a wide range of species: biotypes (the biological classification of the features),</td>
      </tr>
      <tr>
        <td id="L134" class="blob-num js-line-number" data-line-number="134"></td>
        <td id="LC134" class="blob-code blob-code-inner js-file-line">GC content, or chromosome position. The latter can be used to estimate the length of the feature. However, it is more accurate computing the length</td>
      </tr>
      <tr>
        <td id="L135" class="blob-num js-line-number" data-line-number="135"></td>
        <td id="LC135" class="blob-code blob-code-inner js-file-line">from the GTF or GFF annotation file so the introns are not considered.</td>
      </tr>
      <tr>
        <td id="L136" class="blob-num js-line-number" data-line-number="136"></td>
        <td id="LC136" class="blob-code blob-code-inner js-file-line">
</td>
      </tr>
      <tr>
        <td id="L137" class="blob-num js-line-number" data-line-number="137"></td>
        <td id="LC137" class="blob-code blob-code-inner js-file-line">
</td>
      </tr>
      <tr>
        <td id="L138" class="blob-num js-line-number" data-line-number="138"></td>
        <td id="LC138" class="blob-code blob-code-inner js-file-line">
</td>
      </tr>
      <tr>
        <td id="L139" class="blob-num js-line-number" data-line-number="139"></td>
        <td id="LC139" class="blob-code blob-code-inner js-file-line">
</td>
      </tr>
      <tr>
        <td id="L140" class="blob-num js-line-number" data-line-number="140"></td>
        <td id="LC140" class="blob-code blob-code-inner js-file-line">\subsection{Converting data into a \noiseq{} object}</td>
      </tr>
      <tr>
        <td id="L141" class="blob-num js-line-number" data-line-number="141"></td>
        <td id="LC141" class="blob-code blob-code-inner js-file-line">
</td>
      </tr>
      <tr>
        <td id="L142" class="blob-num js-line-number" data-line-number="142"></td>
        <td id="LC142" class="blob-code blob-code-inner js-file-line">Once we have created in R the count data matrix, the data frame for the factors and the biological annotation objects (if needed), we have to pack all this information into a \noiseq{} object by using the \code{readData} function. An example on how it works is shown below: </td>
      </tr>
      <tr>
        <td id="L143" class="blob-num js-line-number" data-line-number="143"></td>
        <td id="LC143" class="blob-code blob-code-inner js-file-line">
</td>
      </tr>
      <tr>
        <td id="L144" class="blob-num js-line-number" data-line-number="144"></td>
        <td id="LC144" class="blob-code blob-code-inner js-file-line">&lt;&lt;readData&gt;&gt;=</td>
      </tr>
      <tr>
        <td id="L145" class="blob-num js-line-number" data-line-number="145"></td>
        <td id="LC145" class="blob-code blob-code-inner js-file-line">mydata &lt;- readData(data=mycounts,length=mylength, gc=mygc, biotype=mybiotypes,</td>
      </tr>
      <tr>
        <td id="L146" class="blob-num js-line-number" data-line-number="146"></td>
        <td id="LC146" class="blob-code blob-code-inner js-file-line">                   chromosome=mychroms, factors=myfactors)</td>
      </tr>
      <tr>
        <td id="L147" class="blob-num js-line-number" data-line-number="147"></td>
        <td id="LC147" class="blob-code blob-code-inner js-file-line">mydata</td>
      </tr>
      <tr>
        <td id="L148" class="blob-num js-line-number" data-line-number="148"></td>
        <td id="LC148" class="blob-code blob-code-inner js-file-line">@ </td>
      </tr>
      <tr>
        <td id="L149" class="blob-num js-line-number" data-line-number="149"></td>
        <td id="LC149" class="blob-code blob-code-inner js-file-line">
</td>
      </tr>
      <tr>
        <td id="L150" class="blob-num js-line-number" data-line-number="150"></td>
        <td id="LC150" class="blob-code blob-code-inner js-file-line">The \code{readData} function returns an object of \emph{Biobase&#39;s eSet} class. To see which information is included in this object, type for instance:</td>
      </tr>
      <tr>
        <td id="L151" class="blob-num js-line-number" data-line-number="151"></td>
        <td id="LC151" class="blob-code blob-code-inner js-file-line">
</td>
      </tr>
      <tr>
        <td id="L152" class="blob-num js-line-number" data-line-number="152"></td>
        <td id="LC152" class="blob-code blob-code-inner js-file-line">&lt;&lt;results=hide&gt;&gt;=</td>
      </tr>
      <tr>
        <td id="L153" class="blob-num js-line-number" data-line-number="153"></td>
        <td id="LC153" class="blob-code blob-code-inner js-file-line">str(mydata)</td>
      </tr>
      <tr>
        <td id="L154" class="blob-num js-line-number" data-line-number="154"></td>
        <td id="LC154" class="blob-code blob-code-inner js-file-line">head(assayData(mydata)$exprs)</td>
      </tr>
      <tr>
        <td id="L155" class="blob-num js-line-number" data-line-number="155"></td>
        <td id="LC155" class="blob-code blob-code-inner js-file-line">head(pData(mydata))</td>
      </tr>
      <tr>
        <td id="L156" class="blob-num js-line-number" data-line-number="156"></td>
        <td id="LC156" class="blob-code blob-code-inner js-file-line">head(featureData(mydata)@data)</td>
      </tr>
      <tr>
        <td id="L157" class="blob-num js-line-number" data-line-number="157"></td>
        <td id="LC157" class="blob-code blob-code-inner js-file-line">@ </td>
      </tr>
      <tr>
        <td id="L158" class="blob-num js-line-number" data-line-number="158"></td>
        <td id="LC158" class="blob-code blob-code-inner js-file-line">
</td>
      </tr>
      <tr>
        <td id="L159" class="blob-num js-line-number" data-line-number="159"></td>
        <td id="LC159" class="blob-code blob-code-inner js-file-line">Note that the features to be used by all the methods in the package will be those in the data expression object. </td>
      </tr>
      <tr>
        <td id="L160" class="blob-num js-line-number" data-line-number="160"></td>
        <td id="LC160" class="blob-code blob-code-inner js-file-line">If any of this features has not been included in the additional biological annotation (when provided), the corresponding value will be NA.</td>
      </tr>
      <tr>
        <td id="L161" class="blob-num js-line-number" data-line-number="161"></td>
        <td id="LC161" class="blob-code blob-code-inner js-file-line">
</td>
      </tr>
      <tr>
        <td id="L162" class="blob-num js-line-number" data-line-number="162"></td>
        <td id="LC162" class="blob-code blob-code-inner js-file-line">
</td>
      </tr>
      <tr>
        <td id="L163" class="blob-num js-line-number" data-line-number="163"></td>
        <td id="LC163" class="blob-code blob-code-inner js-file-line">It is possible to add information to an existing object. For instance, \code{noiseq} function accepts objects generated </td>
      </tr>
      <tr>
        <td id="L164" class="blob-num js-line-number" data-line-number="164"></td>
        <td id="LC164" class="blob-code blob-code-inner js-file-line">while using other packages such as \code{DESeq} package.</td>
      </tr>
      <tr>
        <td id="L165" class="blob-num js-line-number" data-line-number="165"></td>
        <td id="LC165" class="blob-code blob-code-inner js-file-line">In that case, annotations may not be included in the object. The \code{addData} function allows the user to add annotation data to the object. </td>
      </tr>
      <tr>
        <td id="L166" class="blob-num js-line-number" data-line-number="166"></td>
        <td id="LC166" class="blob-code blob-code-inner js-file-line">For instance, if you generated the data object like this:</td>
      </tr>
      <tr>
        <td id="L167" class="blob-num js-line-number" data-line-number="167"></td>
        <td id="LC167" class="blob-code blob-code-inner js-file-line">&lt;&lt;readData2&gt;&gt;=</td>
      </tr>
      <tr>
        <td id="L168" class="blob-num js-line-number" data-line-number="168"></td>
        <td id="LC168" class="blob-code blob-code-inner js-file-line">mydata &lt;- readData(data=mycounts,chromosome=mychroms, factors=myfactors)</td>
      </tr>
      <tr>
        <td id="L169" class="blob-num js-line-number" data-line-number="169"></td>
        <td id="LC169" class="blob-code blob-code-inner js-file-line">@</td>
      </tr>
      <tr>
        <td id="L170" class="blob-num js-line-number" data-line-number="170"></td>
        <td id="LC170" class="blob-code blob-code-inner js-file-line">
</td>
      </tr>
      <tr>
        <td id="L171" class="blob-num js-line-number" data-line-number="171"></td>
        <td id="LC171" class="blob-code blob-code-inner js-file-line">And now you want to include the length and the biotypes, you have to use the \code{addData} function:</td>
      </tr>
      <tr>
        <td id="L172" class="blob-num js-line-number" data-line-number="172"></td>
        <td id="LC172" class="blob-code blob-code-inner js-file-line">
</td>
      </tr>
      <tr>
        <td id="L173" class="blob-num js-line-number" data-line-number="173"></td>
        <td id="LC173" class="blob-code blob-code-inner js-file-line">&lt;&lt;readData3&gt;&gt;=</td>
      </tr>
      <tr>
        <td id="L174" class="blob-num js-line-number" data-line-number="174"></td>
        <td id="LC174" class="blob-code blob-code-inner js-file-line">mydata &lt;- addData(mydata, length=mylength, biotype=mybiotypes, gc = mygc)</td>
      </tr>
      <tr>
        <td id="L175" class="blob-num js-line-number" data-line-number="175"></td>
        <td id="LC175" class="blob-code blob-code-inner js-file-line">@</td>
      </tr>
      <tr>
        <td id="L176" class="blob-num js-line-number" data-line-number="176"></td>
        <td id="LC176" class="blob-code blob-code-inner js-file-line">
</td>
      </tr>
      <tr>
        <td id="L177" class="blob-num js-line-number" data-line-number="177"></td>
        <td id="LC177" class="blob-code blob-code-inner js-file-line">
</td>
      </tr>
      <tr>
        <td id="L178" class="blob-num js-line-number" data-line-number="178"></td>
        <td id="LC178" class="blob-code blob-code-inner js-file-line">\textbf{IMPORTANT}: Some packages such as \emph{ShortRead} also use the \code{readData} function but with different input object and parameters. </td>
      </tr>
      <tr>
        <td id="L179" class="blob-num js-line-number" data-line-number="179"></td>
        <td id="LC179" class="blob-code blob-code-inner js-file-line">Therefore, some incompatibilities may occur that cause errors. To avoid this problem when loading simultaneously packages with functions with the same</td>
      </tr>
      <tr>
        <td id="L180" class="blob-num js-line-number" data-line-number="180"></td>
        <td id="LC180" class="blob-code blob-code-inner js-file-line">name but different use, the following command can be used: \code{NOISeq::readData} instead of simply \code{readData}.</td>
      </tr>
      <tr>
        <td id="L181" class="blob-num js-line-number" data-line-number="181"></td>
        <td id="LC181" class="blob-code blob-code-inner js-file-line">
</td>
      </tr>
      <tr>
        <td id="L182" class="blob-num js-line-number" data-line-number="182"></td>
        <td id="LC182" class="blob-code blob-code-inner js-file-line">
</td>
      </tr>
      <tr>
        <td id="L183" class="blob-num js-line-number" data-line-number="183"></td>
        <td id="LC183" class="blob-code blob-code-inner js-file-line">
</td>
      </tr>
      <tr>
        <td id="L184" class="blob-num js-line-number" data-line-number="184"></td>
        <td id="LC184" class="blob-code blob-code-inner js-file-line">\vspace{1cm}</td>
      </tr>
      <tr>
        <td id="L185" class="blob-num js-line-number" data-line-number="185"></td>
        <td id="LC185" class="blob-code blob-code-inner js-file-line">
</td>
      </tr>
      <tr>
        <td id="L186" class="blob-num js-line-number" data-line-number="186"></td>
        <td id="LC186" class="blob-code blob-code-inner js-file-line">
</td>
      </tr>
      <tr>
        <td id="L187" class="blob-num js-line-number" data-line-number="187"></td>
        <td id="LC187" class="blob-code blob-code-inner js-file-line">
</td>
      </tr>
      <tr>
        <td id="L188" class="blob-num js-line-number" data-line-number="188"></td>
        <td id="LC188" class="blob-code blob-code-inner js-file-line">% \clearpage</td>
      </tr>
      <tr>
        <td id="L189" class="blob-num js-line-number" data-line-number="189"></td>
        <td id="LC189" class="blob-code blob-code-inner js-file-line">
</td>
      </tr>
      <tr>
        <td id="L190" class="blob-num js-line-number" data-line-number="190"></td>
        <td id="LC190" class="blob-code blob-code-inner js-file-line">\section{Quality control of count data}</td>
      </tr>
      <tr>
        <td id="L191" class="blob-num js-line-number" data-line-number="191"></td>
        <td id="LC191" class="blob-code blob-code-inner js-file-line">
</td>
      </tr>
      <tr>
        <td id="L192" class="blob-num js-line-number" data-line-number="192"></td>
        <td id="LC192" class="blob-code blob-code-inner js-file-line">Data processing and sequencing experiment design in RNA-seq are not straightforward. From the</td>
      </tr>
      <tr>
        <td id="L193" class="blob-num js-line-number" data-line-number="193"></td>
        <td id="LC193" class="blob-code blob-code-inner js-file-line">biological samples to the expression quantification, there are many steps in which errors may be produced, despite of the many procedures </td>
      </tr>
      <tr>
        <td id="L194" class="blob-num js-line-number" data-line-number="194"></td>
        <td id="LC194" class="blob-code blob-code-inner js-file-line">developed to reduce noise at each one of these steps and to control the quality of the generated data.</td>
      </tr>
      <tr>
        <td id="L195" class="blob-num js-line-number" data-line-number="195"></td>
        <td id="LC195" class="blob-code blob-code-inner js-file-line">Therefore, once the expression levels (read counts) have been obtained, it is absolutely</td>
      </tr>
      <tr>
        <td id="L196" class="blob-num js-line-number" data-line-number="196"></td>
        <td id="LC196" class="blob-code blob-code-inner js-file-line">necessary to be able to detect potential biases or contamination before proceeding with further analysis (e.g. differential expression). </td>
      </tr>
      <tr>
        <td id="L197" class="blob-num js-line-number" data-line-number="197"></td>
        <td id="LC197" class="blob-code blob-code-inner js-file-line">The technology biases, such as the transcript length, GC content, PCR artifacts, uneven transcript read coverage, contamination by off-target transcripts or big differences in transcript distributions, are factors that interfere in the linear relationship between transcript</td>
      </tr>
      <tr>
        <td id="L198" class="blob-num js-line-number" data-line-number="198"></td>
        <td id="LC198" class="blob-code blob-code-inner js-file-line">abundance and the number of mapped reads at a gene locus (counts).</td>
      </tr>
      <tr>
        <td id="L199" class="blob-num js-line-number" data-line-number="199"></td>
        <td id="LC199" class="blob-code blob-code-inner js-file-line">
</td>
      </tr>
      <tr>
        <td id="L200" class="blob-num js-line-number" data-line-number="200"></td>
        <td id="LC200" class="blob-code blob-code-inner js-file-line">In this section, we present a set of plots to explore the count data that may be helpful to detect these potential biases </td>
      </tr>
      <tr>
        <td id="L201" class="blob-num js-line-number" data-line-number="201"></td>
        <td id="LC201" class="blob-code blob-code-inner js-file-line">so an appropriate normalization procedure can be chosen. For instance, these plots will be useful for seeing which kind of features </td>
      </tr>
      <tr>
        <td id="L202" class="blob-num js-line-number" data-line-number="202"></td>
        <td id="LC202" class="blob-code blob-code-inner js-file-line">(e.g. genes) are being detected in our RNA-seq samples and with how many counts, which technical biases are present, etc. </td>
      </tr>
      <tr>
        <td id="L203" class="blob-num js-line-number" data-line-number="203"></td>
        <td id="LC203" class="blob-code blob-code-inner js-file-line">
</td>
      </tr>
      <tr>
        <td id="L204" class="blob-num js-line-number" data-line-number="204"></td>
        <td id="LC204" class="blob-code blob-code-inner js-file-line">As it will be seen at the end of this section, it is also possible to generate a report in a PDF file including </td>
      </tr>
      <tr>
        <td id="L205" class="blob-num js-line-number" data-line-number="205"></td>
        <td id="LC205" class="blob-code blob-code-inner js-file-line">all these exploratory plots for the comparison of two samples or two experimental conditions.</td>
      </tr>
      <tr>
        <td id="L206" class="blob-num js-line-number" data-line-number="206"></td>
        <td id="LC206" class="blob-code blob-code-inner js-file-line">
</td>
      </tr>
      <tr>
        <td id="L207" class="blob-num js-line-number" data-line-number="207"></td>
        <td id="LC207" class="blob-code blob-code-inner js-file-line">
</td>
      </tr>
      <tr>
        <td id="L208" class="blob-num js-line-number" data-line-number="208"></td>
        <td id="LC208" class="blob-code blob-code-inner js-file-line">
</td>
      </tr>
      <tr>
        <td id="L209" class="blob-num js-line-number" data-line-number="209"></td>
        <td id="LC209" class="blob-code blob-code-inner js-file-line">\subsection{Generating data for exploratory plots}</td>
      </tr>
      <tr>
        <td id="L210" class="blob-num js-line-number" data-line-number="210"></td>
        <td id="LC210" class="blob-code blob-code-inner js-file-line">
</td>
      </tr>
      <tr>
        <td id="L211" class="blob-num js-line-number" data-line-number="211"></td>
        <td id="LC211" class="blob-code blob-code-inner js-file-line">There are several types of exploratory plots that can be obtained. They will be described in detail in the following sections. </td>
      </tr>
      <tr>
        <td id="L212" class="blob-num js-line-number" data-line-number="212"></td>
        <td id="LC212" class="blob-code blob-code-inner js-file-line">To generate any of these plots, first of all, \code{dat} function must be applied on the input data (\noiseq{} object) </td>
      </tr>
      <tr>
        <td id="L213" class="blob-num js-line-number" data-line-number="213"></td>
        <td id="LC213" class="blob-code blob-code-inner js-file-line">to obtain the information to be plotted. </td>
      </tr>
      <tr>
        <td id="L214" class="blob-num js-line-number" data-line-number="214"></td>
        <td id="LC214" class="blob-code blob-code-inner js-file-line">The user must specify the type of plot the data are to be computed for (argument \code{type}). </td>
      </tr>
      <tr>
        <td id="L215" class="blob-num js-line-number" data-line-number="215"></td>
        <td id="LC215" class="blob-code blob-code-inner js-file-line">Once the data for the plot have been generated with \code{dat} function, the plot will be drawn with the \emph{explo.plot} function. </td>
      </tr>
      <tr>
        <td id="L216" class="blob-num js-line-number" data-line-number="216"></td>
        <td id="LC216" class="blob-code blob-code-inner js-file-line">Therefore, for the quality control plots, we will always proceed like in the following example:</td>
      </tr>
      <tr>
        <td id="L217" class="blob-num js-line-number" data-line-number="217"></td>
        <td id="LC217" class="blob-code blob-code-inner js-file-line">
</td>
      </tr>
      <tr>
        <td id="L218" class="blob-num js-line-number" data-line-number="218"></td>
        <td id="LC218" class="blob-code blob-code-inner js-file-line">&lt;&lt;dat&gt;&gt;=</td>
      </tr>
      <tr>
        <td id="L219" class="blob-num js-line-number" data-line-number="219"></td>
        <td id="LC219" class="blob-code blob-code-inner js-file-line">myexplodata &lt;- dat(mydata, type = &quot;biodetection&quot;)</td>
      </tr>
      <tr>
        <td id="L220" class="blob-num js-line-number" data-line-number="220"></td>
        <td id="LC220" class="blob-code blob-code-inner js-file-line">explo.plot(myexplodata, plottype = &quot;persample&quot;)</td>
      </tr>
      <tr>
        <td id="L221" class="blob-num js-line-number" data-line-number="221"></td>
        <td id="LC221" class="blob-code blob-code-inner js-file-line">@</td>
      </tr>
      <tr>
        <td id="L222" class="blob-num js-line-number" data-line-number="222"></td>
        <td id="LC222" class="blob-code blob-code-inner js-file-line">
</td>
      </tr>
      <tr>
        <td id="L223" class="blob-num js-line-number" data-line-number="223"></td>
        <td id="LC223" class="blob-code blob-code-inner js-file-line">
</td>
      </tr>
      <tr>
        <td id="L224" class="blob-num js-line-number" data-line-number="224"></td>
        <td id="LC224" class="blob-code blob-code-inner js-file-line">To save the data in a user-friendly format, the \code{dat2save} function can be used:</td>
      </tr>
      <tr>
        <td id="L225" class="blob-num js-line-number" data-line-number="225"></td>
        <td id="LC225" class="blob-code blob-code-inner js-file-line">
</td>
      </tr>
      <tr>
        <td id="L226" class="blob-num js-line-number" data-line-number="226"></td>
        <td id="LC226" class="blob-code blob-code-inner js-file-line">&lt;&lt;nicedata&gt;&gt;=</td>
      </tr>
      <tr>
        <td id="L227" class="blob-num js-line-number" data-line-number="227"></td>
        <td id="LC227" class="blob-code blob-code-inner js-file-line">mynicedata &lt;- dat2save(myexplodata)</td>
      </tr>
      <tr>
        <td id="L228" class="blob-num js-line-number" data-line-number="228"></td>
        <td id="LC228" class="blob-code blob-code-inner js-file-line">@</td>
      </tr>
      <tr>
        <td id="L229" class="blob-num js-line-number" data-line-number="229"></td>
        <td id="LC229" class="blob-code blob-code-inner js-file-line">
</td>
      </tr>
      <tr>
        <td id="L230" class="blob-num js-line-number" data-line-number="230"></td>
        <td id="LC230" class="blob-code blob-code-inner js-file-line">
</td>
      </tr>
      <tr>
        <td id="L231" class="blob-num js-line-number" data-line-number="231"></td>
        <td id="LC231" class="blob-code blob-code-inner js-file-line">We have grouped the exploratory plots in three categories according to the different questions that may arise during the quality control of the expression data:</td>
      </tr>
      <tr>
        <td id="L232" class="blob-num js-line-number" data-line-number="232"></td>
        <td id="LC232" class="blob-code blob-code-inner js-file-line">
</td>
      </tr>
      <tr>
        <td id="L233" class="blob-num js-line-number" data-line-number="233"></td>
        <td id="LC233" class="blob-code blob-code-inner js-file-line">\begin{itemize}</td>
      </tr>
      <tr>
        <td id="L234" class="blob-num js-line-number" data-line-number="234"></td>
        <td id="LC234" class="blob-code blob-code-inner js-file-line"> \item \textbf{Biotype detection}: Which kind of features are being detected? Is there any abnormal contamination in the data? Did I choose an appropriate protocol?</td>
      </tr>
      <tr>
        <td id="L235" class="blob-num js-line-number" data-line-number="235"></td>
        <td id="LC235" class="blob-code blob-code-inner js-file-line"> \item \textbf{Sequencing depth \&amp; Expression Quantification}: Would it be better to increase the sequencing depth to detect more features? </td>
      </tr>
      <tr>
        <td id="L236" class="blob-num js-line-number" data-line-number="236"></td>
        <td id="LC236" class="blob-code blob-code-inner js-file-line">Are there too many features with low counts? Are the samples very different regarding the expression quantification?</td>
      </tr>
      <tr>
        <td id="L237" class="blob-num js-line-number" data-line-number="237"></td>
        <td id="LC237" class="blob-code blob-code-inner js-file-line"> \item \textbf{Sequencing bias detection}: Should the expression values be corrected for the length or the GC content bias? </td>
      </tr>
      <tr>
        <td id="L238" class="blob-num js-line-number" data-line-number="238"></td>
        <td id="LC238" class="blob-code blob-code-inner js-file-line">Should a normalization procedure be applied to account for the differences among RNA composition among samples?</td>
      </tr>
      <tr>
        <td id="L239" class="blob-num js-line-number" data-line-number="239"></td>
        <td id="LC239" class="blob-code blob-code-inner js-file-line"> \item \textbf{Batch effect exploration}: Are the samples clustered in corcondance with the experimental design or with the batch in which they were processed? </td>
      </tr>
      <tr>
        <td id="L240" class="blob-num js-line-number" data-line-number="240"></td>
        <td id="LC240" class="blob-code blob-code-inner js-file-line">\end{itemize}</td>
      </tr>
      <tr>
        <td id="L241" class="blob-num js-line-number" data-line-number="241"></td>
        <td id="LC241" class="blob-code blob-code-inner js-file-line">
</td>
      </tr>
      <tr>
        <td id="L242" class="blob-num js-line-number" data-line-number="242"></td>
        <td id="LC242" class="blob-code blob-code-inner js-file-line">
</td>
      </tr>
      <tr>
        <td id="L243" class="blob-num js-line-number" data-line-number="243"></td>
        <td id="LC243" class="blob-code blob-code-inner js-file-line">
</td>
      </tr>
      <tr>
        <td id="L244" class="blob-num js-line-number" data-line-number="244"></td>
        <td id="LC244" class="blob-code blob-code-inner js-file-line">
</td>
      </tr>
      <tr>
        <td id="L245" class="blob-num js-line-number" data-line-number="245"></td>
        <td id="LC245" class="blob-code blob-code-inner js-file-line">\subsection{Biotype detection}</td>
      </tr>
      <tr>
        <td id="L246" class="blob-num js-line-number" data-line-number="246"></td>
        <td id="LC246" class="blob-code blob-code-inner js-file-line">
</td>
      </tr>
      <tr>
        <td id="L247" class="blob-num js-line-number" data-line-number="247"></td>
        <td id="LC247" class="blob-code blob-code-inner js-file-line">When a biological classification of the features is provided (e.g. Ensembl biotypes), the following plots are useful to see </td>
      </tr>
      <tr>
        <td id="L248" class="blob-num js-line-number" data-line-number="248"></td>
        <td id="LC248" class="blob-code blob-code-inner js-file-line">which kind of features are being detected.</td>
      </tr>
      <tr>
        <td id="L249" class="blob-num js-line-number" data-line-number="249"></td>
        <td id="LC249" class="blob-code blob-code-inner js-file-line">For instance, in RNA-seq, it is expected that most of the genes will be protein-coding so detecting an enrichment in the sample of any other biotype could point to a potential contamination or at least provide information on the sample composition to take decision on the type of analysis to be performed.</td>
      </tr>
      <tr>
        <td id="L250" class="blob-num js-line-number" data-line-number="250"></td>
        <td id="LC250" class="blob-code blob-code-inner js-file-line">
</td>
      </tr>
      <tr>
        <td id="L251" class="blob-num js-line-number" data-line-number="251"></td>
        <td id="LC251" class="blob-code blob-code-inner js-file-line">\subsubsection{Biodetection plot}</td>
      </tr>
      <tr>
        <td id="L252" class="blob-num js-line-number" data-line-number="252"></td>
        <td id="LC252" class="blob-code blob-code-inner js-file-line">
</td>
      </tr>
      <tr>
        <td id="L253" class="blob-num js-line-number" data-line-number="253"></td>
        <td id="LC253" class="blob-code blob-code-inner js-file-line">The example below shows how to use the \code{dat} and \code{explo.plot} functions to generate the data to be plotted and to draw a biodetection plot per sample. </td>
      </tr>
      <tr>
        <td id="L254" class="blob-num js-line-number" data-line-number="254"></td>
        <td id="LC254" class="blob-code blob-code-inner js-file-line">
</td>
      </tr>
      <tr>
        <td id="L255" class="blob-num js-line-number" data-line-number="255"></td>
        <td id="LC255" class="blob-code blob-code-inner js-file-line">&lt;&lt;fig_biodetection,fig=TRUE,width=12&gt;&gt;=</td>
      </tr>
      <tr>
        <td id="L256" class="blob-num js-line-number" data-line-number="256"></td>
        <td id="LC256" class="blob-code blob-code-inner js-file-line">mybiodetection &lt;- dat(mydata, k = 0, type = &quot;biodetection&quot;, factor = NULL)</td>
      </tr>
      <tr>
        <td id="L257" class="blob-num js-line-number" data-line-number="257"></td>
        <td id="LC257" class="blob-code blob-code-inner js-file-line">par(mfrow = c(1,2))  # we need this instruction because two plots (one per sample) will be generated</td>
      </tr>
      <tr>
        <td id="L258" class="blob-num js-line-number" data-line-number="258"></td>
        <td id="LC258" class="blob-code blob-code-inner js-file-line">explo.plot(mybiodetection, samples=c(1,2), plottype = &quot;persample&quot;)</td>
      </tr>
      <tr>
        <td id="L259" class="blob-num js-line-number" data-line-number="259"></td>
        <td id="LC259" class="blob-code blob-code-inner js-file-line">@ </td>
      </tr>
      <tr>
        <td id="L260" class="blob-num js-line-number" data-line-number="260"></td>
        <td id="LC260" class="blob-code blob-code-inner js-file-line">
</td>
      </tr>
      <tr>
        <td id="L261" class="blob-num js-line-number" data-line-number="261"></td>
        <td id="LC261" class="blob-code blob-code-inner js-file-line">
</td>
      </tr>
      <tr>
        <td id="L262" class="blob-num js-line-number" data-line-number="262"></td>
        <td id="LC262" class="blob-code blob-code-inner js-file-line">Fig. \ref{fig_biodetection} shows the ``biodetection&quot; plot per sample. The gray bar corresponds to the percentage of each biotype in the genome </td>
      </tr>
      <tr>
        <td id="L263" class="blob-num js-line-number" data-line-number="263"></td>
        <td id="LC263" class="blob-code blob-code-inner js-file-line">(i.e. in the whole set of features provided), the stripped color bar is the proportion detected in our sample </td>
      </tr>
      <tr>
        <td id="L264" class="blob-num js-line-number" data-line-number="264"></td>
        <td id="LC264" class="blob-code blob-code-inner js-file-line">(with number of counts higher than \texttt{k}), and the solid color bar is the percentage of each biotype within the sample.</td>
      </tr>
      <tr>
        <td id="L265" class="blob-num js-line-number" data-line-number="265"></td>
        <td id="LC265" class="blob-code blob-code-inner js-file-line">The vertical green line separates the most abundant biotypes (in the left-hand side, corresponding to the left axis scale) </td>
      </tr>
      <tr>
        <td id="L266" class="blob-num js-line-number" data-line-number="266"></td>
        <td id="LC266" class="blob-code blob-code-inner js-file-line">from the rest (in the right-hand side, corresponding to the right axis scale).</td>
      </tr>
      <tr>
        <td id="L267" class="blob-num js-line-number" data-line-number="267"></td>
        <td id="LC267" class="blob-code blob-code-inner js-file-line">
</td>
      </tr>
      <tr>
        <td id="L268" class="blob-num js-line-number" data-line-number="268"></td>
        <td id="LC268" class="blob-code blob-code-inner js-file-line">When \texttt{factor=NULL}, the data for the plot are computed separately for each sample. If \texttt{factor} is a string indicating the name of one of the columns in the factor object, the samples are aggregated within each of these experimental conditions and the data for the plot are computed per condition. In this example, samples in columns 1 and 2 from expression data are plotted and the features (genes) are considered to be detected if having a number of counts higher than \texttt{k=0}.</td>
      </tr>
      <tr>
        <td id="L269" class="blob-num js-line-number" data-line-number="269"></td>
        <td id="LC269" class="blob-code blob-code-inner js-file-line">
</td>
      </tr>
      <tr>
        <td id="L270" class="blob-num js-line-number" data-line-number="270"></td>
        <td id="LC270" class="blob-code blob-code-inner js-file-line">
</td>
      </tr>
      <tr>
        <td id="L271" class="blob-num js-line-number" data-line-number="271"></td>
        <td id="LC271" class="blob-code blob-code-inner js-file-line">\begin{figure}[ht!]</td>
      </tr>
      <tr>
        <td id="L272" class="blob-num js-line-number" data-line-number="272"></td>
        <td id="LC272" class="blob-code blob-code-inner js-file-line">\centering</td>
      </tr>
      <tr>
        <td id="L273" class="blob-num js-line-number" data-line-number="273"></td>
        <td id="LC273" class="blob-code blob-code-inner js-file-line">\includegraphics[width=0.9\textwidth]{NOISeq-fig_biodetection}</td>
      </tr>
      <tr>
        <td id="L274" class="blob-num js-line-number" data-line-number="274"></td>
        <td id="LC274" class="blob-code blob-code-inner js-file-line">\caption{Biodetection plot (per sample)}</td>
      </tr>
      <tr>
        <td id="L275" class="blob-num js-line-number" data-line-number="275"></td>
        <td id="LC275" class="blob-code blob-code-inner js-file-line">\label{fig_biodetection}</td>
      </tr>
      <tr>
        <td id="L276" class="blob-num js-line-number" data-line-number="276"></td>
        <td id="LC276" class="blob-code blob-code-inner js-file-line">\end{figure}</td>
      </tr>
      <tr>
        <td id="L277" class="blob-num js-line-number" data-line-number="277"></td>
        <td id="LC277" class="blob-code blob-code-inner js-file-line">
</td>
      </tr>
      <tr>
        <td id="L278" class="blob-num js-line-number" data-line-number="278"></td>
        <td id="LC278" class="blob-code blob-code-inner js-file-line">
</td>
      </tr>
      <tr>
        <td id="L279" class="blob-num js-line-number" data-line-number="279"></td>
        <td id="LC279" class="blob-code blob-code-inner js-file-line">When two samples or conditions are to be compared, it can be more practical to represent both o them in the same plot. Then, two different plots can be generated: one representing the percentage of each biotype in the genome being detected in the sample, and other representing the relative abundance of each biotype within the sample. The following code can be used to obtain such plots:</td>
      </tr>
      <tr>
        <td id="L280" class="blob-num js-line-number" data-line-number="280"></td>
        <td id="LC280" class="blob-code blob-code-inner js-file-line">
</td>
      </tr>
      <tr>
        <td id="L281" class="blob-num js-line-number" data-line-number="281"></td>
        <td id="LC281" class="blob-code blob-code-inner js-file-line">&lt;&lt;fig_biodetection2,fig=TRUE,width=12&gt;&gt;=</td>
      </tr>
      <tr>
        <td id="L282" class="blob-num js-line-number" data-line-number="282"></td>
        <td id="LC282" class="blob-code blob-code-inner js-file-line">par(mfrow = c(1,2))  # we need this instruction because two plots (one per sample) will be generated</td>
      </tr>
      <tr>
        <td id="L283" class="blob-num js-line-number" data-line-number="283"></td>
        <td id="LC283" class="blob-code blob-code-inner js-file-line">explo.plot(mybiodetection, samples=c(1,2), toplot = &quot;protein_coding&quot;, plottype = &quot;comparison&quot;)</td>
      </tr>
      <tr>
        <td id="L284" class="blob-num js-line-number" data-line-number="284"></td>
        <td id="LC284" class="blob-code blob-code-inner js-file-line">@ </td>
      </tr>
      <tr>
        <td id="L285" class="blob-num js-line-number" data-line-number="285"></td>
        <td id="LC285" class="blob-code blob-code-inner js-file-line">
</td>
      </tr>
      <tr>
        <td id="L286" class="blob-num js-line-number" data-line-number="286"></td>
        <td id="LC286" class="blob-code blob-code-inner js-file-line">
</td>
      </tr>
      <tr>
        <td id="L287" class="blob-num js-line-number" data-line-number="287"></td>
        <td id="LC287" class="blob-code blob-code-inner js-file-line">\begin{figure}[ht!]</td>
      </tr>
      <tr>
        <td id="L288" class="blob-num js-line-number" data-line-number="288"></td>
        <td id="LC288" class="blob-code blob-code-inner js-file-line">\centering</td>
      </tr>
      <tr>
        <td id="L289" class="blob-num js-line-number" data-line-number="289"></td>
        <td id="LC289" class="blob-code blob-code-inner js-file-line">\includegraphics[width=0.9\textwidth]{NOISeq-fig_biodetection2}</td>
      </tr>
      <tr>
        <td id="L290" class="blob-num js-line-number" data-line-number="290"></td>
        <td id="LC290" class="blob-code blob-code-inner js-file-line">\caption{Biodetection plot (comparison of two samples)}</td>
      </tr>
      <tr>
        <td id="L291" class="blob-num js-line-number" data-line-number="291"></td>
        <td id="LC291" class="blob-code blob-code-inner js-file-line">\label{fig_biodetection2}</td>
      </tr>
      <tr>
        <td id="L292" class="blob-num js-line-number" data-line-number="292"></td>
        <td id="LC292" class="blob-code blob-code-inner js-file-line">\end{figure}</td>
      </tr>
      <tr>
        <td id="L293" class="blob-num js-line-number" data-line-number="293"></td>
        <td id="LC293" class="blob-code blob-code-inner js-file-line">
</td>
      </tr>
      <tr>
        <td id="L294" class="blob-num js-line-number" data-line-number="294"></td>
        <td id="LC294" class="blob-code blob-code-inner js-file-line">In addition, the ``biotype comparison&#39;&#39; plot also performs a proportion test for the chosen biotype (argument \texttt{toplot}) to test if the relative abundance of that biotype is different in the two samples or conditions compared.</td>
      </tr>
      <tr>
        <td id="L295" class="blob-num js-line-number" data-line-number="295"></td>
        <td id="LC295" class="blob-code blob-code-inner js-file-line">
</td>
      </tr>
      <tr>
        <td id="L296" class="blob-num js-line-number" data-line-number="296"></td>
        <td id="LC296" class="blob-code blob-code-inner js-file-line">
</td>
      </tr>
      <tr>
        <td id="L297" class="blob-num js-line-number" data-line-number="297"></td>
        <td id="LC297" class="blob-code blob-code-inner js-file-line">
</td>
      </tr>
      <tr>
        <td id="L298" class="blob-num js-line-number" data-line-number="298"></td>
        <td id="LC298" class="blob-code blob-code-inner js-file-line">
</td>
      </tr>
      <tr>
        <td id="L299" class="blob-num js-line-number" data-line-number="299"></td>
        <td id="LC299" class="blob-code blob-code-inner js-file-line">\subsubsection{Count distribution per biotype}</td>
      </tr>
      <tr>
        <td id="L300" class="blob-num js-line-number" data-line-number="300"></td>
        <td id="LC300" class="blob-code blob-code-inner js-file-line">
</td>
      </tr>
      <tr>
        <td id="L301" class="blob-num js-line-number" data-line-number="301"></td>
        <td id="LC301" class="blob-code blob-code-inner js-file-line">The ``countsbio&quot; plot (Fig. \ref{fig_boxplot1}) per biotype allows to see how the counts are distributed within each biological group. In the upper side of the plot, the number of detected features that will be represented in the boxplots is displayed. The values used for the boxplots are either the counts per million (if \texttt{norm = FALSE}) or the values provided by the use (if \texttt{norm = TRUE})</td>
      </tr>
      <tr>
        <td id="L302" class="blob-num js-line-number" data-line-number="302"></td>
        <td id="LC302" class="blob-code blob-code-inner js-file-line">The following code was used to draw the figure. Again, data are computed per sample because no factor was specified </td>
      </tr>
      <tr>
        <td id="L303" class="blob-num js-line-number" data-line-number="303"></td>
        <td id="LC303" class="blob-code blob-code-inner js-file-line">(\texttt{factor=NULL}).</td>
      </tr>
      <tr>
        <td id="L304" class="blob-num js-line-number" data-line-number="304"></td>
        <td id="LC304" class="blob-code blob-code-inner js-file-line">To obtain this plot using the \emph{explo.plot} function and the ``countsbio&quot; data, we have to indicate the ``boxplot&quot; type in the \texttt{plottype} argument, choose only one of the samples (\texttt{samples = 1}, in this case), and all the biotypes (by setting \code{toplot} parameter to 1 or &quot;global&quot;). </td>
      </tr>
      <tr>
        <td id="L305" class="blob-num js-line-number" data-line-number="305"></td>
        <td id="LC305" class="blob-code blob-code-inner js-file-line">
</td>
      </tr>
      <tr>
        <td id="L306" class="blob-num js-line-number" data-line-number="306"></td>
        <td id="LC306" class="blob-code blob-code-inner js-file-line">&lt;&lt;fig_boxplot1,fig=TRUE,width=10&gt;&gt;=</td>
      </tr>
      <tr>
        <td id="L307" class="blob-num js-line-number" data-line-number="307"></td>
        <td id="LC307" class="blob-code blob-code-inner js-file-line">mycountsbio = dat(mydata, factor = NULL, type = &quot;countsbio&quot;)</td>
      </tr>
      <tr>
        <td id="L308" class="blob-num js-line-number" data-line-number="308"></td>
        <td id="LC308" class="blob-code blob-code-inner js-file-line">explo.plot(mycountsbio, toplot = 1, samples = 1, plottype = &quot;boxplot&quot;)</td>
      </tr>
      <tr>
        <td id="L309" class="blob-num js-line-number" data-line-number="309"></td>
        <td id="LC309" class="blob-code blob-code-inner js-file-line">@ </td>
      </tr>
      <tr>
        <td id="L310" class="blob-num js-line-number" data-line-number="310"></td>
        <td id="LC310" class="blob-code blob-code-inner js-file-line">
</td>
      </tr>
      <tr>
        <td id="L311" class="blob-num js-line-number" data-line-number="311"></td>
        <td id="LC311" class="blob-code blob-code-inner js-file-line">\begin{figure}[ht!]</td>
      </tr>
      <tr>
        <td id="L312" class="blob-num js-line-number" data-line-number="312"></td>
        <td id="LC312" class="blob-code blob-code-inner js-file-line">\centering</td>
      </tr>
      <tr>
        <td id="L313" class="blob-num js-line-number" data-line-number="313"></td>
        <td id="LC313" class="blob-code blob-code-inner js-file-line">\includegraphics[width=\textwidth]{NOISeq-fig_boxplot1}</td>
      </tr>
      <tr>
        <td id="L314" class="blob-num js-line-number" data-line-number="314"></td>
        <td id="LC314" class="blob-code blob-code-inner js-file-line">\caption{Count distribution per biotype in one of the samples (for genes with more than 0 counts). At the upper part of the plot, the number of detected features within each biotype group is displayed.}</td>
      </tr>
      <tr>
        <td id="L315" class="blob-num js-line-number" data-line-number="315"></td>
        <td id="LC315" class="blob-code blob-code-inner js-file-line">\label{fig_boxplot1}</td>
      </tr>
      <tr>
        <td id="L316" class="blob-num js-line-number" data-line-number="316"></td>
        <td id="LC316" class="blob-code blob-code-inner js-file-line">\end{figure}</td>
      </tr>
      <tr>
        <td id="L317" class="blob-num js-line-number" data-line-number="317"></td>
        <td id="LC317" class="blob-code blob-code-inner js-file-line">
</td>
      </tr>
      <tr>
        <td id="L318" class="blob-num js-line-number" data-line-number="318"></td>
        <td id="LC318" class="blob-code blob-code-inner js-file-line">
</td>
      </tr>
      <tr>
        <td id="L319" class="blob-num js-line-number" data-line-number="319"></td>
        <td id="LC319" class="blob-code blob-code-inner js-file-line">
</td>
      </tr>
      <tr>
        <td id="L320" class="blob-num js-line-number" data-line-number="320"></td>
        <td id="LC320" class="blob-code blob-code-inner js-file-line">
</td>
      </tr>
      <tr>
        <td id="L321" class="blob-num js-line-number" data-line-number="321"></td>
        <td id="LC321" class="blob-code blob-code-inner js-file-line">% \clearpage</td>
      </tr>
      <tr>
        <td id="L322" class="blob-num js-line-number" data-line-number="322"></td>
        <td id="LC322" class="blob-code blob-code-inner js-file-line">
</td>
      </tr>
      <tr>
        <td id="L323" class="blob-num js-line-number" data-line-number="323"></td>
        <td id="LC323" class="blob-code blob-code-inner js-file-line">\subsection{Sequencing depth \&amp; Expression Quantification}</td>
      </tr>
      <tr>
        <td id="L324" class="blob-num js-line-number" data-line-number="324"></td>
        <td id="LC324" class="blob-code blob-code-inner js-file-line">
</td>
      </tr>
      <tr>
        <td id="L325" class="blob-num js-line-number" data-line-number="325"></td>
        <td id="LC325" class="blob-code blob-code-inner js-file-line">The plots in this section can be generated by only providing the expression data, since no other biological information is required. </td>
      </tr>
      <tr>
        <td id="L326" class="blob-num js-line-number" data-line-number="326"></td>
        <td id="LC326" class="blob-code blob-code-inner js-file-line">Their purpose is to assess if the sequencing depth of the samples is enough to detect the features of interest and to get a good </td>
      </tr>
      <tr>
        <td id="L327" class="blob-num js-line-number" data-line-number="327"></td>
        <td id="LC327" class="blob-code blob-code-inner js-file-line">quantification of their expression.</td>
      </tr>
      <tr>
        <td id="L328" class="blob-num js-line-number" data-line-number="328"></td>
        <td id="LC328" class="blob-code blob-code-inner js-file-line">
</td>
      </tr>
      <tr>
        <td id="L329" class="blob-num js-line-number" data-line-number="329"></td>
        <td id="LC329" class="blob-code blob-code-inner js-file-line">
</td>
      </tr>
      <tr>
        <td id="L330" class="blob-num js-line-number" data-line-number="330"></td>
        <td id="LC330" class="blob-code blob-code-inner js-file-line">\subsubsection{Saturation plot}</td>
      </tr>
      <tr>
        <td id="L331" class="blob-num js-line-number" data-line-number="331"></td>
        <td id="LC331" class="blob-code blob-code-inner js-file-line">
</td>
      </tr>
      <tr>
        <td id="L332" class="blob-num js-line-number" data-line-number="332"></td>
        <td id="LC332" class="blob-code blob-code-inner js-file-line">The ``Saturation&quot; plot shows the number of features in the genome detected with more than \texttt{k} counts with the sequencing depth of the sample, and with higher and lower simulated sequencing depths.  </td>
      </tr>
      <tr>
        <td id="L333" class="blob-num js-line-number" data-line-number="333"></td>
        <td id="LC333" class="blob-code blob-code-inner js-file-line">This plot can be generated by considering either all the features or only the features included in a given biological group (biotype), if this information is available.</td>
      </tr>
      <tr>
        <td id="L334" class="blob-num js-line-number" data-line-number="334"></td>
        <td id="LC334" class="blob-code blob-code-inner js-file-line">First, we have to generate the saturation data with the function \code{dat} and then we can use the resulting object to obtain, </td>
      </tr>
      <tr>
        <td id="L335" class="blob-num js-line-number" data-line-number="335"></td>
        <td id="LC335" class="blob-code blob-code-inner js-file-line">for instance, the plots in Fig. \ref{fig_sat1} and \ref{fig_sat2} by applying \code{explo.plot} function. The lines show how the number of detected features increases with depth. When the number of samples to plot is 1 or 2, </td>
      </tr>
      <tr>
        <td id="L336" class="blob-num js-line-number" data-line-number="336"></td>
        <td id="LC336" class="blob-code blob-code-inner js-file-line">bars indicating the number of new features detected when increasing the sequencing depth in one million of reads are also drawn. In that case, lines values are to be read in the left Y axis and bar values in the right Y axis. </td>
      </tr>
      <tr>
        <td id="L337" class="blob-num js-line-number" data-line-number="337"></td>
        <td id="LC337" class="blob-code blob-code-inner js-file-line">If more than 2 samples are to be plotted, it is difficult to visualize the ``newdetection bars&#39;&#39;, </td>
      </tr>
      <tr>
        <td id="L338" class="blob-num js-line-number" data-line-number="338"></td>
        <td id="LC338" class="blob-code blob-code-inner js-file-line">so only the lines are shown in the plot.</td>
      </tr>
      <tr>
        <td id="L339" class="blob-num js-line-number" data-line-number="339"></td>
        <td id="LC339" class="blob-code blob-code-inner js-file-line">
</td>
      </tr>
      <tr>
        <td id="L340" class="blob-num js-line-number" data-line-number="340"></td>
        <td id="LC340" class="blob-code blob-code-inner js-file-line">&lt;&lt;fig_sat1,fig=TRUE&gt;&gt;=</td>
      </tr>
      <tr>
        <td id="L341" class="blob-num js-line-number" data-line-number="341"></td>
        <td id="LC341" class="blob-code blob-code-inner js-file-line">mysaturation = dat(mydata, k = 0, ndepth = 7, type = &quot;saturation&quot;)</td>
      </tr>
      <tr>
        <td id="L342" class="blob-num js-line-number" data-line-number="342"></td>
        <td id="LC342" class="blob-code blob-code-inner js-file-line">explo.plot(mysaturation, toplot = 1, samples = 1:2, yleftlim = NULL, yrightlim = NULL)</td>
      </tr>
      <tr>
        <td id="L343" class="blob-num js-line-number" data-line-number="343"></td>
        <td id="LC343" class="blob-code blob-code-inner js-file-line">@</td>
      </tr>
      <tr>
        <td id="L344" class="blob-num js-line-number" data-line-number="344"></td>
        <td id="LC344" class="blob-code blob-code-inner js-file-line">&lt;&lt;fig_sat2,fig=TRUE&gt;&gt;=</td>
      </tr>
      <tr>
        <td id="L345" class="blob-num js-line-number" data-line-number="345"></td>
        <td id="LC345" class="blob-code blob-code-inner js-file-line">explo.plot(mysaturation, toplot = &quot;protein_coding&quot;, samples = 1:4)</td>
      </tr>
      <tr>
        <td id="L346" class="blob-num js-line-number" data-line-number="346"></td>
        <td id="LC346" class="blob-code blob-code-inner js-file-line">@ </td>
      </tr>
      <tr>
        <td id="L347" class="blob-num js-line-number" data-line-number="347"></td>
        <td id="LC347" class="blob-code blob-code-inner js-file-line">
</td>
      </tr>
      <tr>
        <td id="L348" class="blob-num js-line-number" data-line-number="348"></td>
        <td id="LC348" class="blob-code blob-code-inner js-file-line">The plot in Fig. \ref{fig_sat1} has been computed for all the features (without specifying a biotype) and for two of the samples. </td>
      </tr>
      <tr>
        <td id="L349" class="blob-num js-line-number" data-line-number="349"></td>
        <td id="LC349" class="blob-code blob-code-inner js-file-line">Left Y axis shows the number of detected genes with more than 0 counts at each sequencing depth, represented by the lines. </td>
      </tr>
      <tr>
        <td id="L350" class="blob-num js-line-number" data-line-number="350"></td>
        <td id="LC350" class="blob-code blob-code-inner js-file-line">The solid point in each line corresponds to the real available sequencing depth. The other sequencing depths are simulated </td>
      </tr>
      <tr>
        <td id="L351" class="blob-num js-line-number" data-line-number="351"></td>
        <td id="LC351" class="blob-code blob-code-inner js-file-line">from this total sequencing depth. The bars are associated to the right Y axis and show the number of new features detected per million of new sequenced reads</td>
      </tr>
      <tr>
        <td id="L352" class="blob-num js-line-number" data-line-number="352"></td>
        <td id="LC352" class="blob-code blob-code-inner js-file-line">at each sequencing depth. The legend in the gray box also indicates the percentage of total features detected with more than $k=0$ counts </td>
      </tr>
      <tr>
        <td id="L353" class="blob-num js-line-number" data-line-number="353"></td>
        <td id="LC353" class="blob-code blob-code-inner js-file-line">at the real sequencing depth.</td>
      </tr>
      <tr>
        <td id="L354" class="blob-num js-line-number" data-line-number="354"></td>
        <td id="LC354" class="blob-code blob-code-inner js-file-line">
</td>
      </tr>
      <tr>
        <td id="L355" class="blob-num js-line-number" data-line-number="355"></td>
        <td id="LC355" class="blob-code blob-code-inner js-file-line">Up to twelve samples can be displayed in this plot. In Fig. \ref{fig_sat2}, four samples are compared and we can see, for instance, </td>
      </tr>
      <tr>
        <td id="L356" class="blob-num js-line-number" data-line-number="356"></td>
        <td id="LC356" class="blob-code blob-code-inner js-file-line">that in kidney samples the number of detected features is higher than in liver samples.  </td>
      </tr>
      <tr>
        <td id="L357" class="blob-num js-line-number" data-line-number="357"></td>
        <td id="LC357" class="blob-code blob-code-inner js-file-line">
</td>
      </tr>
      <tr>
        <td id="L358" class="blob-num js-line-number" data-line-number="358"></td>
        <td id="LC358" class="blob-code blob-code-inner js-file-line">\begin{figure}[ht!]</td>
      </tr>
      <tr>
        <td id="L359" class="blob-num js-line-number" data-line-number="359"></td>
        <td id="LC359" class="blob-code blob-code-inner js-file-line">\centering</td>
      </tr>
      <tr>
        <td id="L360" class="blob-num js-line-number" data-line-number="360"></td>
        <td id="LC360" class="blob-code blob-code-inner js-file-line">\includegraphics[width=0.5\textwidth]{NOISeq-fig_sat1}</td>
      </tr>
      <tr>
        <td id="L361" class="blob-num js-line-number" data-line-number="361"></td>
        <td id="LC361" class="blob-code blob-code-inner js-file-line">\caption{Global saturation plot to compare two samples of kidney and liver, respectively.}</td>
      </tr>
      <tr>
        <td id="L362" class="blob-num js-line-number" data-line-number="362"></td>
        <td id="LC362" class="blob-code blob-code-inner js-file-line">\label{fig_sat1}</td>
      </tr>
      <tr>
        <td id="L363" class="blob-num js-line-number" data-line-number="363"></td>
        <td id="LC363" class="blob-code blob-code-inner js-file-line">\end{figure}</td>
      </tr>
      <tr>
        <td id="L364" class="blob-num js-line-number" data-line-number="364"></td>
        <td id="LC364" class="blob-code blob-code-inner js-file-line">
</td>
      </tr>
      <tr>
        <td id="L365" class="blob-num js-line-number" data-line-number="365"></td>
        <td id="LC365" class="blob-code blob-code-inner js-file-line">
</td>
      </tr>
      <tr>
        <td id="L366" class="blob-num js-line-number" data-line-number="366"></td>
        <td id="LC366" class="blob-code blob-code-inner js-file-line">\begin{figure}[ht!]</td>
      </tr>
      <tr>
        <td id="L367" class="blob-num js-line-number" data-line-number="367"></td>
        <td id="LC367" class="blob-code blob-code-inner js-file-line">\centering</td>
      </tr>
      <tr>
        <td id="L368" class="blob-num js-line-number" data-line-number="368"></td>
        <td id="LC368" class="blob-code blob-code-inner js-file-line">\includegraphics[width=0.5\textwidth]{NOISeq-fig_sat2}</td>
      </tr>
      <tr>
        <td id="L369" class="blob-num js-line-number" data-line-number="369"></td>
        <td id="LC369" class="blob-code blob-code-inner js-file-line">\caption{Saturation plot for protein-coding genes to compare 4 samples: 2 of kidney and 2 of liver.}</td>
      </tr>
      <tr>
        <td id="L370" class="blob-num js-line-number" data-line-number="370"></td>
        <td id="LC370" class="blob-code blob-code-inner js-file-line">\label{fig_sat2}</td>
      </tr>
      <tr>
        <td id="L371" class="blob-num js-line-number" data-line-number="371"></td>
        <td id="LC371" class="blob-code blob-code-inner js-file-line">\end{figure}</td>
      </tr>
      <tr>
        <td id="L372" class="blob-num js-line-number" data-line-number="372"></td>
        <td id="LC372" class="blob-code blob-code-inner js-file-line">
</td>
      </tr>
      <tr>
        <td id="L373" class="blob-num js-line-number" data-line-number="373"></td>
        <td id="LC373" class="blob-code blob-code-inner js-file-line">
</td>
      </tr>
      <tr>
        <td id="L374" class="blob-num js-line-number" data-line-number="374"></td>
        <td id="LC374" class="blob-code blob-code-inner js-file-line">\subsubsection{Count distribution per sample}</td>
      </tr>
      <tr>
        <td id="L375" class="blob-num js-line-number" data-line-number="375"></td>
        <td id="LC375" class="blob-code blob-code-inner js-file-line">
</td>
      </tr>
      <tr>
        <td id="L376" class="blob-num js-line-number" data-line-number="376"></td>
        <td id="LC376" class="blob-code blob-code-inner js-file-line">It is also interesting to visualize the count distribution for all the samples, either for all the features or for the features belonging to a certain biological group (biotype). Fig. \ref{fig_boxplot2} shows this information for the biotype ``protein\_coding&quot;, which can be generated with the following code on the ``countsbio&quot; object obtained in the previous section by setting the \texttt{samples} parameter to  \texttt{NULL}.</td>
      </tr>
      <tr>
        <td id="L377" class="blob-num js-line-number" data-line-number="377"></td>
        <td id="LC377" class="blob-code blob-code-inner js-file-line">&lt;&lt;fig_boxplot2,fig=TRUE&gt;&gt;=</td>
      </tr>
      <tr>
        <td id="L378" class="blob-num js-line-number" data-line-number="378"></td>
        <td id="LC378" class="blob-code blob-code-inner js-file-line">explo.plot(mycountsbio, toplot = &quot;protein_coding&quot;, samples = NULL, plottype = &quot;boxplot&quot;)</td>
      </tr>
      <tr>
        <td id="L379" class="blob-num js-line-number" data-line-number="379"></td>
        <td id="LC379" class="blob-code blob-code-inner js-file-line">@ </td>
      </tr>
      <tr>
        <td id="L380" class="blob-num js-line-number" data-line-number="380"></td>
        <td id="LC380" class="blob-code blob-code-inner js-file-line">
</td>
      </tr>
      <tr>
        <td id="L381" class="blob-num js-line-number" data-line-number="381"></td>
        <td id="LC381" class="blob-code blob-code-inner js-file-line">\begin{figure}[ht!]</td>
      </tr>
      <tr>
        <td id="L382" class="blob-num js-line-number" data-line-number="382"></td>
        <td id="LC382" class="blob-code blob-code-inner js-file-line">\centering</td>
      </tr>
      <tr>
        <td id="L383" class="blob-num js-line-number" data-line-number="383"></td>
        <td id="LC383" class="blob-code blob-code-inner js-file-line">\includegraphics[width=0.45\textwidth]{NOISeq-fig_boxplot2}</td>
      </tr>
      <tr>
        <td id="L384" class="blob-num js-line-number" data-line-number="384"></td>
        <td id="LC384" class="blob-code blob-code-inner js-file-line">\caption{Distribution of counts for protein coding genes in all samples.}</td>
      </tr>
      <tr>
        <td id="L385" class="blob-num js-line-number" data-line-number="385"></td>
        <td id="LC385" class="blob-code blob-code-inner js-file-line">\label{fig_boxplot2}</td>
      </tr>
      <tr>
        <td id="L386" class="blob-num js-line-number" data-line-number="386"></td>
        <td id="LC386" class="blob-code blob-code-inner js-file-line">\end{figure}</td>
      </tr>
      <tr>
        <td id="L387" class="blob-num js-line-number" data-line-number="387"></td>
        <td id="LC387" class="blob-code blob-code-inner js-file-line">
</td>
      </tr>
      <tr>
        <td id="L388" class="blob-num js-line-number" data-line-number="388"></td>
        <td id="LC388" class="blob-code blob-code-inner js-file-line">
</td>
      </tr>
      <tr>
        <td id="L389" class="blob-num js-line-number" data-line-number="389"></td>
        <td id="LC389" class="blob-code blob-code-inner js-file-line">
</td>
      </tr>
      <tr>
        <td id="L390" class="blob-num js-line-number" data-line-number="390"></td>
        <td id="LC390" class="blob-code blob-code-inner js-file-line">\subsubsection{Sensitivity plot}</td>
      </tr>
      <tr>
        <td id="L391" class="blob-num js-line-number" data-line-number="391"></td>
        <td id="LC391" class="blob-code blob-code-inner js-file-line">
</td>
      </tr>
      <tr>
        <td id="L392" class="blob-num js-line-number" data-line-number="392"></td>
        <td id="LC392" class="blob-code blob-code-inner js-file-line">Features with low counts are, in general, less reliable and may introduce noise in the data that makes more difficult to extract the relevant information, </td>
      </tr>
      <tr>
        <td id="L393" class="blob-num js-line-number" data-line-number="393"></td>
        <td id="LC393" class="blob-code blob-code-inner js-file-line">for instance, the differentially expressed features. We have implemented some methods in the \noiseq{} package to filter out these low count features. The ``Sensitivity plot&#39;&#39; in Fig. \ref{fig_boxplot3} helps to decide the threshold to remove low-count features by indicating the proportion of such features that are present in our data. In this plot, the bars show the percentage of features within each sample having more than 0 counts per million (CPM), or more than 1, 2, 5 and 10 CPM. The horizontal lines are the corresponding percentage of features with those CPM in at least one of the samples (or experimental conditions if the \texttt{factor} parameter is not \texttt{NULL}). In the upper side of the plot, the sequencing depth of each sample (in million reads) is given. </td>
      </tr>
      <tr>
        <td id="L394" class="blob-num js-line-number" data-line-number="394"></td>
        <td id="LC394" class="blob-code blob-code-inner js-file-line">The following code can be used for drawing this figure.</td>
      </tr>
      <tr>
        <td id="L395" class="blob-num js-line-number" data-line-number="395"></td>
        <td id="LC395" class="blob-code blob-code-inner js-file-line"> </td>
      </tr>
      <tr>
        <td id="L396" class="blob-num js-line-number" data-line-number="396"></td>
        <td id="LC396" class="blob-code blob-code-inner js-file-line">&lt;&lt;fig_boxplot3,fig=TRUE&gt;&gt;=</td>
      </tr>
      <tr>
        <td id="L397" class="blob-num js-line-number" data-line-number="397"></td>
        <td id="LC397" class="blob-code blob-code-inner js-file-line">explo.plot(mycountsbio, toplot = 1, samples = NULL, plottype = &quot;barplot&quot;)</td>
      </tr>
      <tr>
        <td id="L398" class="blob-num js-line-number" data-line-number="398"></td>
        <td id="LC398" class="blob-code blob-code-inner js-file-line">@ </td>
      </tr>
      <tr>
        <td id="L399" class="blob-num js-line-number" data-line-number="399"></td>
        <td id="LC399" class="blob-code blob-code-inner js-file-line">
</td>
      </tr>
      <tr>
        <td id="L400" class="blob-num js-line-number" data-line-number="400"></td>
        <td id="LC400" class="blob-code blob-code-inner js-file-line">\begin{figure}[ht!]</td>
      </tr>
      <tr>
        <td id="L401" class="blob-num js-line-number" data-line-number="401"></td>
        <td id="LC401" class="blob-code blob-code-inner js-file-line">\centering</td>
      </tr>
      <tr>
        <td id="L402" class="blob-num js-line-number" data-line-number="402"></td>
        <td id="LC402" class="blob-code blob-code-inner js-file-line">\includegraphics[width=0.45\textwidth]{NOISeq-fig_boxplot3}</td>
      </tr>
      <tr>
        <td id="L403" class="blob-num js-line-number" data-line-number="403"></td>
        <td id="LC403" class="blob-code blob-code-inner js-file-line">\caption{Number of features with low counts for each sample.}</td>
      </tr>
      <tr>
        <td id="L404" class="blob-num js-line-number" data-line-number="404"></td>
        <td id="LC404" class="blob-code blob-code-inner js-file-line">\label{fig_boxplot3}</td>
      </tr>
      <tr>
        <td id="L405" class="blob-num js-line-number" data-line-number="405"></td>
        <td id="LC405" class="blob-code blob-code-inner js-file-line">\end{figure}</td>
      </tr>
      <tr>
        <td id="L406" class="blob-num js-line-number" data-line-number="406"></td>
        <td id="LC406" class="blob-code blob-code-inner js-file-line">
</td>
      </tr>
      <tr>
        <td id="L407" class="blob-num js-line-number" data-line-number="407"></td>
        <td id="LC407" class="blob-code blob-code-inner js-file-line">
</td>
      </tr>
      <tr>
        <td id="L408" class="blob-num js-line-number" data-line-number="408"></td>
        <td id="LC408" class="blob-code blob-code-inner js-file-line">% \clearpage</td>
      </tr>
      <tr>
        <td id="L409" class="blob-num js-line-number" data-line-number="409"></td>
        <td id="LC409" class="blob-code blob-code-inner js-file-line">
</td>
      </tr>
      <tr>
        <td id="L410" class="blob-num js-line-number" data-line-number="410"></td>
        <td id="LC410" class="blob-code blob-code-inner js-file-line">
</td>
      </tr>
      <tr>
        <td id="L411" class="blob-num js-line-number" data-line-number="411"></td>
        <td id="LC411" class="blob-code blob-code-inner js-file-line">\subsection{Sequencing bias detection}</td>
      </tr>
      <tr>
        <td id="L412" class="blob-num js-line-number" data-line-number="412"></td>
        <td id="LC412" class="blob-code blob-code-inner js-file-line">
</td>
      </tr>
      <tr>
        <td id="L413" class="blob-num js-line-number" data-line-number="413"></td>
        <td id="LC413" class="blob-code blob-code-inner js-file-line">Prior to perform further analyses such as differential expression, it is essential to normalize data to make the samples comparable and </td>
      </tr>
      <tr>
        <td id="L414" class="blob-num js-line-number" data-line-number="414"></td>
        <td id="LC414" class="blob-code blob-code-inner js-file-line">remove the effect of technical biases from the expression estimation. The plots presented in this section are very useful for detecting the possible biases in</td>
      </tr>
      <tr>
        <td id="L415" class="blob-num js-line-number" data-line-number="415"></td>
        <td id="LC415" class="blob-code blob-code-inner js-file-line">the data. In particular, the biases that can be studied are: the feature length effect, the GC content and the differences in RNA composition. </td>
      </tr>
      <tr>
        <td id="L416" class="blob-num js-line-number" data-line-number="416"></td>
        <td id="LC416" class="blob-code blob-code-inner js-file-line">In addition, these are diagnostic plots, which means that they are not only descriptive but an statistical test is also conducted to help the user to decide whether the bias is present and the data needs normalization.</td>
      </tr>
      <tr>
        <td id="L417" class="blob-num js-line-number" data-line-number="417"></td>
        <td id="LC417" class="blob-code blob-code-inner js-file-line">
</td>
      </tr>
      <tr>
        <td id="L418" class="blob-num js-line-number" data-line-number="418"></td>
        <td id="LC418" class="blob-code blob-code-inner js-file-line">
</td>
      </tr>
      <tr>
        <td id="L419" class="blob-num js-line-number" data-line-number="419"></td>
        <td id="LC419" class="blob-code blob-code-inner js-file-line">\subsubsection{Length bias}</td>
      </tr>
      <tr>
        <td id="L420" class="blob-num js-line-number" data-line-number="420"></td>
        <td id="LC420" class="blob-code blob-code-inner js-file-line">
</td>
      </tr>
      <tr>
        <td id="L421" class="blob-num js-line-number" data-line-number="421"></td>
        <td id="LC421" class="blob-code blob-code-inner js-file-line">The ``lengthbias&quot; plot describes the relationship between the feature length and the expression values. </td>
      </tr>
      <tr>
        <td id="L422" class="blob-num js-line-number" data-line-number="422"></td>
        <td id="LC422" class="blob-code blob-code-inner js-file-line">Hence, the feature length must be included in the input object created using the \code{readData} function. The data for this plot is generated as follows. </td>
      </tr>
      <tr>
        <td id="L423" class="blob-num js-line-number" data-line-number="423"></td>
        <td id="LC423" class="blob-code blob-code-inner js-file-line">The length is divided in intervals (bins) containing 200 features and the middle point of each bin is depicted in X axis. </td>
      </tr>
      <tr>
        <td id="L424" class="blob-num js-line-number" data-line-number="424"></td>
        <td id="LC424" class="blob-code blob-code-inner js-file-line">For each bin, the 5\% trimmed mean of the corresponding expression values (CPM if \texttt{norm=FALSE} or values provided if \texttt{norm=TRUE}) is computed and depicted in Y axis. If the number of samples or conditions </td>
      </tr>
      <tr>
        <td id="L425" class="blob-num js-line-number" data-line-number="425"></td>
        <td id="LC425" class="blob-code blob-code-inner js-file-line">to appear in the plot is 2 or less and no biotype is specified (toplot = ``global&quot;), a diagnostic test is provided. </td>
      </tr>
      <tr>
        <td id="L426" class="blob-num js-line-number" data-line-number="426"></td>
        <td id="LC426" class="blob-code blob-code-inner js-file-line">A cubic spline regression model is fitted to explain the relationship between length and expression. </td>
      </tr>
      <tr>
        <td id="L427" class="blob-num js-line-number" data-line-number="427"></td>
        <td id="LC427" class="blob-code blob-code-inner js-file-line">Both the model p-value and the coefficient of determination (R2) are shown in the plot as well as the fitted regression curve. </td>
      </tr>
      <tr>
        <td id="L428" class="blob-num js-line-number" data-line-number="428"></td>
        <td id="LC428" class="blob-code blob-code-inner js-file-line">If the model p-value is significant and R2 value is high (more than 70\%), the expression depends on the feature length and the curve shows the type of dependence. </td>
      </tr>
      <tr>
        <td id="L429" class="blob-num js-line-number" data-line-number="429"></td>
        <td id="LC429" class="blob-code blob-code-inner js-file-line"> </td>
      </tr>
      <tr>
        <td id="L430" class="blob-num js-line-number" data-line-number="430"></td>
        <td id="LC430" class="blob-code blob-code-inner js-file-line">Fig. \ref{fig_length} shows an example of this plot. In this case, the ``lengthbias&quot; data were generated for each condition (kidney and liver) using the argument \texttt{factor}.</td>
      </tr>
      <tr>
        <td id="L431" class="blob-num js-line-number" data-line-number="431"></td>
        <td id="LC431" class="blob-code blob-code-inner js-file-line">
</td>
      </tr>
      <tr>
        <td id="L432" class="blob-num js-line-number" data-line-number="432"></td>
        <td id="LC432" class="blob-code blob-code-inner js-file-line">&lt;&lt;fig_length,results=hide,fig=TRUE&gt;&gt;=</td>
      </tr>
      <tr>
        <td id="L433" class="blob-num js-line-number" data-line-number="433"></td>
        <td id="LC433" class="blob-code blob-code-inner js-file-line">mylengthbias = dat(mydata, factor = &quot;Tissue&quot;, type = &quot;lengthbias&quot;)</td>
      </tr>
      <tr>
        <td id="L434" class="blob-num js-line-number" data-line-number="434"></td>
        <td id="LC434" class="blob-code blob-code-inner js-file-line">explo.plot(mylengthbias, samples = NULL, toplot = &quot;global&quot;)</td>
      </tr>
      <tr>
        <td id="L435" class="blob-num js-line-number" data-line-number="435"></td>
        <td id="LC435" class="blob-code blob-code-inner js-file-line">@ </td>
      </tr>
      <tr>
        <td id="L436" class="blob-num js-line-number" data-line-number="436"></td>
        <td id="LC436" class="blob-code blob-code-inner js-file-line">
</td>
      </tr>
      <tr>
        <td id="L437" class="blob-num js-line-number" data-line-number="437"></td>
        <td id="LC437" class="blob-code blob-code-inner js-file-line">\begin{figure}[ht]</td>
      </tr>
      <tr>
        <td id="L438" class="blob-num js-line-number" data-line-number="438"></td>
        <td id="LC438" class="blob-code blob-code-inner js-file-line">\centering</td>
      </tr>
      <tr>
        <td id="L439" class="blob-num js-line-number" data-line-number="439"></td>
        <td id="LC439" class="blob-code blob-code-inner js-file-line">\includegraphics[width=\textwidth, height=0.5\textwidth]{NOISeq-fig_length}</td>
      </tr>
      <tr>
        <td id="L440" class="blob-num js-line-number" data-line-number="440"></td>
        <td id="LC440" class="blob-code blob-code-inner js-file-line">\caption{Gene length versus expression.}</td>
      </tr>
      <tr>
        <td id="L441" class="blob-num js-line-number" data-line-number="441"></td>
        <td id="LC441" class="blob-code blob-code-inner js-file-line">\label{fig_length}</td>
      </tr>
      <tr>
        <td id="L442" class="blob-num js-line-number" data-line-number="442"></td>
        <td id="LC442" class="blob-code blob-code-inner js-file-line">\end{figure}</td>
      </tr>
      <tr>
        <td id="L443" class="blob-num js-line-number" data-line-number="443"></td>
        <td id="LC443" class="blob-code blob-code-inner js-file-line">
</td>
      </tr>
      <tr>
        <td id="L444" class="blob-num js-line-number" data-line-number="444"></td>
        <td id="LC444" class="blob-code blob-code-inner js-file-line">More details about the fitted spline regression models can be obtained by using the \code{show} function as per below:</td>
      </tr>
      <tr>
        <td id="L445" class="blob-num js-line-number" data-line-number="445"></td>
        <td id="LC445" class="blob-code blob-code-inner js-file-line">
</td>
      </tr>
      <tr>
        <td id="L446" class="blob-num js-line-number" data-line-number="446"></td>
        <td id="LC446" class="blob-code blob-code-inner js-file-line">&lt;&lt;showmodels&gt;&gt;=</td>
      </tr>
      <tr>
        <td id="L447" class="blob-num js-line-number" data-line-number="447"></td>
        <td id="LC447" class="blob-code blob-code-inner js-file-line">show(mylengthbias)</td>
      </tr>
      <tr>
        <td id="L448" class="blob-num js-line-number" data-line-number="448"></td>
        <td id="LC448" class="blob-code blob-code-inner js-file-line">@ </td>
      </tr>
      <tr>
        <td id="L449" class="blob-num js-line-number" data-line-number="449"></td>
        <td id="LC449" class="blob-code blob-code-inner js-file-line">
</td>
      </tr>
      <tr>
        <td id="L450" class="blob-num js-line-number" data-line-number="450"></td>
        <td id="LC450" class="blob-code blob-code-inner js-file-line">
</td>
      </tr>
      <tr>
        <td id="L451" class="blob-num js-line-number" data-line-number="451"></td>
        <td id="LC451" class="blob-code blob-code-inner js-file-line">
</td>
      </tr>
      <tr>
        <td id="L452" class="blob-num js-line-number" data-line-number="452"></td>
        <td id="LC452" class="blob-code blob-code-inner js-file-line">
</td>
      </tr>
      <tr>
        <td id="L453" class="blob-num js-line-number" data-line-number="453"></td>
        <td id="LC453" class="blob-code blob-code-inner js-file-line">\subsubsection{GC content bias}</td>
      </tr>
      <tr>
        <td id="L454" class="blob-num js-line-number" data-line-number="454"></td>
        <td id="LC454" class="blob-code blob-code-inner js-file-line">
</td>
      </tr>
      <tr>
        <td id="L455" class="blob-num js-line-number" data-line-number="455"></td>
        <td id="LC455" class="blob-code blob-code-inner js-file-line">The ``GCbias&quot; plot describes the relationship between the feature GC content and the expression values. Hence, the feature GC</td>
      </tr>
      <tr>
        <td id="L456" class="blob-num js-line-number" data-line-number="456"></td>
        <td id="LC456" class="blob-code blob-code-inner js-file-line">content must be included in the input object created using the \code{readData} function. </td>
      </tr>
      <tr>
        <td id="L457" class="blob-num js-line-number" data-line-number="457"></td>
        <td id="LC457" class="blob-code blob-code-inner js-file-line">The data for this plot is generated in an analogous way to the ``lengthbias&quot; data. </td>
      </tr>
      <tr>
        <td id="L458" class="blob-num js-line-number" data-line-number="458"></td>
        <td id="LC458" class="blob-code blob-code-inner js-file-line">The GC content is divided in intervals (bins) containing 200 features. The middle point of each bin is</td>
      </tr>
      <tr>
        <td id="L459" class="blob-num js-line-number" data-line-number="459"></td>
        <td id="LC459" class="blob-code blob-code-inner js-file-line">depicted in X axis. For each bin, the 5\% trimmed mean of the corresponding expression values is</td>
      </tr>
      <tr>
        <td id="L460" class="blob-num js-line-number" data-line-number="460"></td>
        <td id="LC460" class="blob-code blob-code-inner js-file-line">computed and depicted in Y axis. If the number of samples or conditions to appear in the plot is 2 or less and no biotype is specified </td>
      </tr>
      <tr>
        <td id="L461" class="blob-num js-line-number" data-line-number="461"></td>
        <td id="LC461" class="blob-code blob-code-inner js-file-line">(toplot = ``global&quot;), a diagnostic test is provided. A cubic spline regression model is fitted to explain the relationship between GC content and expression. </td>
      </tr>
      <tr>
        <td id="L462" class="blob-num js-line-number" data-line-number="462"></td>
        <td id="LC462" class="blob-code blob-code-inner js-file-line">Both the model p-value and the coefficient of determination (R2) are shown in the plot as well as the fitted regression curve. </td>
      </tr>
      <tr>
        <td id="L463" class="blob-num js-line-number" data-line-number="463"></td>
        <td id="LC463" class="blob-code blob-code-inner js-file-line">If the model p-value is significant and R2 value is high (more than 70\%), the expression will depend on the feature GC content </td>
      </tr>
      <tr>
        <td id="L464" class="blob-num js-line-number" data-line-number="464"></td>
        <td id="LC464" class="blob-code blob-code-inner js-file-line">and the curve will show the type of dependence. </td>
      </tr>
      <tr>
        <td id="L465" class="blob-num js-line-number" data-line-number="465"></td>
        <td id="LC465" class="blob-code blob-code-inner js-file-line"> </td>
      </tr>
      <tr>
        <td id="L466" class="blob-num js-line-number" data-line-number="466"></td>
        <td id="LC466" class="blob-code blob-code-inner js-file-line">An example of this plot is in Fig. \ref{fig_GC}. In this case, the ``GCbias&quot; data were also generated for each condition </td>
      </tr>
      <tr>
        <td id="L467" class="blob-num js-line-number" data-line-number="467"></td>
        <td id="LC467" class="blob-code blob-code-inner js-file-line">(kidney and liver) using the argument \texttt{factor}.</td>
      </tr>
      <tr>
        <td id="L468" class="blob-num js-line-number" data-line-number="468"></td>
        <td id="LC468" class="blob-code blob-code-inner js-file-line">
</td>
      </tr>
      <tr>
        <td id="L469" class="blob-num js-line-number" data-line-number="469"></td>
        <td id="LC469" class="blob-code blob-code-inner js-file-line">&lt;&lt;fig_GC,results=hide,fig=TRUE&gt;&gt;=</td>
      </tr>
      <tr>
        <td id="L470" class="blob-num js-line-number" data-line-number="470"></td>
        <td id="LC470" class="blob-code blob-code-inner js-file-line">myGCbias = dat(mydata, factor = &quot;Tissue&quot;, type = &quot;GCbias&quot;)</td>
      </tr>
      <tr>
        <td id="L471" class="blob-num js-line-number" data-line-number="471"></td>
        <td id="LC471" class="blob-code blob-code-inner js-file-line">explo.plot(myGCbias, samples = NULL, toplot = &quot;global&quot;)</td>
      </tr>
      <tr>
        <td id="L472" class="blob-num js-line-number" data-line-number="472"></td>
        <td id="LC472" class="blob-code blob-code-inner js-file-line">@ </td>
      </tr>
      <tr>
        <td id="L473" class="blob-num js-line-number" data-line-number="473"></td>
        <td id="LC473" class="blob-code blob-code-inner js-file-line">
</td>
      </tr>
      <tr>
        <td id="L474" class="blob-num js-line-number" data-line-number="474"></td>
        <td id="LC474" class="blob-code blob-code-inner js-file-line">\begin{figure}[ht]</td>
      </tr>
      <tr>
        <td id="L475" class="blob-num js-line-number" data-line-number="475"></td>
        <td id="LC475" class="blob-code blob-code-inner js-file-line">\centering</td>
      </tr>
      <tr>
        <td id="L476" class="blob-num js-line-number" data-line-number="476"></td>
        <td id="LC476" class="blob-code blob-code-inner js-file-line">\includegraphics[width=\textwidth, height=0.5\textwidth]{NOISeq-fig_GC}</td>
      </tr>
      <tr>
        <td id="L477" class="blob-num js-line-number" data-line-number="477"></td>
        <td id="LC477" class="blob-code blob-code-inner js-file-line">\caption{Gene GC content versus expression.}</td>
      </tr>
      <tr>
        <td id="L478" class="blob-num js-line-number" data-line-number="478"></td>
        <td id="LC478" class="blob-code blob-code-inner js-file-line">\label{fig_GC}</td>
      </tr>
      <tr>
        <td id="L479" class="blob-num js-line-number" data-line-number="479"></td>
        <td id="LC479" class="blob-code blob-code-inner js-file-line">\end{figure}</td>
      </tr>
      <tr>
        <td id="L480" class="blob-num js-line-number" data-line-number="480"></td>
        <td id="LC480" class="blob-code blob-code-inner js-file-line">
</td>
      </tr>
      <tr>
        <td id="L481" class="blob-num js-line-number" data-line-number="481"></td>
        <td id="LC481" class="blob-code blob-code-inner js-file-line">
</td>
      </tr>
      <tr>
        <td id="L482" class="blob-num js-line-number" data-line-number="482"></td>
        <td id="LC482" class="blob-code blob-code-inner js-file-line">
</td>
      </tr>
      <tr>
        <td id="L483" class="blob-num js-line-number" data-line-number="483"></td>
        <td id="LC483" class="blob-code blob-code-inner js-file-line">
</td>
      </tr>
      <tr>
        <td id="L484" class="blob-num js-line-number" data-line-number="484"></td>
        <td id="LC484" class="blob-code blob-code-inner js-file-line">\subsubsection{RNA composition}</td>
      </tr>
      <tr>
        <td id="L485" class="blob-num js-line-number" data-line-number="485"></td>
        <td id="LC485" class="blob-code blob-code-inner js-file-line">
</td>
      </tr>
      <tr>
        <td id="L486" class="blob-num js-line-number" data-line-number="486"></td>
        <td id="LC486" class="blob-code blob-code-inner js-file-line">When two samples have different RNA composition, the distribution of sequencing reads across the features is different in such a way that although </td>
      </tr>
      <tr>
        <td id="L487" class="blob-num js-line-number" data-line-number="487"></td>
        <td id="LC487" class="blob-code blob-code-inner js-file-line">a feature had the same number of read counts in both samples, it would not mean that it was equally expressed in both. </td>
      </tr>
      <tr>
        <td id="L488" class="blob-num js-line-number" data-line-number="488"></td>
        <td id="LC488" class="blob-code blob-code-inner js-file-line">To check if this bias is present in the data, the ``cd&quot; plot and the correponding diagnostic test can be used. In this case, each sample $s$ is compared to the reference sample $r$ (which can be arbitrarily chosen). To do that, M values are computed as $log2(counts_s=counts_r)$. </td>
      </tr>
      <tr>
        <td id="L489" class="blob-num js-line-number" data-line-number="489"></td>
        <td id="LC489" class="blob-code blob-code-inner js-file-line">If no bias is present, it should be expected that the median of M values for each comparison is 0. Otherwise, it would</td>
      </tr>
      <tr>
        <td id="L490" class="blob-num js-line-number" data-line-number="490"></td>
        <td id="LC490" class="blob-code blob-code-inner js-file-line">be indicating that expression levels in one of the samples tend to be higher than in the other, and this could lead to false discoveries </td>
      </tr>
      <tr>
        <td id="L491" class="blob-num js-line-number" data-line-number="491"></td>
        <td id="LC491" class="blob-code blob-code-inner js-file-line">when computing differencial expression. Confidence intervals for the M median are also computed by bootstrapping. If value 0 does not fall inside the interval, </td>
      </tr>
      <tr>
        <td id="L492" class="blob-num js-line-number" data-line-number="492"></td>
        <td id="LC492" class="blob-code blob-code-inner js-file-line">it means that the deviation of the sample with regard to the reference sample is statistically significant. </td>
      </tr>
      <tr>
        <td id="L493" class="blob-num js-line-number" data-line-number="493"></td>
        <td id="LC493" class="blob-code blob-code-inner js-file-line">Therefore, a normalization procedure such as Upper Quartile, TMM or DESeq should be used to correct this effect and make the samples comparable before computing differential expression. Confidence intervals can be visualized by using \texttt{show} function. </td>
      </tr>
      <tr>
        <td id="L494" class="blob-num js-line-number" data-line-number="494"></td>
        <td id="LC494" class="blob-code blob-code-inner js-file-line">
</td>
      </tr>
      <tr>
        <td id="L495" class="blob-num js-line-number" data-line-number="495"></td>
        <td id="LC495" class="blob-code blob-code-inner js-file-line">See below an usage example and the resulting plot in Fig. \ref{fig_countdistr}. It must be indicated if the data provided are already normalized (\texttt{norm=TRUE}) or not (\texttt{norm=FALSE}). The reference sample may be indicated with the refColumn parameter (by default, the first column is used). </td>
      </tr>
      <tr>
        <td id="L496" class="blob-num js-line-number" data-line-number="496"></td>
        <td id="LC496" class="blob-code blob-code-inner js-file-line">Additional plot parameters may also be used to modify some aspects of the plot.</td>
      </tr>
      <tr>
        <td id="L497" class="blob-num js-line-number" data-line-number="497"></td>
        <td id="LC497" class="blob-code blob-code-inner js-file-line">&lt;&lt;fig_countdistr,fig=TRUE&gt;&gt;=</td>
      </tr>
      <tr>
        <td id="L498" class="blob-num js-line-number" data-line-number="498"></td>
        <td id="LC498" class="blob-code blob-code-inner js-file-line">mycd = dat(mydata, type = &quot;cd&quot;, norm = FALSE, refColumn = 1)</td>
      </tr>
      <tr>
        <td id="L499" class="blob-num js-line-number" data-line-number="499"></td>
        <td id="LC499" class="blob-code blob-code-inner js-file-line">explo.plot(mycd)</td>
      </tr>
      <tr>
        <td id="L500" class="blob-num js-line-number" data-line-number="500"></td>
        <td id="LC500" class="blob-code blob-code-inner js-file-line">@ </td>
      </tr>
      <tr>
        <td id="L501" class="blob-num js-line-number" data-line-number="501"></td>
        <td id="LC501" class="blob-code blob-code-inner js-file-line">
</td>
      </tr>
      <tr>
        <td id="L502" class="blob-num js-line-number" data-line-number="502"></td>
        <td id="LC502" class="blob-code blob-code-inner js-file-line">\begin{figure}[ht]</td>
      </tr>
      <tr>
        <td id="L503" class="blob-num js-line-number" data-line-number="503"></td>
        <td id="LC503" class="blob-code blob-code-inner js-file-line">\centering</td>
      </tr>
      <tr>
        <td id="L504" class="blob-num js-line-number" data-line-number="504"></td>
        <td id="LC504" class="blob-code blob-code-inner js-file-line">\includegraphics[width=0.5\textwidth]{NOISeq-fig_countdistr}</td>
      </tr>
      <tr>
        <td id="L505" class="blob-num js-line-number" data-line-number="505"></td>
        <td id="LC505" class="blob-code blob-code-inner js-file-line">\caption{RNA composition plot}</td>
      </tr>
      <tr>
        <td id="L506" class="blob-num js-line-number" data-line-number="506"></td>
        <td id="LC506" class="blob-code blob-code-inner js-file-line">\label{fig_countdistr}</td>
      </tr>
      <tr>
        <td id="L507" class="blob-num js-line-number" data-line-number="507"></td>
        <td id="LC507" class="blob-code blob-code-inner js-file-line">\end{figure}</td>
      </tr>
      <tr>
        <td id="L508" class="blob-num js-line-number" data-line-number="508"></td>
        <td id="LC508" class="blob-code blob-code-inner js-file-line">
</td>
      </tr>
      <tr>
        <td id="L509" class="blob-num js-line-number" data-line-number="509"></td>
        <td id="LC509" class="blob-code blob-code-inner js-file-line">In the plot can be seen that the $M$ median is deviated from 0 in most of the cases. This is corraborated by the confidence intervals for the $M$ median.</td>
      </tr>
      <tr>
        <td id="L510" class="blob-num js-line-number" data-line-number="510"></td>
        <td id="LC510" class="blob-code blob-code-inner js-file-line">
</td>
      </tr>
      <tr>
        <td id="L511" class="blob-num js-line-number" data-line-number="511"></td>
        <td id="LC511" class="blob-code blob-code-inner js-file-line">% \clearpage</td>
      </tr>
      <tr>
        <td id="L512" class="blob-num js-line-number" data-line-number="512"></td>
        <td id="LC512" class="blob-code blob-code-inner js-file-line">
</td>
      </tr>
      <tr>
        <td id="L513" class="blob-num js-line-number" data-line-number="513"></td>
        <td id="LC513" class="blob-code blob-code-inner js-file-line">
</td>
      </tr>
      <tr>
        <td id="L514" class="blob-num js-line-number" data-line-number="514"></td>
        <td id="LC514" class="blob-code blob-code-inner js-file-line">\subsection{Batch effect exploration} \label{sec_PCA}</td>
      </tr>
      <tr>
        <td id="L515" class="blob-num js-line-number" data-line-number="515"></td>
        <td id="LC515" class="blob-code blob-code-inner js-file-line">
</td>
      </tr>
      <tr>
        <td id="L516" class="blob-num js-line-number" data-line-number="516"></td>
        <td id="LC516" class="blob-code blob-code-inner js-file-line">One of the techniques that can be used to visualize if the experimental samples are clustered according to the experimental design or if there is an unwanted source of noise in the data that hampers this clustering is the Principal Component Analysis (PCA).</td>
      </tr>
      <tr>
        <td id="L517" class="blob-num js-line-number" data-line-number="517"></td>
        <td id="LC517" class="blob-code blob-code-inner js-file-line">
</td>
      </tr>
      <tr>
        <td id="L518" class="blob-num js-line-number" data-line-number="518"></td>
        <td id="LC518" class="blob-code blob-code-inner js-file-line">To illustrate the utility of the PCA plots, we took Marioni&#39;s data and artificially added a batch effect to the first four samples that would belong then to bath 1. The rest of samples would belong to batch2, so we also create an additional factor to collect the batch information.</td>
      </tr>
      <tr>
        <td id="L519" class="blob-num js-line-number" data-line-number="519"></td>
        <td id="LC519" class="blob-code blob-code-inner js-file-line">
</td>
      </tr>
      <tr>
        <td id="L520" class="blob-num js-line-number" data-line-number="520"></td>
        <td id="LC520" class="blob-code blob-code-inner js-file-line">&lt;&lt;randomBatchEffect&gt;&gt;=</td>
      </tr>
      <tr>
        <td id="L521" class="blob-num js-line-number" data-line-number="521"></td>
        <td id="LC521" class="blob-code blob-code-inner js-file-line">set.seed(123)</td>
      </tr>
      <tr>
        <td id="L522" class="blob-num js-line-number" data-line-number="522"></td>
        <td id="LC522" class="blob-code blob-code-inner js-file-line">mycounts2 = mycounts</td>
      </tr>
      <tr>
        <td id="L523" class="blob-num js-line-number" data-line-number="523"></td>
        <td id="LC523" class="blob-code blob-code-inner js-file-line">mycounts2[,1:4] = mycounts2[,1:4] + runif(nrow(mycounts2)*4, 3, 5)</td>
      </tr>
      <tr>
        <td id="L524" class="blob-num js-line-number" data-line-number="524"></td>
        <td id="LC524" class="blob-code blob-code-inner js-file-line">myfactors = data.frame(myfactors, &quot;batch&quot; = c(rep(1,4), rep(2,6)))</td>
      </tr>
      <tr>
        <td id="L525" class="blob-num js-line-number" data-line-number="525"></td>
        <td id="LC525" class="blob-code blob-code-inner js-file-line">mydata2 = readData(mycounts2, factors = myfactors)</td>
      </tr>
      <tr>
        <td id="L526" class="blob-num js-line-number" data-line-number="526"></td>
        <td id="LC526" class="blob-code blob-code-inner js-file-line">@</td>
      </tr>
      <tr>
        <td id="L527" class="blob-num js-line-number" data-line-number="527"></td>
        <td id="LC527" class="blob-code blob-code-inner js-file-line">
</td>
      </tr>
      <tr>
        <td id="L528" class="blob-num js-line-number" data-line-number="528"></td>
        <td id="LC528" class="blob-code blob-code-inner js-file-line">Now we can run the following code to plot the samples scores for the two principal components of the PCA and color them by the factor ``Tissue&#39;&#39; (left hand plot) or by the factor ``batch&#39;&#39; (right hand plot): </td>
      </tr>
      <tr>
        <td id="L529" class="blob-num js-line-number" data-line-number="529"></td>
        <td id="LC529" class="blob-code blob-code-inner js-file-line">
</td>
      </tr>
      <tr>
        <td id="L530" class="blob-num js-line-number" data-line-number="530"></td>
        <td id="LC530" class="blob-code blob-code-inner js-file-line">&lt;&lt;fig_PCA,fig=TRUE&gt;&gt;=</td>
      </tr>
      <tr>
        <td id="L531" class="blob-num js-line-number" data-line-number="531"></td>
        <td id="LC531" class="blob-code blob-code-inner js-file-line">myPCA = dat(mydata2, type = &quot;PCA&quot;)</td>
      </tr>
      <tr>
        <td id="L532" class="blob-num js-line-number" data-line-number="532"></td>
        <td id="LC532" class="blob-code blob-code-inner js-file-line">par(mfrow = c(1,2))</td>
      </tr>
      <tr>
        <td id="L533" class="blob-num js-line-number" data-line-number="533"></td>
        <td id="LC533" class="blob-code blob-code-inner js-file-line">explo.plot(myPCA, factor = &quot;Tissue&quot;)</td>
      </tr>
      <tr>
        <td id="L534" class="blob-num js-line-number" data-line-number="534"></td>
        <td id="LC534" class="blob-code blob-code-inner js-file-line">explo.plot(myPCA, factor = &quot;batch&quot;)</td>
      </tr>
      <tr>
        <td id="L535" class="blob-num js-line-number" data-line-number="535"></td>
        <td id="LC535" class="blob-code blob-code-inner js-file-line">@</td>
      </tr>
      <tr>
        <td id="L536" class="blob-num js-line-number" data-line-number="536"></td>
        <td id="LC536" class="blob-code blob-code-inner js-file-line">
</td>
      </tr>
      <tr>
        <td id="L537" class="blob-num js-line-number" data-line-number="537"></td>
        <td id="LC537" class="blob-code blob-code-inner js-file-line">\begin{figure}[ht]</td>
      </tr>
      <tr>
        <td id="L538" class="blob-num js-line-number" data-line-number="538"></td>
        <td id="LC538" class="blob-code blob-code-inner js-file-line">\centering</td>
      </tr>
      <tr>
        <td id="L539" class="blob-num js-line-number" data-line-number="539"></td>
        <td id="LC539" class="blob-code blob-code-inner js-file-line">\includegraphics[width=\textwidth, height=0.5\textwidth]{NOISeq-fig_PCA}</td>
      </tr>
      <tr>
        <td id="L540" class="blob-num js-line-number" data-line-number="540"></td>
        <td id="LC540" class="blob-code blob-code-inner js-file-line">\caption{PCA plot colored by tissue (left) and by batch (right)}</td>
      </tr>
      <tr>
        <td id="L541" class="blob-num js-line-number" data-line-number="541"></td>
        <td id="LC541" class="blob-code blob-code-inner js-file-line">\label{fig_PCA}</td>
      </tr>
      <tr>
        <td id="L542" class="blob-num js-line-number" data-line-number="542"></td>
        <td id="LC542" class="blob-code blob-code-inner js-file-line">\end{figure}</td>
      </tr>
      <tr>
        <td id="L543" class="blob-num js-line-number" data-line-number="543"></td>
        <td id="LC543" class="blob-code blob-code-inner js-file-line">
</td>
      </tr>
      <tr>
        <td id="L544" class="blob-num js-line-number" data-line-number="544"></td>
        <td id="LC544" class="blob-code blob-code-inner js-file-line">We can appreciate in these plots that the two batches are quite separated so removing the batch effect should improve the clustering of the samples. More information on how to do that with \noiseq{} can be found in Section \ref{sec_batch}.</td>
      </tr>
      <tr>
        <td id="L545" class="blob-num js-line-number" data-line-number="545"></td>
        <td id="LC545" class="blob-code blob-code-inner js-file-line">
</td>
      </tr>
      <tr>
        <td id="L546" class="blob-num js-line-number" data-line-number="546"></td>
        <td id="LC546" class="blob-code blob-code-inner js-file-line">
</td>
      </tr>
      <tr>
        <td id="L547" class="blob-num js-line-number" data-line-number="547"></td>
        <td id="LC547" class="blob-code blob-code-inner js-file-line">
</td>
      </tr>
      <tr>
        <td id="L548" class="blob-num js-line-number" data-line-number="548"></td>
        <td id="LC548" class="blob-code blob-code-inner js-file-line">\subsection{Quality Control report}</td>
      </tr>
      <tr>
        <td id="L549" class="blob-num js-line-number" data-line-number="549"></td>
        <td id="LC549" class="blob-code blob-code-inner js-file-line">
</td>
      </tr>
      <tr>
        <td id="L550" class="blob-num js-line-number" data-line-number="550"></td>
        <td id="LC550" class="blob-code blob-code-inner js-file-line">The \code{QCreport} function allows the user to quickly generate a pdf report showing the exploratory plots described in this section </td>
      </tr>
      <tr>
        <td id="L551" class="blob-num js-line-number" data-line-number="551"></td>
        <td id="LC551" class="blob-code blob-code-inner js-file-line">to compare either two samples (if \texttt{factor=NULL}) or two experimental conditions (if \texttt{factor} is indicated). </td>
      </tr>
      <tr>
        <td id="L552" class="blob-num js-line-number" data-line-number="552"></td>
        <td id="LC552" class="blob-code blob-code-inner js-file-line">Depending on the biological information provided (biotypes, length or GC content), the number of plots included in the report may differ. </td>
      </tr>
      <tr>
        <td id="L553" class="blob-num js-line-number" data-line-number="553"></td>
        <td id="LC553" class="blob-code blob-code-inner js-file-line">
</td>
      </tr>
      <tr>
        <td id="L554" class="blob-num js-line-number" data-line-number="554"></td>
        <td id="LC554" class="blob-code blob-code-inner js-file-line">&lt;&lt;QCreportExample,results=hide&gt;&gt;=</td>
      </tr>
      <tr>
        <td id="L555" class="blob-num js-line-number" data-line-number="555"></td>
        <td id="LC555" class="blob-code blob-code-inner js-file-line">QCreport(mydata, samples = NULL, factor = &quot;Tissue&quot;, norm = FALSE)</td>
      </tr>
      <tr>
        <td id="L556" class="blob-num js-line-number" data-line-number="556"></td>
        <td id="LC556" class="blob-code blob-code-inner js-file-line">@ </td>
      </tr>
      <tr>
        <td id="L557" class="blob-num js-line-number" data-line-number="557"></td>
        <td id="LC557" class="blob-code blob-code-inner js-file-line">
</td>
      </tr>
      <tr>
        <td id="L558" class="blob-num js-line-number" data-line-number="558"></td>
        <td id="LC558" class="blob-code blob-code-inner js-file-line">This report can be generated before normalizing the data (\texttt{norm = FALSE}) or after normalization to check if unwanted effects were corrected (\texttt{norm = TRUE}).</td>
      </tr>
      <tr>
        <td id="L559" class="blob-num js-line-number" data-line-number="559"></td>
        <td id="LC559" class="blob-code blob-code-inner js-file-line">
</td>
      </tr>
      <tr>
        <td id="L560" class="blob-num js-line-number" data-line-number="560"></td>
        <td id="LC560" class="blob-code blob-code-inner js-file-line">Please note that the data are log-transformed when computing Principal Component Analysis (PCA).</td>
      </tr>
      <tr>
        <td id="L561" class="blob-num js-line-number" data-line-number="561"></td>
        <td id="LC561" class="blob-code blob-code-inner js-file-line">
</td>
      </tr>
      <tr>
        <td id="L562" class="blob-num js-line-number" data-line-number="562"></td>
        <td id="LC562" class="blob-code blob-code-inner js-file-line">
</td>
      </tr>
      <tr>
        <td id="L563" class="blob-num js-line-number" data-line-number="563"></td>
        <td id="LC563" class="blob-code blob-code-inner js-file-line">
</td>
      </tr>
      <tr>
        <td id="L564" class="blob-num js-line-number" data-line-number="564"></td>
        <td id="LC564" class="blob-code blob-code-inner js-file-line">\vspace{1cm}</td>
      </tr>
      <tr>
        <td id="L565" class="blob-num js-line-number" data-line-number="565"></td>
        <td id="LC565" class="blob-code blob-code-inner js-file-line">
</td>
      </tr>
      <tr>
        <td id="L566" class="blob-num js-line-number" data-line-number="566"></td>
        <td id="LC566" class="blob-code blob-code-inner js-file-line">
</td>
      </tr>
      <tr>
        <td id="L567" class="blob-num js-line-number" data-line-number="567"></td>
        <td id="LC567" class="blob-code blob-code-inner js-file-line">\section{Normalization, Low-count filtering \&amp; Batch effect correction}</td>
      </tr>
      <tr>
        <td id="L568" class="blob-num js-line-number" data-line-number="568"></td>
        <td id="LC568" class="blob-code blob-code-inner js-file-line">
</td>
      </tr>
      <tr>
        <td id="L569" class="blob-num js-line-number" data-line-number="569"></td>
        <td id="LC569" class="blob-code blob-code-inner js-file-line">The normalization step is very important in order to make the samples comparable and to remove possibles biases in the data. </td>
      </tr>
      <tr>
        <td id="L570" class="blob-num js-line-number" data-line-number="570"></td>
        <td id="LC570" class="blob-code blob-code-inner js-file-line">It might also be useful to filter out low expression data prior to differential expression analysis, since they are less reliable </td>
      </tr>
      <tr>
        <td id="L571" class="blob-num js-line-number" data-line-number="571"></td>
        <td id="LC571" class="blob-code blob-code-inner js-file-line">and may introduce noise in the analysis.</td>
      </tr>
      <tr>
        <td id="L572" class="blob-num js-line-number" data-line-number="572"></td>
        <td id="LC572" class="blob-code blob-code-inner js-file-line">
</td>
      </tr>
      <tr>
        <td id="L573" class="blob-num js-line-number" data-line-number="573"></td>
        <td id="LC573" class="blob-code blob-code-inner js-file-line">Next sections explain how to use \noiseq{} package to normalize and filter data before performing any statistical analysis.</td>
      </tr>
      <tr>
        <td id="L574" class="blob-num js-line-number" data-line-number="574"></td>
        <td id="LC574" class="blob-code blob-code-inner js-file-line">
</td>
      </tr>
      <tr>
        <td id="L575" class="blob-num js-line-number" data-line-number="575"></td>
        <td id="LC575" class="blob-code blob-code-inner js-file-line">
</td>
      </tr>
      <tr>
        <td id="L576" class="blob-num js-line-number" data-line-number="576"></td>
        <td id="LC576" class="blob-code blob-code-inner js-file-line">\subsection{Normalization} \label{sec_norm}</td>
      </tr>
      <tr>
        <td id="L577" class="blob-num js-line-number" data-line-number="577"></td>
        <td id="LC577" class="blob-code blob-code-inner js-file-line">
</td>
      </tr>
      <tr>
        <td id="L578" class="blob-num js-line-number" data-line-number="578"></td>
        <td id="LC578" class="blob-code blob-code-inner js-file-line">We strongly recommend to normalize the counts to correct, at least, sequencing depth bias. </td>
      </tr>
      <tr>
        <td id="L579" class="blob-num js-line-number" data-line-number="579"></td>
        <td id="LC579" class="blob-code blob-code-inner js-file-line">The normalization techniques implemented in \noiseq{} are RPKM \cite{Mortazavi2008},</td>
      </tr>
      <tr>
        <td id="L580" class="blob-num js-line-number" data-line-number="580"></td>
        <td id="LC580" class="blob-code blob-code-inner js-file-line">Upper Quartile \cite{Bullard2010} and TMM, which stands for Trimmed Mean of M values \cite{Robinson2010}, but the package accepts data normalized with any other method as well as data previously transformed to remove batch effects or to reduce noise. </td>
      </tr>
      <tr>
        <td id="L581" class="blob-num js-line-number" data-line-number="581"></td>
        <td id="LC581" class="blob-code blob-code-inner js-file-line">
</td>
      </tr>
      <tr>
        <td id="L582" class="blob-num js-line-number" data-line-number="582"></td>
        <td id="LC582" class="blob-code blob-code-inner js-file-line">The normalization functions (\code{rpkm}, \code{tmm} and \code{uqua}) can be applied to common R matrix and data frame objects. Please, find below some examples on how to apply them to data matrix extracted from \noiseq{} data objects:</td>
      </tr>
      <tr>
        <td id="L583" class="blob-num js-line-number" data-line-number="583"></td>
        <td id="LC583" class="blob-code blob-code-inner js-file-line">
</td>
      </tr>
      <tr>
        <td id="L584" class="blob-num js-line-number" data-line-number="584"></td>
        <td id="LC584" class="blob-code blob-code-inner js-file-line">&lt;&lt;normalization&gt;&gt;=</td>
      </tr>
      <tr>
        <td id="L585" class="blob-num js-line-number" data-line-number="585"></td>
        <td id="LC585" class="blob-code blob-code-inner js-file-line">myRPKM = rpkm(assayData(mydata)$exprs, long = mylength, k = 0, lc = 1)</td>
      </tr>
      <tr>
        <td id="L586" class="blob-num js-line-number" data-line-number="586"></td>
        <td id="LC586" class="blob-code blob-code-inner js-file-line">myUQUA = uqua(assayData(mydata)$exprs, long = mylength, lc = 0.5, k = 0)</td>
      </tr>
      <tr>
        <td id="L587" class="blob-num js-line-number" data-line-number="587"></td>
        <td id="LC587" class="blob-code blob-code-inner js-file-line">myTMM = tmm(assayData(mydata)$exprs, long = 1000, lc = 0)</td>
      </tr>
      <tr>
        <td id="L588" class="blob-num js-line-number" data-line-number="588"></td>
        <td id="LC588" class="blob-code blob-code-inner js-file-line">head(myRPKM[,1:4])</td>
      </tr>
      <tr>
        <td id="L589" class="blob-num js-line-number" data-line-number="589"></td>
        <td id="LC589" class="blob-code blob-code-inner js-file-line">@ </td>
      </tr>
      <tr>
        <td id="L590" class="blob-num js-line-number" data-line-number="590"></td>
        <td id="LC590" class="blob-code blob-code-inner js-file-line">
</td>
      </tr>
      <tr>
        <td id="L591" class="blob-num js-line-number" data-line-number="591"></td>
        <td id="LC591" class="blob-code blob-code-inner js-file-line">If the length of the features is provided to any of the normalization functions, the expression values are divided by </td>
      </tr>
      <tr>
        <td id="L592" class="blob-num js-line-number" data-line-number="592"></td>
        <td id="LC592" class="blob-code blob-code-inner js-file-line">$(length/1000)^{lc}$. Thus, although Upper Quartile and TMM methods themselves do not correct for the length of the features, </td>
      </tr>
      <tr>
        <td id="L593" class="blob-num js-line-number" data-line-number="593"></td>
        <td id="LC593" class="blob-code blob-code-inner js-file-line">\noiseq{} allows the users to combine these normalization procedures with an additional length correction whenever the length information is available. If $lc = 0$, no length correction is applied.</td>
      </tr>
      <tr>
        <td id="L594" class="blob-num js-line-number" data-line-number="594"></td>
        <td id="LC594" class="blob-code blob-code-inner js-file-line">To obtain RPKM values, $lc = 1$ in \code{rpkm} function must be indicated. If $long = 1000$ in \code{rpkm} function, CPM values (counts per million) are returned.   </td>
      </tr>
      <tr>
        <td id="L595" class="blob-num js-line-number" data-line-number="595"></td>
        <td id="LC595" class="blob-code blob-code-inner js-file-line">
</td>
      </tr>
      <tr>
        <td id="L596" class="blob-num js-line-number" data-line-number="596"></td>
        <td id="LC596" class="blob-code blob-code-inner js-file-line">The $k$ parameter is used to replace the zero values in the expression matrix with other non-zero value in order to avoid indetermination in some calculations such as fold-change. If $k=NULL$, each 0 is replaced with the midpoint between 0 and the next non-zero value in the matrix. </td>
      </tr>
      <tr>
        <td id="L597" class="blob-num js-line-number" data-line-number="597"></td>
        <td id="LC597" class="blob-code blob-code-inner js-file-line">
</td>
      </tr>
      <tr>
        <td id="L598" class="blob-num js-line-number" data-line-number="598"></td>
        <td id="LC598" class="blob-code blob-code-inner js-file-line">
</td>
      </tr>
      <tr>
        <td id="L599" class="blob-num js-line-number" data-line-number="599"></td>
        <td id="LC599" class="blob-code blob-code-inner js-file-line">
</td>
      </tr>
      <tr>
        <td id="L600" class="blob-num js-line-number" data-line-number="600"></td>
        <td id="LC600" class="blob-code blob-code-inner js-file-line">\subsection{Low-count filtering} \label{sec_filt}</td>
      </tr>
      <tr>
        <td id="L601" class="blob-num js-line-number" data-line-number="601"></td>
        <td id="LC601" class="blob-code blob-code-inner js-file-line">
</td>
      </tr>
      <tr>
        <td id="L602" class="blob-num js-line-number" data-line-number="602"></td>
        <td id="LC602" class="blob-code blob-code-inner js-file-line">Excluding features with low counts improves, in general, differential expression results, no matter the method being used,</td>
      </tr>
      <tr>
        <td id="L603" class="blob-num js-line-number" data-line-number="603"></td>
        <td id="LC603" class="blob-code blob-code-inner js-file-line">since noise in the data is reduced. However, the best procedure</td>
      </tr>
      <tr>
        <td id="L604" class="blob-num js-line-number" data-line-number="604"></td>
        <td id="LC604" class="blob-code blob-code-inner js-file-line">to filter these low count features has not been yet decided nor</td>
      </tr>
      <tr>
        <td id="L605" class="blob-num js-line-number" data-line-number="605"></td>
        <td id="LC605" class="blob-code blob-code-inner js-file-line">implemented in the differential expression packages. \noiseq{} includes three methods to filter out features with low</td>
      </tr>
      <tr>
        <td id="L606" class="blob-num js-line-number" data-line-number="606"></td>
        <td id="LC606" class="blob-code blob-code-inner js-file-line">counts:</td>
      </tr>
      <tr>
        <td id="L607" class="blob-num js-line-number" data-line-number="607"></td>
        <td id="LC607" class="blob-code blob-code-inner js-file-line">\begin{itemize}</td>
      </tr>
      <tr>
        <td id="L608" class="blob-num js-line-number" data-line-number="608"></td>
        <td id="LC608" class="blob-code blob-code-inner js-file-line">\item \textbf{CPM} (method 1): The user chooses a value for the parameter counts per million (CPM) in a sample under which a feature is</td>
      </tr>
      <tr>
        <td id="L609" class="blob-num js-line-number" data-line-number="609"></td>
        <td id="LC609" class="blob-code blob-code-inner js-file-line">considered to have low counts. The cutoff for a condition with $s$ samples is</td>
      </tr>
      <tr>
        <td id="L610" class="blob-num js-line-number" data-line-number="610"></td>
        <td id="LC610" class="blob-code blob-code-inner js-file-line">$CPM \times s$. Features with sum of expression values below the condition cutoff in all</td>
      </tr>
      <tr>
        <td id="L611" class="blob-num js-line-number" data-line-number="611"></td>
        <td id="LC611" class="blob-code blob-code-inner js-file-line">conditions are removed. Also a cutoff for the coefficient of variation (in percentage) per condition</td>
      </tr>
      <tr>
        <td id="L612" class="blob-num js-line-number" data-line-number="612"></td>
        <td id="LC612" class="blob-code blob-code-inner js-file-line">may be established to eliminate features with inconsistent expression values.</td>
      </tr>
      <tr>
        <td id="L613" class="blob-num js-line-number" data-line-number="613"></td>
        <td id="LC613" class="blob-code blob-code-inner js-file-line">\item \textbf{Wilcoxon test} (method 2): For each feature and condition,  $H_0: m=0$ is tested versus $H_1: m&gt;0$, where $m$ is the median of counts per condition. </td>
      </tr>
      <tr>
        <td id="L614" class="blob-num js-line-number" data-line-number="614"></td>
        <td id="LC614" class="blob-code blob-code-inner js-file-line">Features with p-value $&gt; 0.05$ in all conditions are filtered out. P-values can be corrected for multiple testing using the \texttt{p.adj} option. This method is only recommended when the number of replicates per condition is at least 5.</td>
      </tr>
      <tr>
        <td id="L615" class="blob-num js-line-number" data-line-number="615"></td>
        <td id="LC615" class="blob-code blob-code-inner js-file-line">\item \textbf{Proportion test} (method 3): Similar procedure to the Wilcoxon test but testing $H_0: p=p_0$ versus $H_1: p&gt;p_0$, where $p$ is the </td>
      </tr>
      <tr>
        <td id="L616" class="blob-num js-line-number" data-line-number="616"></td>
        <td id="LC616" class="blob-code blob-code-inner js-file-line">feature relative expression and $p_0 = CPM/10^6$. Features with p-value $&gt; 0.05$ in all conditions are filtered out. P-values can be corrected for multiple testing using the \texttt{p.adj} option.</td>
      </tr>
      <tr>
        <td id="L617" class="blob-num js-line-number" data-line-number="617"></td>
        <td id="LC617" class="blob-code blob-code-inner js-file-line">\end{itemize}</td>
      </tr>
      <tr>
        <td id="L618" class="blob-num js-line-number" data-line-number="618"></td>
        <td id="LC618" class="blob-code blob-code-inner js-file-line">
</td>
      </tr>
      <tr>
        <td id="L619" class="blob-num js-line-number" data-line-number="619"></td>
        <td id="LC619" class="blob-code blob-code-inner js-file-line">This is an usage example of function \code{filtered.data} directly on count data with CPM method (method 1): </td>
      </tr>
      <tr>
        <td id="L620" class="blob-num js-line-number" data-line-number="620"></td>
        <td id="LC620" class="blob-code blob-code-inner js-file-line">&lt;&lt;filtering&gt;&gt;=</td>
      </tr>
      <tr>
        <td id="L621" class="blob-num js-line-number" data-line-number="621"></td>
        <td id="LC621" class="blob-code blob-code-inner js-file-line">myfilt = filtered.data(mycounts, factor = myfactors$Tissue, norm = FALSE, depth = NULL, method = 1, cv.cutoff = 100, cpm = 1, p.adj = &quot;fdr&quot;)</td>
      </tr>
      <tr>
        <td id="L622" class="blob-num js-line-number" data-line-number="622"></td>
        <td id="LC622" class="blob-code blob-code-inner js-file-line">@ </td>
      </tr>
      <tr>
        <td id="L623" class="blob-num js-line-number" data-line-number="623"></td>
        <td id="LC623" class="blob-code blob-code-inner js-file-line">
</td>
      </tr>
      <tr>
        <td id="L624" class="blob-num js-line-number" data-line-number="624"></td>
        <td id="LC624" class="blob-code blob-code-inner js-file-line">The ``Sensitivity plot&#39;&#39; described in previous section can help to take decisions on the CPM threshold to use in methods 1 and 3.</td>
      </tr>
      <tr>
        <td id="L625" class="blob-num js-line-number" data-line-number="625"></td>
        <td id="LC625" class="blob-code blob-code-inner js-file-line">
</td>
      </tr>
      <tr>
        <td id="L626" class="blob-num js-line-number" data-line-number="626"></td>
        <td id="LC626" class="blob-code blob-code-inner js-file-line">
</td>
      </tr>
      <tr>
        <td id="L627" class="blob-num js-line-number" data-line-number="627"></td>
        <td id="LC627" class="blob-code blob-code-inner js-file-line">
</td>
      </tr>
      <tr>
        <td id="L628" class="blob-num js-line-number" data-line-number="628"></td>
        <td id="LC628" class="blob-code blob-code-inner js-file-line">
</td>
      </tr>
      <tr>
        <td id="L629" class="blob-num js-line-number" data-line-number="629"></td>
        <td id="LC629" class="blob-code blob-code-inner js-file-line">\subsection{Batch effect correction} \label{sec_batch}</td>
      </tr>
      <tr>
        <td id="L630" class="blob-num js-line-number" data-line-number="630"></td>
        <td id="LC630" class="blob-code blob-code-inner js-file-line">
</td>
      </tr>
      <tr>
        <td id="L631" class="blob-num js-line-number" data-line-number="631"></td>
        <td id="LC631" class="blob-code blob-code-inner js-file-line">When a batch effect is detected in the data or the samples are not properly clustered due to an unknown source of technical noise, it is usually appropriate to remove this batch effect or noise before proceeding with the differential expression analysis (or any other type of analysis).</td>
      </tr>
      <tr>
        <td id="L632" class="blob-num js-line-number" data-line-number="632"></td>
        <td id="LC632" class="blob-code blob-code-inner js-file-line">
</td>
      </tr>
      <tr>
        <td id="L633" class="blob-num js-line-number" data-line-number="633"></td>
        <td id="LC633" class="blob-code blob-code-inner js-file-line">\texttt{ARSyNseq} (ASCA Removal of Systematic Noise for sequencing data) is an R function implemented in \noiseq{} package that is designed for filtering the noise associated to identified or unidentified batch effects. The ARSyN method \cite{nueda2012} combines analysis</td>
      </tr>
      <tr>
        <td id="L634" class="blob-num js-line-number" data-line-number="634"></td>
        <td id="LC634" class="blob-code blob-code-inner js-file-line">of variance (ANOVA) modeling and multivariate analysis of estimated effects (PCA) to identify the structured variation of either the effect of the batch (if the batch information is provided) or the ANOVA errors (if the batch information is unknown). Thus, ARSyNseq returns a filtered data set that is rich in the information of interest and includes only the random noise required for inferential analysis.</td>
      </tr>
      <tr>
        <td id="L635" class="blob-num js-line-number" data-line-number="635"></td>
        <td id="LC635" class="blob-code blob-code-inner js-file-line">
</td>
      </tr>
      <tr>
        <td id="L636" class="blob-num js-line-number" data-line-number="636"></td>
        <td id="LC636" class="blob-code blob-code-inner js-file-line">The main arguments of the \texttt{ARSyNseq} function are:</td>
      </tr>
      <tr>
        <td id="L637" class="blob-num js-line-number" data-line-number="637"></td>
        <td id="LC637" class="blob-code blob-code-inner js-file-line">\begin{itemize}</td>
      </tr>
      <tr>
        <td id="L638" class="blob-num js-line-number" data-line-number="638"></td>
        <td id="LC638" class="blob-code blob-code-inner js-file-line"> \item \texttt{data}: A Biobase&#39;s eSet object created with the \texttt{readData} function.</td>
      </tr>
      <tr>
        <td id="L639" class="blob-num js-line-number" data-line-number="639"></td>
        <td id="LC639" class="blob-code blob-code-inner js-file-line"> \item \texttt{factor}: Name of the factor (as it was given to the \texttt{readData} function) to be used in the ARSyN model (e.g. the factor containing the batch information). When it is NULL, all the factors are considered.</td>
      </tr>
      <tr>
        <td id="L640" class="blob-num js-line-number" data-line-number="640"></td>
        <td id="LC640" class="blob-code blob-code-inner js-file-line">  \item \texttt{batch}: TRUE to indicate that the \texttt{factor} argument indicates the batch information. In this case, the \texttt{factor} argument must be used to specify the names of the onlu factor containing the information of the batch.</td>
      </tr>
      <tr>
        <td id="L641" class="blob-num js-line-number" data-line-number="641"></td>
        <td id="LC641" class="blob-code blob-code-inner js-file-line">  \item \texttt{norm}: Type of normalization to be used. One of ``rpkm&#39;&#39; (default), ``uqua&#39;&#39;, ``tmm&#39;&#39; or ``n&#39;&#39; (if data are already normalized). If length was provided through the \texttt{readData} function, it will be considered for the normalization (except for ``n&#39;&#39;). Please note that if a normalization method if used, the arguments \texttt{lc} and  \texttt{k} are set to 1 and 0 respectively.</td>
      </tr>
      <tr>
        <td id="L642" class="blob-num js-line-number" data-line-number="642"></td>
        <td id="LC642" class="blob-code blob-code-inner js-file-line">  \item \texttt{logtransf}: If FALSE, a log-transformation will be applied on the data before computing ARSyN model </td>
      </tr>
      <tr>
        <td id="L643" class="blob-num js-line-number" data-line-number="643"></td>
        <td id="LC643" class="blob-code blob-code-inner js-file-line">  to improve the results of PCA on count data.</td>
      </tr>
      <tr>
        <td id="L644" class="blob-num js-line-number" data-line-number="644"></td>
        <td id="LC644" class="blob-code blob-code-inner js-file-line">\end{itemize}</td>
      </tr>
      <tr>
        <td id="L645" class="blob-num js-line-number" data-line-number="645"></td>
        <td id="LC645" class="blob-code blob-code-inner js-file-line">
</td>
      </tr>
      <tr>
        <td id="L646" class="blob-num js-line-number" data-line-number="646"></td>
        <td id="LC646" class="blob-code blob-code-inner js-file-line">Therefore, we can differentiate two types of analysis:</td>
      </tr>
      <tr>
        <td id="L647" class="blob-num js-line-number" data-line-number="647"></td>
        <td id="LC647" class="blob-code blob-code-inner js-file-line">\begin{enumerate}</td>
      </tr>
      <tr>
        <td id="L648" class="blob-num js-line-number" data-line-number="648"></td>
        <td id="LC648" class="blob-code blob-code-inner js-file-line">\item When batch is identified with one of the factors described in the argument \texttt{factor} of the \texttt{data} object, \texttt{ARSyNseq}</td>
      </tr>
      <tr>
        <td id="L649" class="blob-num js-line-number" data-line-number="649"></td>
        <td id="LC649" class="blob-code blob-code-inner js-file-line">  estimates this effect and removes it by estimating the main PCs of the ANOVA effects associated. In such case \texttt{factor} argument </td>
      </tr>
      <tr>
        <td id="L650" class="blob-num js-line-number" data-line-number="650"></td>
        <td id="LC650" class="blob-code blob-code-inner js-file-line">    will be the name of the batch and \texttt{batch=TRUE}.  </td>
      </tr>
      <tr>
        <td id="L651" class="blob-num js-line-number" data-line-number="651"></td>
        <td id="LC651" class="blob-code blob-code-inner js-file-line">\item When batch is not identified, the model estimates the effects associated to each factor of interest and analyses if there exists </td>
      </tr>
      <tr>
        <td id="L652" class="blob-num js-line-number" data-line-number="652"></td>
        <td id="LC652" class="blob-code blob-code-inner js-file-line">  systematic noise in the residuals. If there is batch effect, it will be identified and removed by estimating the main PCs of these residuals.</td>
      </tr>
      <tr>
        <td id="L653" class="blob-num js-line-number" data-line-number="653"></td>
        <td id="LC653" class="blob-code blob-code-inner js-file-line">  In such case \texttt{factor} argument can have several factors and \texttt{batch=FALSE}.  </td>
      </tr>
      <tr>
        <td id="L654" class="blob-num js-line-number" data-line-number="654"></td>
        <td id="LC654" class="blob-code blob-code-inner js-file-line">\end{enumerate}</td>
      </tr>
      <tr>
        <td id="L655" class="blob-num js-line-number" data-line-number="655"></td>
        <td id="LC655" class="blob-code blob-code-inner js-file-line">
</td>
      </tr>
      <tr>
        <td id="L656" class="blob-num js-line-number" data-line-number="656"></td>
        <td id="LC656" class="blob-code blob-code-inner js-file-line">
</td>
      </tr>
      <tr>
        <td id="L657" class="blob-num js-line-number" data-line-number="657"></td>
        <td id="LC657" class="blob-code blob-code-inner js-file-line">We will use the toy example generated in Section \ref{sec_PCA} to illustrate how \texttt{ARSyNseq} works. This is the code to use </td>
      </tr>
      <tr>
        <td id="L658" class="blob-num js-line-number" data-line-number="658"></td>
        <td id="LC658" class="blob-code blob-code-inner js-file-line">\texttt{ARSyNseq} batch effect correction when the user knows the batch in which the samples were processed, and to represent a PCA with the filtered data in order to see how the batch effect was corrected (Figure \ref{fig_knownBatch}:</td>
      </tr>
      <tr>
        <td id="L659" class="blob-num js-line-number" data-line-number="659"></td>
        <td id="LC659" class="blob-code blob-code-inner js-file-line">
</td>
      </tr>
      <tr>
        <td id="L660" class="blob-num js-line-number" data-line-number="660"></td>
        <td id="LC660" class="blob-code blob-code-inner js-file-line">&lt;&lt;fig_knownBatch,fig=TRUE&gt;&gt;=</td>
      </tr>
      <tr>
        <td id="L661" class="blob-num js-line-number" data-line-number="661"></td>
        <td id="LC661" class="blob-code blob-code-inner js-file-line">mydata2corr1 = ARSyNseq(mydata2, factor = &quot;batch&quot;, batch = TRUE, norm = &quot;rpkm&quot;,  logtransf = FALSE)</td>
      </tr>
      <tr>
        <td id="L662" class="blob-num js-line-number" data-line-number="662"></td>
        <td id="LC662" class="blob-code blob-code-inner js-file-line">myPCA = dat(mydata2corr1, type = &quot;PCA&quot;)</td>
      </tr>
      <tr>
        <td id="L663" class="blob-num js-line-number" data-line-number="663"></td>
        <td id="LC663" class="blob-code blob-code-inner js-file-line">par(mfrow = c(1,2))</td>
      </tr>
      <tr>
        <td id="L664" class="blob-num js-line-number" data-line-number="664"></td>
        <td id="LC664" class="blob-code blob-code-inner js-file-line">explo.plot(myPCA, factor = &quot;Tissue&quot;)</td>
      </tr>
      <tr>
        <td id="L665" class="blob-num js-line-number" data-line-number="665"></td>
        <td id="LC665" class="blob-code blob-code-inner js-file-line">explo.plot(myPCA, factor = &quot;batch&quot;)</td>
      </tr>
      <tr>
        <td id="L666" class="blob-num js-line-number" data-line-number="666"></td>
        <td id="LC666" class="blob-code blob-code-inner js-file-line">@</td>
      </tr>
      <tr>
        <td id="L667" class="blob-num js-line-number" data-line-number="667"></td>
        <td id="LC667" class="blob-code blob-code-inner js-file-line">
</td>
      </tr>
      <tr>
        <td id="L668" class="blob-num js-line-number" data-line-number="668"></td>
        <td id="LC668" class="blob-code blob-code-inner js-file-line">\begin{figure}[ht!]</td>
      </tr>
      <tr>
        <td id="L669" class="blob-num js-line-number" data-line-number="669"></td>
        <td id="LC669" class="blob-code blob-code-inner js-file-line">\centering</td>
      </tr>
      <tr>
        <td id="L670" class="blob-num js-line-number" data-line-number="670"></td>
        <td id="LC670" class="blob-code blob-code-inner js-file-line">\includegraphics[width=\textwidth, height=0.5\textwidth]{NOISeq-fig_knownBatch}</td>
      </tr>
      <tr>
        <td id="L671" class="blob-num js-line-number" data-line-number="671"></td>
        <td id="LC671" class="blob-code blob-code-inner js-file-line">\caption{PCA plot after correcting a known batch effect with \texttt{ARSyNseq}. The samples are colored by tissue (left) and by batch (right)}</td>
      </tr>
      <tr>
        <td id="L672" class="blob-num js-line-number" data-line-number="672"></td>
        <td id="LC672" class="blob-code blob-code-inner js-file-line">\label{fig_knownBatch}</td>
      </tr>
      <tr>
        <td id="L673" class="blob-num js-line-number" data-line-number="673"></td>
        <td id="LC673" class="blob-code blob-code-inner js-file-line">\end{figure}</td>
      </tr>
      <tr>
        <td id="L674" class="blob-num js-line-number" data-line-number="674"></td>
        <td id="LC674" class="blob-code blob-code-inner js-file-line">
</td>
      </tr>
      <tr>
        <td id="L675" class="blob-num js-line-number" data-line-number="675"></td>
        <td id="LC675" class="blob-code blob-code-inner js-file-line">Let us suppose now that we do not know the batch information. However, we can appreciate in the PCA plot of Section \ref{sec_PCA} that there is an unknown source of noise that prevents the samples from clustering well. In this case, we can run the following code to reduce </td>
      </tr>
      <tr>
        <td id="L676" class="blob-num js-line-number" data-line-number="676"></td>
        <td id="LC676" class="blob-code blob-code-inner js-file-line">the unidentified batch effect and to draw the PCA plots on the filtered data:</td>
      </tr>
      <tr>
        <td id="L677" class="blob-num js-line-number" data-line-number="677"></td>
        <td id="LC677" class="blob-code blob-code-inner js-file-line">
</td>
      </tr>
      <tr>
        <td id="L678" class="blob-num js-line-number" data-line-number="678"></td>
        <td id="LC678" class="blob-code blob-code-inner js-file-line">&lt;&lt;fig_unknownBatch,fig=TRUE&gt;&gt;=</td>
      </tr>
      <tr>
        <td id="L679" class="blob-num js-line-number" data-line-number="679"></td>
        <td id="LC679" class="blob-code blob-code-inner js-file-line">mydata2corr2 = ARSyNseq(mydata2, factor = &quot;Tissue&quot;, batch = FALSE, norm = &quot;rpkm&quot;,  logtransf = FALSE)</td>
      </tr>
      <tr>
        <td id="L680" class="blob-num js-line-number" data-line-number="680"></td>
        <td id="LC680" class="blob-code blob-code-inner js-file-line">myPCA = dat(mydata2corr2, type = &quot;PCA&quot;)</td>
      </tr>
      <tr>
        <td id="L681" class="blob-num js-line-number" data-line-number="681"></td>
        <td id="LC681" class="blob-code blob-code-inner js-file-line">par(mfrow = c(1,2))</td>
      </tr>
      <tr>
        <td id="L682" class="blob-num js-line-number" data-line-number="682"></td>
        <td id="LC682" class="blob-code blob-code-inner js-file-line">explo.plot(myPCA, factor = &quot;Tissue&quot;)</td>
      </tr>
      <tr>
        <td id="L683" class="blob-num js-line-number" data-line-number="683"></td>
        <td id="LC683" class="blob-code blob-code-inner js-file-line">explo.plot(myPCA, factor = &quot;batch&quot;)</td>
      </tr>
      <tr>
        <td id="L684" class="blob-num js-line-number" data-line-number="684"></td>
        <td id="LC684" class="blob-code blob-code-inner js-file-line">@</td>
      </tr>
      <tr>
        <td id="L685" class="blob-num js-line-number" data-line-number="685"></td>
        <td id="LC685" class="blob-code blob-code-inner js-file-line">
</td>
      </tr>
      <tr>
        <td id="L686" class="blob-num js-line-number" data-line-number="686"></td>
        <td id="LC686" class="blob-code blob-code-inner js-file-line">\begin{figure}[ht!]</td>
      </tr>
      <tr>
        <td id="L687" class="blob-num js-line-number" data-line-number="687"></td>
        <td id="LC687" class="blob-code blob-code-inner js-file-line">\centering</td>
      </tr>
      <tr>
        <td id="L688" class="blob-num js-line-number" data-line-number="688"></td>
        <td id="LC688" class="blob-code blob-code-inner js-file-line">\includegraphics[width=\textwidth, height=0.5\textwidth]{NOISeq-fig_unknownBatch}</td>
      </tr>
      <tr>
        <td id="L689" class="blob-num js-line-number" data-line-number="689"></td>
        <td id="LC689" class="blob-code blob-code-inner js-file-line">\caption{PCA plot after correcting an unidentified batch effect with \texttt{ARSyNseq}. The samples are colored by tissue (left) and by batch (right)}</td>
      </tr>
      <tr>
        <td id="L690" class="blob-num js-line-number" data-line-number="690"></td>
        <td id="LC690" class="blob-code blob-code-inner js-file-line">\label{fig_unknownBatch}</td>
      </tr>
      <tr>
        <td id="L691" class="blob-num js-line-number" data-line-number="691"></td>
        <td id="LC691" class="blob-code blob-code-inner js-file-line">\end{figure}</td>
      </tr>
      <tr>
        <td id="L692" class="blob-num js-line-number" data-line-number="692"></td>
        <td id="LC692" class="blob-code blob-code-inner js-file-line">
</td>
      </tr>
      <tr>
        <td id="L693" class="blob-num js-line-number" data-line-number="693"></td>
        <td id="LC693" class="blob-code blob-code-inner js-file-line">
</td>
      </tr>
      <tr>
        <td id="L694" class="blob-num js-line-number" data-line-number="694"></td>
        <td id="LC694" class="blob-code blob-code-inner js-file-line">\vspace{1cm}</td>
      </tr>
      <tr>
        <td id="L695" class="blob-num js-line-number" data-line-number="695"></td>
        <td id="LC695" class="blob-code blob-code-inner js-file-line">
</td>
      </tr>
      <tr>
        <td id="L696" class="blob-num js-line-number" data-line-number="696"></td>
        <td id="LC696" class="blob-code blob-code-inner js-file-line">
</td>
      </tr>
      <tr>
        <td id="L697" class="blob-num js-line-number" data-line-number="697"></td>
        <td id="LC697" class="blob-code blob-code-inner js-file-line">
</td>
      </tr>
      <tr>
        <td id="L698" class="blob-num js-line-number" data-line-number="698"></td>
        <td id="LC698" class="blob-code blob-code-inner js-file-line">
</td>
      </tr>
      <tr>
        <td id="L699" class="blob-num js-line-number" data-line-number="699"></td>
        <td id="LC699" class="blob-code blob-code-inner js-file-line">\section{Differential expression}</td>
      </tr>
      <tr>
        <td id="L700" class="blob-num js-line-number" data-line-number="700"></td>
        <td id="LC700" class="blob-code blob-code-inner js-file-line">
</td>
      </tr>
      <tr>
        <td id="L701" class="blob-num js-line-number" data-line-number="701"></td>
        <td id="LC701" class="blob-code blob-code-inner js-file-line">The \noiseq{} package computes differential expression between two experimental conditions given the expression level of the considered features. </td>
      </tr>
      <tr>
        <td id="L702" class="blob-num js-line-number" data-line-number="702"></td>
        <td id="LC702" class="blob-code blob-code-inner js-file-line">The package includes two non-parametric approaches for differential expression analysis: \noiseq{} \cite{tarazona2011} for technical replicates or no replication at all, and \noiseqbio{} \cite{tarazona2015}, which is optimized for the use of biological replicates. </td>
      </tr>
      <tr>
        <td id="L703" class="blob-num js-line-number" data-line-number="703"></td>
        <td id="LC703" class="blob-code blob-code-inner js-file-line">Both methods take read counts from RNA-seq as the expression values, in addition to previously normalized data and</td>
      </tr>
      <tr>
        <td id="L704" class="blob-num js-line-number" data-line-number="704"></td>
        <td id="LC704" class="blob-code blob-code-inner js-file-line">read counts from other NGS technologies.</td>
      </tr>
      <tr>
        <td id="L705" class="blob-num js-line-number" data-line-number="705"></td>
        <td id="LC705" class="blob-code blob-code-inner js-file-line">
</td>
      </tr>
      <tr>
        <td id="L706" class="blob-num js-line-number" data-line-number="706"></td>
        <td id="LC706" class="blob-code blob-code-inner js-file-line">In the previous section, we described how to use normalization and filtering functions prior to perform differential expression analysis.</td>
      </tr>
      <tr>
        <td id="L707" class="blob-num js-line-number" data-line-number="707"></td>
        <td id="LC707" class="blob-code blob-code-inner js-file-line">However, when using \noiseq{} or \noiseqbio{} to compute differential expression, it is not necessary to normalize or filter low counts before applying these methods because they include these options. Thus, normalization can be done automatically </td>
      </tr>
      <tr>
        <td id="L708" class="blob-num js-line-number" data-line-number="708"></td>
        <td id="LC708" class="blob-code blob-code-inner js-file-line">by choosing the corresponding value for the parameter \texttt{norm}. Furthermore, they also accept expression values normalized with other packages or procedures. If the data have been previously normalized, \texttt{norm} parameter must be set to ``n&#39;&#39;. Regarding the low-count filtering, it is not necessary to filter in \noiseq{} method. In contrast, it is recommended to do it in \noiseqbio{}, which by default filters out low-count features with CPM method (\texttt{filter=1}).</td>
      </tr>
      <tr>
        <td id="L709" class="blob-num js-line-number" data-line-number="709"></td>
        <td id="LC709" class="blob-code blob-code-inner js-file-line">
</td>
      </tr>
      <tr>
        <td id="L710" class="blob-num js-line-number" data-line-number="710"></td>
        <td id="LC710" class="blob-code blob-code-inner js-file-line">The following sections describe in more detail the \noiseq{} and \noiseqbio{} methods.</td>
      </tr>
      <tr>
        <td id="L711" class="blob-num js-line-number" data-line-number="711"></td>
        <td id="LC711" class="blob-code blob-code-inner js-file-line">
</td>
      </tr>
      <tr>
        <td id="L712" class="blob-num js-line-number" data-line-number="712"></td>
        <td id="LC712" class="blob-code blob-code-inner js-file-line">
</td>
      </tr>
      <tr>
        <td id="L713" class="blob-num js-line-number" data-line-number="713"></td>
        <td id="LC713" class="blob-code blob-code-inner js-file-line">
</td>
      </tr>
      <tr>
        <td id="L714" class="blob-num js-line-number" data-line-number="714"></td>
        <td id="LC714" class="blob-code blob-code-inner js-file-line">\subsection{NOISeq} \label{sec_param1}</td>
      </tr>
      <tr>
        <td id="L715" class="blob-num js-line-number" data-line-number="715"></td>
        <td id="LC715" class="blob-code blob-code-inner js-file-line">
</td>
      </tr>
      <tr>
        <td id="L716" class="blob-num js-line-number" data-line-number="716"></td>
        <td id="LC716" class="blob-code blob-code-inner js-file-line">\noiseq{} method was designed to compute differential expression on data with technical replicates (NOISeq-real) or no replicates at all (NOISeq-sim). </td>
      </tr>
      <tr>
        <td id="L717" class="blob-num js-line-number" data-line-number="717"></td>
        <td id="LC717" class="blob-code blob-code-inner js-file-line">If there are technical replicates available, it summarizes them by summing up them. It is also possible to apply this method on biological replicates, that are averaged instead of summed. However, for biological replicates we strongly recommend \noiseqbio{}. </td>
      </tr>
      <tr>
        <td id="L718" class="blob-num js-line-number" data-line-number="718"></td>
        <td id="LC718" class="blob-code blob-code-inner js-file-line">\noiseq{} computes the following differential expression statistics for each feature: </td>
      </tr>
      <tr>
        <td id="L719" class="blob-num js-line-number" data-line-number="719"></td>
        <td id="LC719" class="blob-code blob-code-inner js-file-line">$M$ (which is the $log_2$-ratio of the two conditions) and $D$ (the value of the difference between conditions). </td>
      </tr>
      <tr>
        <td id="L720" class="blob-num js-line-number" data-line-number="720"></td>
        <td id="LC720" class="blob-code blob-code-inner js-file-line">Expression levels equal to 0 are replaced with the given constant $k&gt;0$, in order to avoid </td>
      </tr>
      <tr>
        <td id="L721" class="blob-num js-line-number" data-line-number="721"></td>
        <td id="LC721" class="blob-code blob-code-inner js-file-line">infinite or undetermined $M$-values. If $k=NULL$, the 0 is replaced by the midpoint between 0 and the next non-zero value </td>
      </tr>
      <tr>
        <td id="L722" class="blob-num js-line-number" data-line-number="722"></td>
        <td id="LC722" class="blob-code blob-code-inner js-file-line">in the expression matrix.</td>
      </tr>
      <tr>
        <td id="L723" class="blob-num js-line-number" data-line-number="723"></td>
        <td id="LC723" class="blob-code blob-code-inner js-file-line">A feature is considered to be differentially expressed if its corresponding $M$ and $D$ values are likely to be higher than in noise. Noise distribution is obtained by comparing all pairs of replicates within the same condition. </td>
      </tr>
      <tr>
        <td id="L724" class="blob-num js-line-number" data-line-number="724"></td>
        <td id="LC724" class="blob-code blob-code-inner js-file-line">The corresponding $M$ and $D$ values are pooled together to generate the distribution. </td>
      </tr>
      <tr>
        <td id="L725" class="blob-num js-line-number" data-line-number="725"></td>
        <td id="LC725" class="blob-code blob-code-inner js-file-line">Changes in expression between conditions with the same magnitude than changes in expression between replicates within the same condition </td>
      </tr>
      <tr>
        <td id="L726" class="blob-num js-line-number" data-line-number="726"></td>
        <td id="LC726" class="blob-code blob-code-inner js-file-line">should not be considered as differential expression. Thus, by comparing the $(M,D)$ values of a given feature against the noise</td>
      </tr>
      <tr>
        <td id="L727" class="blob-num js-line-number" data-line-number="727"></td>
        <td id="LC727" class="blob-code blob-code-inner js-file-line">distribution, \noiseq{} obtains the ``probability of differential expression&#39;&#39; for this feature. </td>
      </tr>
      <tr>
        <td id="L728" class="blob-num js-line-number" data-line-number="728"></td>
        <td id="LC728" class="blob-code blob-code-inner js-file-line">If the odds Pr(differential expression)/Pr(non-differential expression) are higher</td>
      </tr>
      <tr>
        <td id="L729" class="blob-num js-line-number" data-line-number="729"></td>
        <td id="LC729" class="blob-code blob-code-inner js-file-line">than a given threshold, the feature is considered to be differentially expressed between conditions. </td>
      </tr>
      <tr>
        <td id="L730" class="blob-num js-line-number" data-line-number="730"></td>
        <td id="LC730" class="blob-code blob-code-inner js-file-line">For instance, an odds value of 4:1 is equivalent to $q$ = Pr(differential expression) = 0.8 and it means that the feature is 4 times more likely to be differentially expressed than non-differentially expressed. The \noiseq{} algorithm compares replicates</td>
      </tr>
      <tr>
        <td id="L731" class="blob-num js-line-number" data-line-number="731"></td>
        <td id="LC731" class="blob-code blob-code-inner js-file-line">within the same condition to estimate noise distribution (NOISeq-real). When no replicates are available, NOISeq-sim simulates technical replicates in order to estimate the differential expression probability. Please remember that to obtain a really reliable statistical results, you need biological replicates. NOISeq-sim simulates technical replicates from a multinomial distribution, </td>
      </tr>
      <tr>
        <td id="L732" class="blob-num js-line-number" data-line-number="732"></td>
        <td id="LC732" class="blob-code blob-code-inner js-file-line">so be careful with the interpretation of the results when having no replicates, since they are only an</td>
      </tr>
      <tr>
        <td id="L733" class="blob-num js-line-number" data-line-number="733"></td>
        <td id="LC733" class="blob-code blob-code-inner js-file-line">approximation and are only showing which genes are presenting a higher change between conditions in your particular samples.</td>
      </tr>
      <tr>
        <td id="L734" class="blob-num js-line-number" data-line-number="734"></td>
        <td id="LC734" class="blob-code blob-code-inner js-file-line">
</td>
      </tr>
      <tr>
        <td id="L735" class="blob-num js-line-number" data-line-number="735"></td>
        <td id="LC735" class="blob-code blob-code-inner js-file-line">Table \ref{table:summary} summarizes all the input options and includes some recommendations for the values of the parameters when using \noiseq{}: </td>
      </tr>
      <tr>
        <td id="L736" class="blob-num js-line-number" data-line-number="736"></td>
        <td id="LC736" class="blob-code blob-code-inner js-file-line">
</td>
      </tr>
      <tr>
        <td id="L737" class="blob-num js-line-number" data-line-number="737"></td>
        <td id="LC737" class="blob-code blob-code-inner js-file-line">\begin{table}[ht]</td>
      </tr>
      <tr>
        <td id="L738" class="blob-num js-line-number" data-line-number="738"></td>
        <td id="LC738" class="blob-code blob-code-inner js-file-line">\caption{Possibilities for the values of the parameters} % title name of the table</td>
      </tr>
      <tr>
        <td id="L739" class="blob-num js-line-number" data-line-number="739"></td>
        <td id="LC739" class="blob-code blob-code-inner js-file-line">\centering % centering table</td>
      </tr>
      <tr>
        <td id="L740" class="blob-num js-line-number" data-line-number="740"></td>
        <td id="LC740" class="blob-code blob-code-inner js-file-line">\begin{tabular}{llllllll} % creating 10 columns</td>
      </tr>
      <tr>
        <td id="L741" class="blob-num js-line-number" data-line-number="741"></td>
        <td id="LC741" class="blob-code blob-code-inner js-file-line">\hline\hline % inserting double-line</td>
      </tr>
      <tr>
        <td id="L742" class="blob-num js-line-number" data-line-number="742"></td>
        <td id="LC742" class="blob-code blob-code-inner js-file-line">\textbf{Method} &amp;\textbf{Replicates} &amp; \textbf{Counts} &amp;\textbf{norm} &amp;\textbf{k} &amp;\textbf{nss} &amp;\textbf{pnr} &amp;\textbf{v}</td>
      </tr>
      <tr>
        <td id="L743" class="blob-num js-line-number" data-line-number="743"></td>
        <td id="LC743" class="blob-code blob-code-inner js-file-line">% &amp;\multicolumn{7}{c}{Sum of Extracted Bits}</td>
      </tr>
      <tr>
        <td id="L744" class="blob-num js-line-number" data-line-number="744"></td>
        <td id="LC744" class="blob-code blob-code-inner js-file-line">\\ [0.5ex]</td>
      </tr>
      <tr>
        <td id="L745" class="blob-num js-line-number" data-line-number="745"></td>
        <td id="LC745" class="blob-code blob-code-inner js-file-line">\hline</td>
      </tr>
      <tr>
        <td id="L746" class="blob-num js-line-number" data-line-number="746"></td>
        <td id="LC746" class="blob-code blob-code-inner js-file-line">% Entering 1st row</td>
      </tr>
      <tr>
        <td id="L747" class="blob-num js-line-number" data-line-number="747"></td>
        <td id="LC747" class="blob-code blob-code-inner js-file-line">&amp; &amp;Raw &amp;rpkm, uqua, tmm &amp;0.5 \\[-1ex]</td>
      </tr>
      <tr>
        <td id="L748" class="blob-num js-line-number" data-line-number="748"></td>
        <td id="LC748" class="blob-code blob-code-inner js-file-line">\raisebox{1.5ex}{NOISeq-real} &amp; \raisebox{1.5ex}{Technical/Biological}</td>
      </tr>
      <tr>
        <td id="L749" class="blob-num js-line-number" data-line-number="749"></td>
        <td id="LC749" class="blob-code blob-code-inner js-file-line">&amp;Normalized &amp;n &amp;NULL &amp;\raisebox{1.5ex}{0} &amp;\raisebox{1.5ex}{-} &amp;\raisebox{1.5ex}{-} \\[1ex]</td>
      </tr>
      <tr>
        <td id="L750" class="blob-num js-line-number" data-line-number="750"></td>
        <td id="LC750" class="blob-code blob-code-inner js-file-line">\hline</td>
      </tr>
      <tr>
        <td id="L751" class="blob-num js-line-number" data-line-number="751"></td>
        <td id="LC751" class="blob-code blob-code-inner js-file-line">% Entering 2nd row</td>
      </tr>
      <tr>
        <td id="L752" class="blob-num js-line-number" data-line-number="752"></td>
        <td id="LC752" class="blob-code blob-code-inner js-file-line">&amp; &amp;Raw &amp;rpkm, uqua, tmm &amp;0.5 \\[-1ex]</td>
      </tr>
      <tr>
        <td id="L753" class="blob-num js-line-number" data-line-number="753"></td>
        <td id="LC753" class="blob-code blob-code-inner js-file-line">\raisebox{1.5ex}{NOISeq-sim} &amp; \raisebox{1.5ex}{None}</td>
      </tr>
      <tr>
        <td id="L754" class="blob-num js-line-number" data-line-number="754"></td>
        <td id="LC754" class="blob-code blob-code-inner js-file-line">&amp;Normalized &amp;n &amp;NULL &amp;\raisebox{1.5ex}{$\geq5$} &amp;\raisebox{1.5ex}{0.2} &amp;\raisebox{1.5ex}{0.02} \\[1ex]</td>
      </tr>
      <tr>
        <td id="L755" class="blob-num js-line-number" data-line-number="755"></td>
        <td id="LC755" class="blob-code blob-code-inner js-file-line">
</td>
      </tr>
      <tr>
        <td id="L756" class="blob-num js-line-number" data-line-number="756"></td>
        <td id="LC756" class="blob-code blob-code-inner js-file-line">\hline % inserts single-line</td>
      </tr>
      <tr>
        <td id="L757" class="blob-num js-line-number" data-line-number="757"></td>
        <td id="LC757" class="blob-code blob-code-inner js-file-line">\end{tabular}</td>
      </tr>
      <tr>
        <td id="L758" class="blob-num js-line-number" data-line-number="758"></td>
        <td id="LC758" class="blob-code blob-code-inner js-file-line">\label{table:summary}</td>
      </tr>
      <tr>
        <td id="L759" class="blob-num js-line-number" data-line-number="759"></td>
        <td id="LC759" class="blob-code blob-code-inner js-file-line">\end{table}</td>
      </tr>
      <tr>
        <td id="L760" class="blob-num js-line-number" data-line-number="760"></td>
        <td id="LC760" class="blob-code blob-code-inner js-file-line">
</td>
      </tr>
      <tr>
        <td id="L761" class="blob-num js-line-number" data-line-number="761"></td>
        <td id="LC761" class="blob-code blob-code-inner js-file-line">
</td>
      </tr>
      <tr>
        <td id="L762" class="blob-num js-line-number" data-line-number="762"></td>
        <td id="LC762" class="blob-code blob-code-inner js-file-line">Please note that \texttt{norm = &quot;n&quot;} argument should be used in \texttt{noiseq} or \texttt{noiseqbio} whenever the data have been previously normalized or corrected for a batch effect.</td>
      </tr>
      <tr>
        <td id="L763" class="blob-num js-line-number" data-line-number="763"></td>
        <td id="LC763" class="blob-code blob-code-inner js-file-line">
</td>
      </tr>
      <tr>
        <td id="L764" class="blob-num js-line-number" data-line-number="764"></td>
        <td id="LC764" class="blob-code blob-code-inner js-file-line">
</td>
      </tr>
      <tr>
        <td id="L765" class="blob-num js-line-number" data-line-number="765"></td>
        <td id="LC765" class="blob-code blob-code-inner js-file-line">
</td>
      </tr>
      <tr>
        <td id="L766" class="blob-num js-line-number" data-line-number="766"></td>
        <td id="LC766" class="blob-code blob-code-inner js-file-line">\subsubsection{NOISeq-real: using available replicates}</td>
      </tr>
      <tr>
        <td id="L767" class="blob-num js-line-number" data-line-number="767"></td>
        <td id="LC767" class="blob-code blob-code-inner js-file-line">
</td>
      </tr>
      <tr>
        <td id="L768" class="blob-num js-line-number" data-line-number="768"></td>
        <td id="LC768" class="blob-code blob-code-inner js-file-line">NOISeq-real estimates the probability distribution for M and D in an empirical way, by computing M and D values for every pair </td>
      </tr>
      <tr>
        <td id="L769" class="blob-num js-line-number" data-line-number="769"></td>
        <td id="LC769" class="blob-code blob-code-inner js-file-line">of replicates within the same experimental condition and for every feature. Then, all these values are pooled together to </td>
      </tr>
      <tr>
        <td id="L770" class="blob-num js-line-number" data-line-number="770"></td>
        <td id="LC770" class="blob-code blob-code-inner js-file-line">generate the noise distribution. Two replicates in one of the experimental conditions are enough to run the algorithm. If </td>
      </tr>
      <tr>
        <td id="L771" class="blob-num js-line-number" data-line-number="771"></td>
        <td id="LC771" class="blob-code blob-code-inner js-file-line">the number of possible comparisons within a certain condition is higher than 30, in order to reduce computation time, 30 </td>
      </tr>
      <tr>
        <td id="L772" class="blob-num js-line-number" data-line-number="772"></td>
        <td id="LC772" class="blob-code blob-code-inner js-file-line">pairwise comparisons are randomly chosen when estimating noise distribution.</td>
      </tr>
      <tr>
        <td id="L773" class="blob-num js-line-number" data-line-number="773"></td>
        <td id="LC773" class="blob-code blob-code-inner js-file-line">
</td>
      </tr>
      <tr>
        <td id="L774" class="blob-num js-line-number" data-line-number="774"></td>
        <td id="LC774" class="blob-code blob-code-inner js-file-line">It should be noted that biological replicates are necessary if the goal is to make any inference about the population. Deriving </td>
      </tr>
      <tr>
        <td id="L775" class="blob-num js-line-number" data-line-number="775"></td>
        <td id="LC775" class="blob-code blob-code-inner js-file-line">differential expression from technical replicates is useful for drawing conclusions about the specific samples being compared in the</td>
      </tr>
      <tr>
        <td id="L776" class="blob-num js-line-number" data-line-number="776"></td>
        <td id="LC776" class="blob-code blob-code-inner js-file-line"> study but not for extending these conclusions to the whole population.</td>
      </tr>
      <tr>
        <td id="L777" class="blob-num js-line-number" data-line-number="777"></td>
        <td id="LC777" class="blob-code blob-code-inner js-file-line">
</td>
      </tr>
      <tr>
        <td id="L778" class="blob-num js-line-number" data-line-number="778"></td>
        <td id="LC778" class="blob-code blob-code-inner js-file-line">In RNA-seq or similar sequencing technologies, the counts from technical replicates (e.g. lanes) can be summed up. </td>
      </tr>
      <tr>
        <td id="L779" class="blob-num js-line-number" data-line-number="779"></td>
        <td id="LC779" class="blob-code blob-code-inner js-file-line">Thus, this is the way the algorithm summarizes the information from</td>
      </tr>
      <tr>
        <td id="L780" class="blob-num js-line-number" data-line-number="780"></td>
        <td id="LC780" class="blob-code blob-code-inner js-file-line">technical replicates to compute M and D signal values (between different conditions). However, for biological replicates, </td>
      </tr>
      <tr>
        <td id="L781" class="blob-num js-line-number" data-line-number="781"></td>
        <td id="LC781" class="blob-code blob-code-inner js-file-line">other summary statistics such us the mean may be more meaningful. \noiseq{} calculates the mean of the biological replicates </td>
      </tr>
      <tr>
        <td id="L782" class="blob-num js-line-number" data-line-number="782"></td>
        <td id="LC782" class="blob-code blob-code-inner js-file-line">but we strongly recommend to use \noiseqbio{} when having biological replicates.</td>
      </tr>
      <tr>
        <td id="L783" class="blob-num js-line-number" data-line-number="783"></td>
        <td id="LC783" class="blob-code blob-code-inner js-file-line">
</td>
      </tr>
      <tr>
        <td id="L784" class="blob-num js-line-number" data-line-number="784"></td>
        <td id="LC784" class="blob-code blob-code-inner js-file-line">Here there is an example with technical replicates and count data normalized by \code{rpkm} method. Please note that, </td>
      </tr>
      <tr>
        <td id="L785" class="blob-num js-line-number" data-line-number="785"></td>
        <td id="LC785" class="blob-code blob-code-inner js-file-line">since the factor ``Tissue&#39;&#39; has two levels, we do not need to indicate which conditions are to be compared. </td>
      </tr>
      <tr>
        <td id="L786" class="blob-num js-line-number" data-line-number="786"></td>
        <td id="LC786" class="blob-code blob-code-inner js-file-line">&lt;&lt;results&gt;&gt;=</td>
      </tr>
      <tr>
        <td id="L787" class="blob-num js-line-number" data-line-number="787"></td>
        <td id="LC787" class="blob-code blob-code-inner js-file-line">mynoiseq = noiseq(mydata, k = 0.5, norm = &quot;rpkm&quot;, factor=&quot;Tissue&quot;, pnr = 0.2, </td>
      </tr>
      <tr>
        <td id="L788" class="blob-num js-line-number" data-line-number="788"></td>
        <td id="LC788" class="blob-code blob-code-inner js-file-line">                  nss = 5, v = 0.02, lc = 1, replicates = &quot;technical&quot;)</td>
      </tr>
      <tr>
        <td id="L789" class="blob-num js-line-number" data-line-number="789"></td>
        <td id="LC789" class="blob-code blob-code-inner js-file-line">head(mynoiseq@results[[1]])</td>
      </tr>
      <tr>
        <td id="L790" class="blob-num js-line-number" data-line-number="790"></td>
        <td id="LC790" class="blob-code blob-code-inner js-file-line">@ </td>
      </tr>
      <tr>
        <td id="L791" class="blob-num js-line-number" data-line-number="791"></td>
        <td id="LC791" class="blob-code blob-code-inner js-file-line">
</td>
      </tr>
      <tr>
        <td id="L792" class="blob-num js-line-number" data-line-number="792"></td>
        <td id="LC792" class="blob-code blob-code-inner js-file-line">NA values would be returned if the gene had 0 counts in all the samples. In that case, the gene would not be used to compute differential expression.</td>
      </tr>
      <tr>
        <td id="L793" class="blob-num js-line-number" data-line-number="793"></td>
        <td id="LC793" class="blob-code blob-code-inner js-file-line">
</td>
      </tr>
      <tr>
        <td id="L794" class="blob-num js-line-number" data-line-number="794"></td>
        <td id="LC794" class="blob-code blob-code-inner js-file-line">Now imagine you want to compare tissues within the same sequencing run. Then, see the following example on how to apply NOISeq </td>
      </tr>
      <tr>
        <td id="L795" class="blob-num js-line-number" data-line-number="795"></td>
        <td id="LC795" class="blob-code blob-code-inner js-file-line">on count data with technical replicates, TMM normalization, and no length correction. </td>
      </tr>
      <tr>
        <td id="L796" class="blob-num js-line-number" data-line-number="796"></td>
        <td id="LC796" class="blob-code blob-code-inner js-file-line">As ``TissueRun&#39;&#39; has more than two levels we have to indicate which levels (conditions) are to be compared: </td>
      </tr>
      <tr>
        <td id="L797" class="blob-num js-line-number" data-line-number="797"></td>
        <td id="LC797" class="blob-code blob-code-inner js-file-line">&lt;&lt;results=hide&gt;&gt;=</td>
      </tr>
      <tr>
        <td id="L798" class="blob-num js-line-number" data-line-number="798"></td>
        <td id="LC798" class="blob-code blob-code-inner js-file-line">mynoiseq.tmm = noiseq(mydata, k = 0.5, norm = &quot;tmm&quot;, factor=&quot;TissueRun&quot;, </td>
      </tr>
      <tr>
        <td id="L799" class="blob-num js-line-number" data-line-number="799"></td>
        <td id="LC799" class="blob-code blob-code-inner js-file-line">                      conditions = c(&quot;Kidney_1&quot;,&quot;Liver_1&quot;), lc = 0, replicates = &quot;technical&quot;)</td>
      </tr>
      <tr>
        <td id="L800" class="blob-num js-line-number" data-line-number="800"></td>
        <td id="LC800" class="blob-code blob-code-inner js-file-line">@ </td>
      </tr>
      <tr>
        <td id="L801" class="blob-num js-line-number" data-line-number="801"></td>
        <td id="LC801" class="blob-code blob-code-inner js-file-line">
</td>
      </tr>
      <tr>
        <td id="L802" class="blob-num js-line-number" data-line-number="802"></td>
        <td id="LC802" class="blob-code blob-code-inner js-file-line">
</td>
      </tr>
      <tr>
        <td id="L803" class="blob-num js-line-number" data-line-number="803"></td>
        <td id="LC803" class="blob-code blob-code-inner js-file-line">
</td>
      </tr>
      <tr>
        <td id="L804" class="blob-num js-line-number" data-line-number="804"></td>
        <td id="LC804" class="blob-code blob-code-inner js-file-line">
</td>
      </tr>
      <tr>
        <td id="L805" class="blob-num js-line-number" data-line-number="805"></td>
        <td id="LC805" class="blob-code blob-code-inner js-file-line">\subsubsection{NOISeq-sim: no replicates available}</td>
      </tr>
      <tr>
        <td id="L806" class="blob-num js-line-number" data-line-number="806"></td>
        <td id="LC806" class="blob-code blob-code-inner js-file-line">
</td>
      </tr>
      <tr>
        <td id="L807" class="blob-num js-line-number" data-line-number="807"></td>
        <td id="LC807" class="blob-code blob-code-inner js-file-line">When there are no replicates available for any of the experimental conditions, \noiseq{} can simulate technical replicates. </td>
      </tr>
      <tr>
        <td id="L808" class="blob-num js-line-number" data-line-number="808"></td>
        <td id="LC808" class="blob-code blob-code-inner js-file-line">The simulation relies on the assumption that read counts follow a multinomial distribution, where </td>
      </tr>
      <tr>
        <td id="L809" class="blob-num js-line-number" data-line-number="809"></td>
        <td id="LC809" class="blob-code blob-code-inner js-file-line">probabilities for each class (feature) in the multinomial distribution are the probability of a read to </td>
      </tr>
      <tr>
        <td id="L810" class="blob-num js-line-number" data-line-number="810"></td>
        <td id="LC810" class="blob-code blob-code-inner js-file-line">map to that feature. These mapping probabilities are approximated by using counts in the only sample of the</td>
      </tr>
      <tr>
        <td id="L811" class="blob-num js-line-number" data-line-number="811"></td>
        <td id="LC811" class="blob-code blob-code-inner js-file-line"> corresponding experimental condition. Counts equal to zero are replaced with $k$&gt;0 to give all features</td>
      </tr>
      <tr>
        <td id="L812" class="blob-num js-line-number" data-line-number="812"></td>
        <td id="LC812" class="blob-code blob-code-inner js-file-line"> some chance to appear.</td>
      </tr>
      <tr>
        <td id="L813" class="blob-num js-line-number" data-line-number="813"></td>
        <td id="LC813" class="blob-code blob-code-inner js-file-line">
</td>
      </tr>
      <tr>
        <td id="L814" class="blob-num js-line-number" data-line-number="814"></td>
        <td id="LC814" class="blob-code blob-code-inner js-file-line">Given the sequencing depth (total amount of reads) of the unique available sample, the size of the simulated samples is a percentage</td>
      </tr>
      <tr>
        <td id="L815" class="blob-num js-line-number" data-line-number="815"></td>
        <td id="LC815" class="blob-code blob-code-inner js-file-line"> (parameter $pnr$) of this sequencing depth, allowing a small variability (given by the parameter $v$). The number of </td>
      </tr>
      <tr>
        <td id="L816" class="blob-num js-line-number" data-line-number="816"></td>
        <td id="LC816" class="blob-code blob-code-inner js-file-line">replicates to be simulated is provided by $nss$ parameter.</td>
      </tr>
      <tr>
        <td id="L817" class="blob-num js-line-number" data-line-number="817"></td>
        <td id="LC817" class="blob-code blob-code-inner js-file-line">
</td>
      </tr>
      <tr>
        <td id="L818" class="blob-num js-line-number" data-line-number="818"></td>
        <td id="LC818" class="blob-code blob-code-inner js-file-line">Our dataset do has replicates but, providing it had not, you would use NOISeq-sim as in the following example in which the simulation parameters </td>
      </tr>
      <tr>
        <td id="L819" class="blob-num js-line-number" data-line-number="819"></td>
        <td id="LC819" class="blob-code blob-code-inner js-file-line">have to be chosen ($pnr$, $nss$ and $v$): </td>
      </tr>
      <tr>
        <td id="L820" class="blob-num js-line-number" data-line-number="820"></td>
        <td id="LC820" class="blob-code blob-code-inner js-file-line">
</td>
      </tr>
      <tr>
        <td id="L821" class="blob-num js-line-number" data-line-number="821"></td>
        <td id="LC821" class="blob-code blob-code-inner js-file-line">&lt;&lt;results=hide&gt;&gt;=</td>
      </tr>
      <tr>
        <td id="L822" class="blob-num js-line-number" data-line-number="822"></td>
        <td id="LC822" class="blob-code blob-code-inner js-file-line">myresults &lt;- noiseq(mydata, factor = &quot;Tissue&quot;, k = NULL, norm=&quot;n&quot;, pnr = 0.2, </td>
      </tr>
      <tr>
        <td id="L823" class="blob-num js-line-number" data-line-number="823"></td>
        <td id="LC823" class="blob-code blob-code-inner js-file-line">                    nss = 5, v = 0.02, lc = 1, replicates = &quot;no&quot;)</td>
      </tr>
      <tr>
        <td id="L824" class="blob-num js-line-number" data-line-number="824"></td>
        <td id="LC824" class="blob-code blob-code-inner js-file-line">@</td>
      </tr>
      <tr>
        <td id="L825" class="blob-num js-line-number" data-line-number="825"></td>
        <td id="LC825" class="blob-code blob-code-inner js-file-line">
</td>
      </tr>
      <tr>
        <td id="L826" class="blob-num js-line-number" data-line-number="826"></td>
        <td id="LC826" class="blob-code blob-code-inner js-file-line">
</td>
      </tr>
      <tr>
        <td id="L827" class="blob-num js-line-number" data-line-number="827"></td>
        <td id="LC827" class="blob-code blob-code-inner js-file-line">
</td>
      </tr>
      <tr>
        <td id="L828" class="blob-num js-line-number" data-line-number="828"></td>
        <td id="LC828" class="blob-code blob-code-inner js-file-line">\subsubsection{NOISeqBIO} \label{sec_param2}</td>
      </tr>
      <tr>
        <td id="L829" class="blob-num js-line-number" data-line-number="829"></td>
        <td id="LC829" class="blob-code blob-code-inner js-file-line">
</td>
      </tr>
      <tr>
        <td id="L830" class="blob-num js-line-number" data-line-number="830"></td>
        <td id="LC830" class="blob-code blob-code-inner js-file-line">NOISeqBIO is optimized for the use on biological replicates (at least 2 per condition). It was developed by joining the philosophy of our previous work </td>
      </tr>
      <tr>
        <td id="L831" class="blob-num js-line-number" data-line-number="831"></td>
        <td id="LC831" class="blob-code blob-code-inner js-file-line">together with the ideas from Efron \emph{et al.} in \cite{Efron2001}. In our case, we defined the differential expression statistic </td>
      </tr>
      <tr>
        <td id="L832" class="blob-num js-line-number" data-line-number="832"></td>
        <td id="LC832" class="blob-code blob-code-inner js-file-line">$\theta$ as $(M+D)/2$, where $M$ and $D$ are the statistics defined in the previous section but</td>
      </tr>
      <tr>
        <td id="L833" class="blob-num js-line-number" data-line-number="833"></td>
        <td id="LC833" class="blob-code blob-code-inner js-file-line">including a correction for the biological variability of the corresponding feature. </td>
      </tr>
      <tr>
        <td id="L834" class="blob-num js-line-number" data-line-number="834"></td>
        <td id="LC834" class="blob-code blob-code-inner js-file-line">The probability distribution of $\theta$ can be described as a mixture of two distributions: one for features changing between conditions and the other</td>
      </tr>
      <tr>
        <td id="L835" class="blob-num js-line-number" data-line-number="835"></td>
        <td id="LC835" class="blob-code blob-code-inner js-file-line">for invariant features. Thus, the mixture distribution $f$ can be written as: $f(\theta) = p_{0}f_{0}(\theta)+p_{1}f_{1}(\theta)$, </td>
      </tr>
      <tr>
        <td id="L836" class="blob-num js-line-number" data-line-number="836"></td>
        <td id="LC836" class="blob-code blob-code-inner js-file-line">where $p_{0}$ is the probability for a feature to have the same expression in both conditions and</td>
      </tr>
      <tr>
        <td id="L837" class="blob-num js-line-number" data-line-number="837"></td>
        <td id="LC837" class="blob-code blob-code-inner js-file-line">$p_{1} = 1-p_{0}$ is the probability for a feature to have different expression between conditions. </td>
      </tr>
      <tr>
        <td id="L838" class="blob-num js-line-number" data-line-number="838"></td>
        <td id="LC838" class="blob-code blob-code-inner js-file-line">$f_{0}$ and $f_{1}$ are, respectively, the densities of $\theta$ for features with no change in expression</td>
      </tr>
      <tr>
        <td id="L839" class="blob-num js-line-number" data-line-number="839"></td>
        <td id="LC839" class="blob-code blob-code-inner js-file-line">between conditions and for differentially expressed features. If one of both distributions can be estimated, </td>
      </tr>
      <tr>
        <td id="L840" class="blob-num js-line-number" data-line-number="840"></td>
        <td id="LC840" class="blob-code blob-code-inner js-file-line">the probability of a feature to belong to one of the two groups can be calculated. Thus, the algorithm consists of the following steps:</td>
      </tr>
      <tr>
        <td id="L841" class="blob-num js-line-number" data-line-number="841"></td>
        <td id="LC841" class="blob-code blob-code-inner js-file-line">\begin{enumerate}</td>
      </tr>
      <tr>
        <td id="L842" class="blob-num js-line-number" data-line-number="842"></td>
        <td id="LC842" class="blob-code blob-code-inner js-file-line"> \item Computing $\theta$ values. \\</td>
      </tr>
      <tr>
        <td id="L843" class="blob-num js-line-number" data-line-number="843"></td>
        <td id="LC843" class="blob-code blob-code-inner js-file-line">$M$ and $D$ are corrected for the biological variability: $M^* = \dfrac{M}{a_{0}+\hat \sigma_M}$ and $D^* = \dfrac{D_s}{a_{0}+\hat \sigma_D}$,</td>
      </tr>
      <tr>
        <td id="L844" class="blob-num js-line-number" data-line-number="844"></td>
        <td id="LC844" class="blob-code blob-code-inner js-file-line">where $\hat \sigma^2_M$ and $\hat \sigma^2_D$  are the standard errors of $M_s$ and $D_s$ statistics, respectively, and $a_0$ is computed as a given percentile of all the values in</td>
      </tr>
      <tr>
        <td id="L845" class="blob-num js-line-number" data-line-number="845"></td>
        <td id="LC845" class="blob-code blob-code-inner js-file-line">$\hat \sigma_M$ or $\hat \sigma_D$,  as in \cite{Efron2001} (the authors suggest the percentile 90th as the best option, which is the default option of the parameter ``a0per&quot; that</td>
      </tr>
      <tr>
        <td id="L846" class="blob-num js-line-number" data-line-number="846"></td>
        <td id="LC846" class="blob-code blob-code-inner js-file-line">may be changed by the user). To compute the $\theta$ statistic, the $M$ and $D$ statistics are combined:  $\theta = \dfrac{M^* + D^*}{2}$.</td>
      </tr>
      <tr>
        <td id="L847" class="blob-num js-line-number" data-line-number="847"></td>
        <td id="LC847" class="blob-code blob-code-inner js-file-line">
</td>
      </tr>
      <tr>
        <td id="L848" class="blob-num js-line-number" data-line-number="848"></td>
        <td id="LC848" class="blob-code blob-code-inner js-file-line"> \item Estimating the values of the $\theta$ statistic when there is no change in expression, i.e. the null statistic $\theta_{0}$. \\</td>
      </tr>
      <tr>
        <td id="L849" class="blob-num js-line-number" data-line-number="849"></td>
        <td id="LC849" class="blob-code blob-code-inner js-file-line">In order to compute the null density $f_{0}$ afterwards, we first need to estimate the values of the $\theta$-scores for features with no change</td>
      </tr>
      <tr>
        <td id="L850" class="blob-num js-line-number" data-line-number="850"></td>
        <td id="LC850" class="blob-code blob-code-inner js-file-line">between conditions. To do that, we permute $r$ times (parameter that may be set by the user) the labels of samples between conditions, compute $\theta$ values as above and pool</td>
      </tr>
      <tr>
        <td id="L851" class="blob-num js-line-number" data-line-number="851"></td>
        <td id="LC851" class="blob-code blob-code-inner js-file-line">them to obtain $\theta_{0}$.</td>
      </tr>
      <tr>
        <td id="L852" class="blob-num js-line-number" data-line-number="852"></td>
        <td id="LC852" class="blob-code blob-code-inner js-file-line">
</td>
      </tr>
      <tr>
        <td id="L853" class="blob-num js-line-number" data-line-number="853"></td>
        <td id="LC853" class="blob-code blob-code-inner js-file-line"> \item Estimating the probability density functions $f$ and $f_{0}$. \\</td>
      </tr>
      <tr>
        <td id="L854" class="blob-num js-line-number" data-line-number="854"></td>
        <td id="LC854" class="blob-code blob-code-inner js-file-line">We estimate $f$ and $f_{0}$ with a kernel density estimator (KDE) with Gaussian kernel and smoothing parameter ``adj&quot; as indicated by the user.</td>
      </tr>
      <tr>
        <td id="L855" class="blob-num js-line-number" data-line-number="855"></td>
        <td id="LC855" class="blob-code blob-code-inner js-file-line">
</td>
      </tr>
      <tr>
        <td id="L856" class="blob-num js-line-number" data-line-number="856"></td>
        <td id="LC856" class="blob-code blob-code-inner js-file-line"> \item Computing the probability of differential expression given the ratio $f_{0}/f$ and an estimation $\hat{p}_{0}$ for $p_{0}$. If $\theta=z$</td>
      </tr>
      <tr>
        <td id="L857" class="blob-num js-line-number" data-line-number="857"></td>
        <td id="LC857" class="blob-code blob-code-inner js-file-line">for a given feature, this probability of differential expression can be computed as $p_{1}(z)=1-\hat{p}_{0}f_{0}(z)/f(z)$.\\</td>
      </tr>
      <tr>
        <td id="L858" class="blob-num js-line-number" data-line-number="858"></td>
        <td id="LC858" class="blob-code blob-code-inner js-file-line">To estimate $p_{0}$, the following upper bound is taken, as suggested in \cite{Efron2001}: $p_{0} \leq \min_{Z} \{f(Z)/f_{0}(Z) \}$.\\</td>
      </tr>
      <tr>
        <td id="L859" class="blob-num js-line-number" data-line-number="859"></td>
        <td id="LC859" class="blob-code blob-code-inner js-file-line">Moreover, it is shown in \cite{Efron2001} that the FDR defined by Benjamini and Hochberg can be considered equivalent</td>
      </tr>
      <tr>
        <td id="L860" class="blob-num js-line-number" data-line-number="860"></td>
        <td id="LC860" class="blob-code blob-code-inner js-file-line">to the \emph{a posteriori} probability $p_0(z) = 1 - p_1(z)$ we are calculating.</td>
      </tr>
      <tr>
        <td id="L861" class="blob-num js-line-number" data-line-number="861"></td>
        <td id="LC861" class="blob-code blob-code-inner js-file-line">\end{enumerate}</td>
      </tr>
      <tr>
        <td id="L862" class="blob-num js-line-number" data-line-number="862"></td>
        <td id="LC862" class="blob-code blob-code-inner js-file-line">
</td>
      </tr>
      <tr>
        <td id="L863" class="blob-num js-line-number" data-line-number="863"></td>
        <td id="LC863" class="blob-code blob-code-inner js-file-line">When too few replicates are available for each condition, the null distribution is very poor since the number of different permutations is</td>
      </tr>
      <tr>
        <td id="L864" class="blob-num js-line-number" data-line-number="864"></td>
        <td id="LC864" class="blob-code blob-code-inner js-file-line">low. For those cases (number of replicates in one of the conditions less than 5), it is convenient to borrow information across genes. Our proposal</td>
      </tr>
      <tr>
        <td id="L865" class="blob-num js-line-number" data-line-number="865"></td>
        <td id="LC865" class="blob-code blob-code-inner js-file-line">consists of clustering all genes according to their</td>
      </tr>
      <tr>
        <td id="L866" class="blob-num js-line-number" data-line-number="866"></td>
        <td id="LC866" class="blob-code blob-code-inner js-file-line">expression values across replicates using the k-means method. For each cluster  $k$ of genes, we consider the expression values of all the</td>
      </tr>
      <tr>
        <td id="L867" class="blob-num js-line-number" data-line-number="867"></td>
        <td id="LC867" class="blob-code blob-code-inner js-file-line">genes in the cluster as observations within the corresponding condition (replicates) and then we shuffle this submatrix $r \times g_k$</td>
      </tr>
      <tr>
        <td id="L868" class="blob-num js-line-number" data-line-number="868"></td>
        <td id="LC868" class="blob-code blob-code-inner js-file-line">times, where $g_k$ is the number of genes within cluster $k$. If $r \times g_k$ is higher than 1000, we compute 1000 permutations in that cluster.  For each permutation, we calculate $M$ and $D$ values and their</td>
      </tr>
      <tr>
        <td id="L869" class="blob-num js-line-number" data-line-number="869"></td>
        <td id="LC869" class="blob-code blob-code-inner js-file-line">corresponding standard errors. In order to reduce the computing time, if $g_k \geq 1000$, we again subdivide cluster $k$ in subclusters</td>
      </tr>
      <tr>
        <td id="L870" class="blob-num js-line-number" data-line-number="870"></td>
        <td id="LC870" class="blob-code blob-code-inner js-file-line">with k-means algorithm.</td>
      </tr>
      <tr>
        <td id="L871" class="blob-num js-line-number" data-line-number="871"></td>
        <td id="LC871" class="blob-code blob-code-inner js-file-line">
</td>
      </tr>
      <tr>
        <td id="L872" class="blob-num js-line-number" data-line-number="872"></td>
        <td id="LC872" class="blob-code blob-code-inner js-file-line">We will consider that Marioni&#39;s data have biological replicates for the following example. In this case, the method 2 (Wilcoxon test) to filter low counts</td>
      </tr>
      <tr>
        <td id="L873" class="blob-num js-line-number" data-line-number="873"></td>
        <td id="LC873" class="blob-code blob-code-inner js-file-line">is used. Please, use \code{?noiseqbio} to know more about the parameters of the function.</td>
      </tr>
      <tr>
        <td id="L874" class="blob-num js-line-number" data-line-number="874"></td>
        <td id="LC874" class="blob-code blob-code-inner js-file-line">
</td>
      </tr>
      <tr>
        <td id="L875" class="blob-num js-line-number" data-line-number="875"></td>
        <td id="LC875" class="blob-code blob-code-inner js-file-line">&lt;&lt;results=hide&gt;&gt;=</td>
      </tr>
      <tr>
        <td id="L876" class="blob-num js-line-number" data-line-number="876"></td>
        <td id="LC876" class="blob-code blob-code-inner js-file-line">mynoiseqbio = noiseqbio(mydata, k = 0.5, norm = &quot;rpkm&quot;, factor=&quot;Tissue&quot;, lc = 1, r = 20, adj = 1.5, plot = FALSE,</td>
      </tr>
      <tr>
        <td id="L877" class="blob-num js-line-number" data-line-number="877"></td>
        <td id="LC877" class="blob-code blob-code-inner js-file-line">                        a0per = 0.9, random.seed = 12345, filter = 2)</td>
      </tr>
      <tr>
        <td id="L878" class="blob-num js-line-number" data-line-number="878"></td>
        <td id="LC878" class="blob-code blob-code-inner js-file-line">@ </td>
      </tr>
      <tr>
        <td id="L879" class="blob-num js-line-number" data-line-number="879"></td>
        <td id="LC879" class="blob-code blob-code-inner js-file-line">
</td>
      </tr>
      <tr>
        <td id="L880" class="blob-num js-line-number" data-line-number="880"></td>
        <td id="LC880" class="blob-code blob-code-inner js-file-line">
</td>
      </tr>
      <tr>
        <td id="L881" class="blob-num js-line-number" data-line-number="881"></td>
        <td id="LC881" class="blob-code blob-code-inner js-file-line">
</td>
      </tr>
      <tr>
        <td id="L882" class="blob-num js-line-number" data-line-number="882"></td>
        <td id="LC882" class="blob-code blob-code-inner js-file-line">
</td>
      </tr>
      <tr>
        <td id="L883" class="blob-num js-line-number" data-line-number="883"></td>
        <td id="LC883" class="blob-code blob-code-inner js-file-line">\subsection{Results}\label{sec_deg}</td>
      </tr>
      <tr>
        <td id="L884" class="blob-num js-line-number" data-line-number="884"></td>
        <td id="LC884" class="blob-code blob-code-inner js-file-line">
</td>
      </tr>
      <tr>
        <td id="L885" class="blob-num js-line-number" data-line-number="885"></td>
        <td id="LC885" class="blob-code blob-code-inner js-file-line">\subsubsection{NOISeq output object}</td>
      </tr>
      <tr>
        <td id="L886" class="blob-num js-line-number" data-line-number="886"></td>
        <td id="LC886" class="blob-code blob-code-inner js-file-line">
</td>
      </tr>
      <tr>
        <td id="L887" class="blob-num js-line-number" data-line-number="887"></td>
        <td id="LC887" class="blob-code blob-code-inner js-file-line">\noiseq{} returns an \code{Output} object containing the following elements:</td>
      </tr>
      <tr>
        <td id="L888" class="blob-num js-line-number" data-line-number="888"></td>
        <td id="LC888" class="blob-code blob-code-inner js-file-line">
</td>
      </tr>
      <tr>
        <td id="L889" class="blob-num js-line-number" data-line-number="889"></td>
        <td id="LC889" class="blob-code blob-code-inner js-file-line">\begin{itemize}</td>
      </tr>
      <tr>
        <td id="L890" class="blob-num js-line-number" data-line-number="890"></td>
        <td id="LC890" class="blob-code blob-code-inner js-file-line">
</td>
      </tr>
      <tr>
        <td id="L891" class="blob-num js-line-number" data-line-number="891"></td>
        <td id="LC891" class="blob-code blob-code-inner js-file-line">\item</td>
      </tr>
      <tr>
        <td id="L892" class="blob-num js-line-number" data-line-number="892"></td>
        <td id="LC892" class="blob-code blob-code-inner js-file-line">  \texttt{comparison}: String indicating the two experimental conditions being compared and the sense of the comparison.</td>
      </tr>
      <tr>
        <td id="L893" class="blob-num js-line-number" data-line-number="893"></td>
        <td id="LC893" class="blob-code blob-code-inner js-file-line">\item</td>
      </tr>
      <tr>
        <td id="L894" class="blob-num js-line-number" data-line-number="894"></td>
        <td id="LC894" class="blob-code blob-code-inner js-file-line">  \texttt{factor}: String indicating the factor chosen to compute the differential expression.</td>
      </tr>
      <tr>
        <td id="L895" class="blob-num js-line-number" data-line-number="895"></td>
        <td id="LC895" class="blob-code blob-code-inner js-file-line">\item</td>
      </tr>
      <tr>
        <td id="L896" class="blob-num js-line-number" data-line-number="896"></td>
        <td id="LC896" class="blob-code blob-code-inner js-file-line">  \texttt{k}: Value to replace zeros in order to avoid indetermination when computing logarithms. </td>
      </tr>
      <tr>
        <td id="L897" class="blob-num js-line-number" data-line-number="897"></td>
        <td id="LC897" class="blob-code blob-code-inner js-file-line">\item</td>
      </tr>
      <tr>
        <td id="L898" class="blob-num js-line-number" data-line-number="898"></td>
        <td id="LC898" class="blob-code blob-code-inner js-file-line">  \texttt{lc}: Correction factor for length normalization. Counts are divided by $length^{lc}$. </td>
      </tr>
      <tr>
        <td id="L899" class="blob-num js-line-number" data-line-number="899"></td>
        <td id="LC899" class="blob-code blob-code-inner js-file-line">\item</td>
      </tr>
      <tr>
        <td id="L900" class="blob-num js-line-number" data-line-number="900"></td>
        <td id="LC900" class="blob-code blob-code-inner js-file-line">  \texttt{method}: Normalization method chosen. </td>
      </tr>
      <tr>
        <td id="L901" class="blob-num js-line-number" data-line-number="901"></td>
        <td id="LC901" class="blob-code blob-code-inner js-file-line">\item</td>
      </tr>
      <tr>
        <td id="L902" class="blob-num js-line-number" data-line-number="902"></td>
        <td id="LC902" class="blob-code blob-code-inner js-file-line">  \texttt{replicates}: Type of replicates: ``technical&quot; for technical replicates and ``biological&quot; for biological ones.</td>
      </tr>
      <tr>
        <td id="L903" class="blob-num js-line-number" data-line-number="903"></td>
        <td id="LC903" class="blob-code blob-code-inner js-file-line">\item</td>
      </tr>
      <tr>
        <td id="L904" class="blob-num js-line-number" data-line-number="904"></td>
        <td id="LC904" class="blob-code blob-code-inner js-file-line">  \texttt{results}: R data frame containing the differential expression results, where each row corresponds to a feature. The columns are: Expression values for each condition to</td>
      </tr>
      <tr>
        <td id="L905" class="blob-num js-line-number" data-line-number="905"></td>
        <td id="LC905" class="blob-code blob-code-inner js-file-line">be used by \code{NOISeq} or \code{NOISeqBIO} (the columns names are the levels of the factor); differential expression statistics (columns``M&quot; and ``D&quot; for \code{NOISeq} or</td>
      </tr>
      <tr>
        <td id="L906" class="blob-num js-line-number" data-line-number="906"></td>
        <td id="LC906" class="blob-code blob-code-inner js-file-line">``theta&quot; for \code{NOISeqBIO}); probability of differential expression (``prob&quot;); ``ranking&quot;, which is a summary statistic of ``M&quot; and ``D&quot; values equal to $-sign(M) \times</td>
      </tr>
      <tr>
        <td id="L907" class="blob-num js-line-number" data-line-number="907"></td>
        <td id="LC907" class="blob-code blob-code-inner js-file-line">\sqrt{M^2 + D^2}$, than can be used for instance in gene set enrichment analysis (only for \code{NOISeq}); ``Length&quot; of each feature (if provided); ``GC&quot; content of each feature</td>
      </tr>
      <tr>
        <td id="L908" class="blob-num js-line-number" data-line-number="908"></td>
        <td id="LC908" class="blob-code blob-code-inner js-file-line">(if provided);</td>
      </tr>
      <tr>
        <td id="L909" class="blob-num js-line-number" data-line-number="909"></td>
        <td id="LC909" class="blob-code blob-code-inner js-file-line">chromosome where the feature is (``Chrom&quot;), if provided; start and end position of the feature within the chromosome (``GeneStart&quot;, ``GeneEnd&quot;), if provided; feature biotype</td>
      </tr>
      <tr>
        <td id="L910" class="blob-num js-line-number" data-line-number="910"></td>
        <td id="LC910" class="blob-code blob-code-inner js-file-line">(``Biotype&quot;), if provided. </td>
      </tr>
      <tr>
        <td id="L911" class="blob-num js-line-number" data-line-number="911"></td>
        <td id="LC911" class="blob-code blob-code-inner js-file-line">\item</td>
      </tr>
      <tr>
        <td id="L912" class="blob-num js-line-number" data-line-number="912"></td>
        <td id="LC912" class="blob-code blob-code-inner js-file-line">  \texttt{nss}: Number of samples to be simulated for each condition (only when there are not replicates available). </td>
      </tr>
      <tr>
        <td id="L913" class="blob-num js-line-number" data-line-number="913"></td>
        <td id="LC913" class="blob-code blob-code-inner js-file-line">\item</td>
      </tr>
      <tr>
        <td id="L914" class="blob-num js-line-number" data-line-number="914"></td>
        <td id="LC914" class="blob-code blob-code-inner js-file-line">  \texttt{pnr}: Percentage of the total sequencing depth to be used in each simulated replicate (only when there are not replicates available). For instance, if pnr = 0.2 , each</td>
      </tr>
      <tr>
        <td id="L915" class="blob-num js-line-number" data-line-number="915"></td>
        <td id="LC915" class="blob-code blob-code-inner js-file-line">simulated replicate will have 20\% of the total reads of the only available replicate in that condition.</td>
      </tr>
      <tr>
        <td id="L916" class="blob-num js-line-number" data-line-number="916"></td>
        <td id="LC916" class="blob-code blob-code-inner js-file-line">\item</td>
      </tr>
      <tr>
        <td id="L917" class="blob-num js-line-number" data-line-number="917"></td>
        <td id="LC917" class="blob-code blob-code-inner js-file-line">  \texttt{v}: Variability of the size of each simulated replicate (only used by NOISeq-sim).  </td>
      </tr>
      <tr>
        <td id="L918" class="blob-num js-line-number" data-line-number="918"></td>
        <td id="LC918" class="blob-code blob-code-inner js-file-line">\end{itemize}</td>
      </tr>
      <tr>
        <td id="L919" class="blob-num js-line-number" data-line-number="919"></td>
        <td id="LC919" class="blob-code blob-code-inner js-file-line">
</td>
      </tr>
      <tr>
        <td id="L920" class="blob-num js-line-number" data-line-number="920"></td>
        <td id="LC920" class="blob-code blob-code-inner js-file-line">For example, you can use the following instruction to see the differential expression results for \code{NOISeq}:</td>
      </tr>
      <tr>
        <td id="L921" class="blob-num js-line-number" data-line-number="921"></td>
        <td id="LC921" class="blob-code blob-code-inner js-file-line">
</td>
      </tr>
      <tr>
        <td id="L922" class="blob-num js-line-number" data-line-number="922"></td>
        <td id="LC922" class="blob-code blob-code-inner js-file-line">&lt;&lt;&gt;&gt;=</td>
      </tr>
      <tr>
        <td id="L923" class="blob-num js-line-number" data-line-number="923"></td>
        <td id="LC923" class="blob-code blob-code-inner js-file-line">head(mynoiseq@results[[1]])</td>
      </tr>
      <tr>
        <td id="L924" class="blob-num js-line-number" data-line-number="924"></td>
        <td id="LC924" class="blob-code blob-code-inner js-file-line">@ </td>
      </tr>
      <tr>
        <td id="L925" class="blob-num js-line-number" data-line-number="925"></td>
        <td id="LC925" class="blob-code blob-code-inner js-file-line">
</td>
      </tr>
      <tr>
        <td id="L926" class="blob-num js-line-number" data-line-number="926"></td>
        <td id="LC926" class="blob-code blob-code-inner js-file-line">The output \code{myresults@results[[1]]\$prob} gives the estimated probability of differential expression for each feature. </td>
      </tr>
      <tr>
        <td id="L927" class="blob-num js-line-number" data-line-number="927"></td>
        <td id="LC927" class="blob-code blob-code-inner js-file-line">Note that when using \noiseq{}, these probabilities are not equivalent to p-values. The higher the probability, the more likely that the difference in expression </td>
      </tr>
      <tr>
        <td id="L928" class="blob-num js-line-number" data-line-number="928"></td>
        <td id="LC928" class="blob-code blob-code-inner js-file-line">is due to the change in the experimental condition and not to chance. See Section \ref{sec_deg} to learn how to obtain the differentially expressed features.</td>
      </tr>
      <tr>
        <td id="L929" class="blob-num js-line-number" data-line-number="929"></td>
        <td id="LC929" class="blob-code blob-code-inner js-file-line">
</td>
      </tr>
      <tr>
        <td id="L930" class="blob-num js-line-number" data-line-number="930"></td>
        <td id="LC930" class="blob-code blob-code-inner js-file-line">
</td>
      </tr>
      <tr>
        <td id="L931" class="blob-num js-line-number" data-line-number="931"></td>
        <td id="LC931" class="blob-code blob-code-inner js-file-line">
</td>
      </tr>
      <tr>
        <td id="L932" class="blob-num js-line-number" data-line-number="932"></td>
        <td id="LC932" class="blob-code blob-code-inner js-file-line">
</td>
      </tr>
      <tr>
        <td id="L933" class="blob-num js-line-number" data-line-number="933"></td>
        <td id="LC933" class="blob-code blob-code-inner js-file-line">
</td>
      </tr>
      <tr>
        <td id="L934" class="blob-num js-line-number" data-line-number="934"></td>
        <td id="LC934" class="blob-code blob-code-inner js-file-line">
</td>
      </tr>
      <tr>
        <td id="L935" class="blob-num js-line-number" data-line-number="935"></td>
        <td id="LC935" class="blob-code blob-code-inner js-file-line">\subsubsection{How to select the differentially expressed features} </td>
      </tr>
      <tr>
        <td id="L936" class="blob-num js-line-number" data-line-number="936"></td>
        <td id="LC936" class="blob-code blob-code-inner js-file-line">
</td>
      </tr>
      <tr>
        <td id="L937" class="blob-num js-line-number" data-line-number="937"></td>
        <td id="LC937" class="blob-code blob-code-inner js-file-line">Once we have obtained the differential expression probability for each one of the features by using \code{NOISeq} or \code{NOISeqBIO} function, </td>
      </tr>
      <tr>
        <td id="L938" class="blob-num js-line-number" data-line-number="938"></td>
        <td id="LC938" class="blob-code blob-code-inner js-file-line">we may want to select the differentially expressed features for a given threshold $q$. This can be done with \code{degenes} function </td>
      </tr>
      <tr>
        <td id="L939" class="blob-num js-line-number" data-line-number="939"></td>
        <td id="LC939" class="blob-code blob-code-inner js-file-line">on the ``output&quot; object using the parameter \code{q}. With the argument \code{M} we choose if we want all the differentially expressed features, </td>
      </tr>
      <tr>
        <td id="L940" class="blob-num js-line-number" data-line-number="940"></td>
        <td id="LC940" class="blob-code blob-code-inner js-file-line">only the differentially expressed features that are more expressed in condition 1 than in condition 2 (M = ``up&quot;) </td>
      </tr>
      <tr>
        <td id="L941" class="blob-num js-line-number" data-line-number="941"></td>
        <td id="LC941" class="blob-code blob-code-inner js-file-line">or only the differentially expressed features that are under-expressed in condition 1 with regard to condition 2 (M = ``down&quot;):</td>
      </tr>
      <tr>
        <td id="L942" class="blob-num js-line-number" data-line-number="942"></td>
        <td id="LC942" class="blob-code blob-code-inner js-file-line">
</td>
      </tr>
      <tr>
        <td id="L943" class="blob-num js-line-number" data-line-number="943"></td>
        <td id="LC943" class="blob-code blob-code-inner js-file-line">&lt;&lt;&gt;&gt;=</td>
      </tr>
      <tr>
        <td id="L944" class="blob-num js-line-number" data-line-number="944"></td>
        <td id="LC944" class="blob-code blob-code-inner js-file-line">mynoiseq.deg = degenes(mynoiseq, q = 0.8, M = NULL)</td>
      </tr>
      <tr>
        <td id="L945" class="blob-num js-line-number" data-line-number="945"></td>
        <td id="LC945" class="blob-code blob-code-inner js-file-line">mynoiseq.deg1 = degenes(mynoiseq, q = 0.8, M = &quot;up&quot;)</td>
      </tr>
      <tr>
        <td id="L946" class="blob-num js-line-number" data-line-number="946"></td>
        <td id="LC946" class="blob-code blob-code-inner js-file-line">mynoiseq.deg2 = degenes(mynoiseq, q = 0.8, M = &quot;down&quot;)</td>
      </tr>
      <tr>
        <td id="L947" class="blob-num js-line-number" data-line-number="947"></td>
        <td id="LC947" class="blob-code blob-code-inner js-file-line">@ </td>
      </tr>
      <tr>
        <td id="L948" class="blob-num js-line-number" data-line-number="948"></td>
        <td id="LC948" class="blob-code blob-code-inner js-file-line">
</td>
      </tr>
      <tr>
        <td id="L949" class="blob-num js-line-number" data-line-number="949"></td>
        <td id="LC949" class="blob-code blob-code-inner js-file-line">Please remember that, when using \code{NOISeq}, the probability of differential expression is not equivalent to $1-pvalue$. </td>
      </tr>
      <tr>
        <td id="L950" class="blob-num js-line-number" data-line-number="950"></td>
        <td id="LC950" class="blob-code blob-code-inner js-file-line">We recommend for $q$ to use values around $0.8$. If \code{NOISeq-sim} has been used because no replicates are available, </td>
      </tr>
      <tr>
        <td id="L951" class="blob-num js-line-number" data-line-number="951"></td>
        <td id="LC951" class="blob-code blob-code-inner js-file-line">then it is preferable to use a higher threshold such as $q=0.9$. However, when using \code{NOISeqBIO}, the</td>
      </tr>
      <tr>
        <td id="L952" class="blob-num js-line-number" data-line-number="952"></td>
        <td id="LC952" class="blob-code blob-code-inner js-file-line">probability of differential expression would be equivalent to $1-FDR$, where $FDR$ can be considered as an adjusted p-value. </td>
      </tr>
      <tr>
        <td id="L953" class="blob-num js-line-number" data-line-number="953"></td>
        <td id="LC953" class="blob-code blob-code-inner js-file-line">Hence, in this case, it would be more convenient to use $q=0.95$.</td>
      </tr>
      <tr>
        <td id="L954" class="blob-num js-line-number" data-line-number="954"></td>
        <td id="LC954" class="blob-code blob-code-inner js-file-line">
</td>
      </tr>
      <tr>
        <td id="L955" class="blob-num js-line-number" data-line-number="955"></td>
        <td id="LC955" class="blob-code blob-code-inner js-file-line">
</td>
      </tr>
      <tr>
        <td id="L956" class="blob-num js-line-number" data-line-number="956"></td>
        <td id="LC956" class="blob-code blob-code-inner js-file-line">
</td>
      </tr>
      <tr>
        <td id="L957" class="blob-num js-line-number" data-line-number="957"></td>
        <td id="LC957" class="blob-code blob-code-inner js-file-line">
</td>
      </tr>
      <tr>
        <td id="L958" class="blob-num js-line-number" data-line-number="958"></td>
        <td id="LC958" class="blob-code blob-code-inner js-file-line">\subsubsection{Plots on differential expression results}</td>
      </tr>
      <tr>
        <td id="L959" class="blob-num js-line-number" data-line-number="959"></td>
        <td id="LC959" class="blob-code blob-code-inner js-file-line">
</td>
      </tr>
      <tr>
        <td id="L960" class="blob-num js-line-number" data-line-number="960"></td>
        <td id="LC960" class="blob-code blob-code-inner js-file-line">\textbf{Expression plot}</td>
      </tr>
      <tr>
        <td id="L961" class="blob-num js-line-number" data-line-number="961"></td>
        <td id="LC961" class="blob-code blob-code-inner js-file-line">
</td>
      </tr>
      <tr>
        <td id="L962" class="blob-num js-line-number" data-line-number="962"></td>
        <td id="LC962" class="blob-code blob-code-inner js-file-line">Once differential expression has been computed, it is interesting to plot the average expression values of each condition and highlight the features declared as differentially expressed.</td>
      </tr>
      <tr>
        <td id="L963" class="blob-num js-line-number" data-line-number="963"></td>
        <td id="LC963" class="blob-code blob-code-inner js-file-line">It can be done with the \code{DE.plot}. </td>
      </tr>
      <tr>
        <td id="L964" class="blob-num js-line-number" data-line-number="964"></td>
        <td id="LC964" class="blob-code blob-code-inner js-file-line">
</td>
      </tr>
      <tr>
        <td id="L965" class="blob-num js-line-number" data-line-number="965"></td>
        <td id="LC965" class="blob-code blob-code-inner js-file-line">To plot the summary of the expression values in both conditions as in Fig. \ref{fig_summ_expr}, please write the following code </td>
      </tr>
      <tr>
        <td id="L966" class="blob-num js-line-number" data-line-number="966"></td>
        <td id="LC966" class="blob-code blob-code-inner js-file-line">(many graphical parameters can be adjusted, see the function help). Note that by giving $q=0.9$, differentially expressed features </td>
      </tr>
      <tr>
        <td id="L967" class="blob-num js-line-number" data-line-number="967"></td>
        <td id="LC967" class="blob-code blob-code-inner js-file-line">considering this threshold will be highlighted in red:</td>
      </tr>
      <tr>
        <td id="L968" class="blob-num js-line-number" data-line-number="968"></td>
        <td id="LC968" class="blob-code blob-code-inner js-file-line">&lt;&lt;fig_summ_expr,fig=TRUE&gt;&gt;=</td>
      </tr>
      <tr>
        <td id="L969" class="blob-num js-line-number" data-line-number="969"></td>
        <td id="LC969" class="blob-code blob-code-inner js-file-line">DE.plot(mynoiseq, q = 0.9, graphic = &quot;expr&quot;, log.scale = TRUE)</td>
      </tr>
      <tr>
        <td id="L970" class="blob-num js-line-number" data-line-number="970"></td>
        <td id="LC970" class="blob-code blob-code-inner js-file-line">@ </td>
      </tr>
      <tr>
        <td id="L971" class="blob-num js-line-number" data-line-number="971"></td>
        <td id="LC971" class="blob-code blob-code-inner js-file-line">
</td>
      </tr>
      <tr>
        <td id="L972" class="blob-num js-line-number" data-line-number="972"></td>
        <td id="LC972" class="blob-code blob-code-inner js-file-line">\begin{figure}[ht!]</td>
      </tr>
      <tr>
        <td id="L973" class="blob-num js-line-number" data-line-number="973"></td>
        <td id="LC973" class="blob-code blob-code-inner js-file-line">\centering</td>
      </tr>
      <tr>
        <td id="L974" class="blob-num js-line-number" data-line-number="974"></td>
        <td id="LC974" class="blob-code blob-code-inner js-file-line">\includegraphics[width=0.6\textwidth]{NOISeq-fig_summ_expr}</td>
      </tr>
      <tr>
        <td id="L975" class="blob-num js-line-number" data-line-number="975"></td>
        <td id="LC975" class="blob-code blob-code-inner js-file-line">\caption{Summary plot of the expression values for both conditions (black), where differentially expressed genes are highlighted (red).}</td>
      </tr>
      <tr>
        <td id="L976" class="blob-num js-line-number" data-line-number="976"></td>
        <td id="LC976" class="blob-code blob-code-inner js-file-line">\label{fig_summ_expr}</td>
      </tr>
      <tr>
        <td id="L977" class="blob-num js-line-number" data-line-number="977"></td>
        <td id="LC977" class="blob-code blob-code-inner js-file-line">\end{figure}</td>
      </tr>
      <tr>
        <td id="L978" class="blob-num js-line-number" data-line-number="978"></td>
        <td id="LC978" class="blob-code blob-code-inner js-file-line">
</td>
      </tr>
      <tr>
        <td id="L979" class="blob-num js-line-number" data-line-number="979"></td>
        <td id="LC979" class="blob-code blob-code-inner js-file-line">
</td>
      </tr>
      <tr>
        <td id="L980" class="blob-num js-line-number" data-line-number="980"></td>
        <td id="LC980" class="blob-code blob-code-inner js-file-line">\textbf{MD plot}</td>
      </tr>
      <tr>
        <td id="L981" class="blob-num js-line-number" data-line-number="981"></td>
        <td id="LC981" class="blob-code blob-code-inner js-file-line">
</td>
      </tr>
      <tr>
        <td id="L982" class="blob-num js-line-number" data-line-number="982"></td>
        <td id="LC982" class="blob-code blob-code-inner js-file-line">Instead of plotting the expression values, it is also interesting to plot the log-fold change ($M$) and the absolute value of the difference in expression between conditions ($D$) as in Fig. \ref{fig_summ_MD}. </td>
      </tr>
      <tr>
        <td id="L983" class="blob-num js-line-number" data-line-number="983"></td>
        <td id="LC983" class="blob-code blob-code-inner js-file-line">This is an example of the code to get such a plot ($D$ values are displayed in log-scale) from \code{NOISeq} output (it is analogous for \code{NOISeqBIO} ouput). </td>
      </tr>
      <tr>
        <td id="L984" class="blob-num js-line-number" data-line-number="984"></td>
        <td id="LC984" class="blob-code blob-code-inner js-file-line">&lt;&lt;fig_summ_MD,fig=TRUE&gt;&gt;=</td>
      </tr>
      <tr>
        <td id="L985" class="blob-num js-line-number" data-line-number="985"></td>
        <td id="LC985" class="blob-code blob-code-inner js-file-line">DE.plot(mynoiseq, q = 0.8, graphic = &quot;MD&quot;)</td>
      </tr>
      <tr>
        <td id="L986" class="blob-num js-line-number" data-line-number="986"></td>
        <td id="LC986" class="blob-code blob-code-inner js-file-line">@ </td>
      </tr>
      <tr>
        <td id="L987" class="blob-num js-line-number" data-line-number="987"></td>
        <td id="LC987" class="blob-code blob-code-inner js-file-line">
</td>
      </tr>
      <tr>
        <td id="L988" class="blob-num js-line-number" data-line-number="988"></td>
        <td id="LC988" class="blob-code blob-code-inner js-file-line">\begin{figure}[ht!]</td>
      </tr>
      <tr>
        <td id="L989" class="blob-num js-line-number" data-line-number="989"></td>
        <td id="LC989" class="blob-code blob-code-inner js-file-line">\centering</td>
      </tr>
      <tr>
        <td id="L990" class="blob-num js-line-number" data-line-number="990"></td>
        <td id="LC990" class="blob-code blob-code-inner js-file-line">\includegraphics[width=0.6\textwidth]{NOISeq-fig_summ_MD}</td>
      </tr>
      <tr>
        <td id="L991" class="blob-num js-line-number" data-line-number="991"></td>
        <td id="LC991" class="blob-code blob-code-inner js-file-line">\caption{Summary plot for (M,D) values (black) and the differentially expressed genes (red).}</td>
      </tr>
      <tr>
        <td id="L992" class="blob-num js-line-number" data-line-number="992"></td>
        <td id="LC992" class="blob-code blob-code-inner js-file-line">\label{fig_summ_MD}</td>
      </tr>
      <tr>
        <td id="L993" class="blob-num js-line-number" data-line-number="993"></td>
        <td id="LC993" class="blob-code blob-code-inner js-file-line">\end{figure}</td>
      </tr>
      <tr>
        <td id="L994" class="blob-num js-line-number" data-line-number="994"></td>
        <td id="LC994" class="blob-code blob-code-inner js-file-line">
</td>
      </tr>
      <tr>
        <td id="L995" class="blob-num js-line-number" data-line-number="995"></td>
        <td id="LC995" class="blob-code blob-code-inner js-file-line">
</td>
      </tr>
      <tr>
        <td id="L996" class="blob-num js-line-number" data-line-number="996"></td>
        <td id="LC996" class="blob-code blob-code-inner js-file-line">\textbf{Manhattan plot}</td>
      </tr>
      <tr>
        <td id="L997" class="blob-num js-line-number" data-line-number="997"></td>
        <td id="LC997" class="blob-code blob-code-inner js-file-line">
</td>
      </tr>
      <tr>
        <td id="L998" class="blob-num js-line-number" data-line-number="998"></td>
        <td id="LC998" class="blob-code blob-code-inner js-file-line">The Manhattan plot can be used to display the expression of the genes across the chromosomes. </td>
      </tr>
      <tr>
        <td id="L999" class="blob-num js-line-number" data-line-number="999"></td>
        <td id="LC999" class="blob-code blob-code-inner js-file-line">The expression for both conditions under comparison is shown in the plot. </td>
      </tr>
      <tr>
        <td id="L1000" class="blob-num js-line-number" data-line-number="1000"></td>
        <td id="LC1000" class="blob-code blob-code-inner js-file-line">The users may choose either plotting all the chromosomes or only some of them, </td>
      </tr>
      <tr>
        <td id="L1001" class="blob-num js-line-number" data-line-number="1001"></td>
        <td id="LC1001" class="blob-code blob-code-inner js-file-line">and also if the chromosomes are depicted consecutively (useful for prokaryote organisms) </td>
      </tr>
      <tr>
        <td id="L1002" class="blob-num js-line-number" data-line-number="1002"></td>
        <td id="LC1002" class="blob-code blob-code-inner js-file-line">or separately (one per line).</td>
      </tr>
      <tr>
        <td id="L1003" class="blob-num js-line-number" data-line-number="1003"></td>
        <td id="LC1003" class="blob-code blob-code-inner js-file-line">If a $q$ cutoff is provided, then differentially expressed features are highlighted in a different color.</td>
      </tr>
      <tr>
        <td id="L1004" class="blob-num js-line-number" data-line-number="1004"></td>
        <td id="LC1004" class="blob-code blob-code-inner js-file-line">The following code shows how to draw the Manhattan plot from the output object returned by \code{NOISeq} or \code{NOISeqBIO}. </td>
      </tr>
      <tr>
        <td id="L1005" class="blob-num js-line-number" data-line-number="1005"></td>
        <td id="LC1005" class="blob-code blob-code-inner js-file-line">In this case, using Marioni&#39;s data, the expression (log-transformed) is represented for two chromosomes (see Fig. \ref{fig_manhattan}). </td>
      </tr>
      <tr>
        <td id="L1006" class="blob-num js-line-number" data-line-number="1006"></td>
        <td id="LC1006" class="blob-code blob-code-inner js-file-line">Note that the chromosomes will be depicted in the same order that are given to ``chromosomes&quot; parameter. </td>
      </tr>
      <tr>
        <td id="L1007" class="blob-num js-line-number" data-line-number="1007"></td>
        <td id="LC1007" class="blob-code blob-code-inner js-file-line">
</td>
      </tr>
      <tr>
        <td id="L1008" class="blob-num js-line-number" data-line-number="1008"></td>
        <td id="LC1008" class="blob-code blob-code-inner js-file-line">Gene expression is represented in gray. Lines above 0 correspond to the first condition under comparison (kidney) </td>
      </tr>
      <tr>
        <td id="L1009" class="blob-num js-line-number" data-line-number="1009"></td>
        <td id="LC1009" class="blob-code blob-code-inner js-file-line">and lines below 0 are for the second condition (liver). Genes up-regulated in the first condition are highlighted in red, </td>
      </tr>
      <tr>
        <td id="L1010" class="blob-num js-line-number" data-line-number="1010"></td>
        <td id="LC1010" class="blob-code blob-code-inner js-file-line">while genes up-regulated in the second condition are highlighted in green. The blue lines on the horizontal axis (Y=0)</td>
      </tr>
      <tr>
        <td id="L1011" class="blob-num js-line-number" data-line-number="1011"></td>
        <td id="LC1011" class="blob-code blob-code-inner js-file-line">correspond to the annotated genes. X scale shows the location in the chromosome.</td>
      </tr>
      <tr>
        <td id="L1012" class="blob-num js-line-number" data-line-number="1012"></td>
        <td id="LC1012" class="blob-code blob-code-inner js-file-line">&lt;&lt;fig_manhattan,fig=TRUE,width=12,height=5&gt;&gt;=</td>
      </tr>
      <tr>
        <td id="L1013" class="blob-num js-line-number" data-line-number="1013"></td>
        <td id="LC1013" class="blob-code blob-code-inner js-file-line">DE.plot(mynoiseq, chromosomes = c(1,2), log.scale = TRUE,</td>
      </tr>
      <tr>
        <td id="L1014" class="blob-num js-line-number" data-line-number="1014"></td>
        <td id="LC1014" class="blob-code blob-code-inner js-file-line">        join = FALSE, q = 0.8, graphic = &quot;chrom&quot;)</td>
      </tr>
      <tr>
        <td id="L1015" class="blob-num js-line-number" data-line-number="1015"></td>
        <td id="LC1015" class="blob-code blob-code-inner js-file-line">@ </td>
      </tr>
      <tr>
        <td id="L1016" class="blob-num js-line-number" data-line-number="1016"></td>
        <td id="LC1016" class="blob-code blob-code-inner js-file-line">
</td>
      </tr>
      <tr>
        <td id="L1017" class="blob-num js-line-number" data-line-number="1017"></td>
        <td id="LC1017" class="blob-code blob-code-inner js-file-line">
</td>
      </tr>
      <tr>
        <td id="L1018" class="blob-num js-line-number" data-line-number="1018"></td>
        <td id="LC1018" class="blob-code blob-code-inner js-file-line">\begin{figure}[ht!]</td>
      </tr>
      <tr>
        <td id="L1019" class="blob-num js-line-number" data-line-number="1019"></td>
        <td id="LC1019" class="blob-code blob-code-inner js-file-line">\centering</td>
      </tr>
      <tr>
        <td id="L1020" class="blob-num js-line-number" data-line-number="1020"></td>
        <td id="LC1020" class="blob-code blob-code-inner js-file-line">\includegraphics[width=\textwidth]{NOISeq-fig_manhattan}</td>
      </tr>
      <tr>
        <td id="L1021" class="blob-num js-line-number" data-line-number="1021"></td>
        <td id="LC1021" class="blob-code blob-code-inner js-file-line">\caption{Manhattan plot for chromosomes 1 and 2}</td>
      </tr>
      <tr>
        <td id="L1022" class="blob-num js-line-number" data-line-number="1022"></td>
        <td id="LC1022" class="blob-code blob-code-inner js-file-line">\label{fig_manhattan}</td>
      </tr>
      <tr>
        <td id="L1023" class="blob-num js-line-number" data-line-number="1023"></td>
        <td id="LC1023" class="blob-code blob-code-inner js-file-line">\end{figure}</td>
      </tr>
      <tr>
        <td id="L1024" class="blob-num js-line-number" data-line-number="1024"></td>
        <td id="LC1024" class="blob-code blob-code-inner js-file-line">
</td>
      </tr>
      <tr>
        <td id="L1025" class="blob-num js-line-number" data-line-number="1025"></td>
        <td id="LC1025" class="blob-code blob-code-inner js-file-line">It is advisable, in this kind of plots, to save the figure in a file, for instance, a pdf file (as in the following code), </td>
      </tr>
      <tr>
        <td id="L1026" class="blob-num js-line-number" data-line-number="1026"></td>
        <td id="LC1026" class="blob-code blob-code-inner js-file-line">in order to get a better visualization with the zoom.</td>
      </tr>
      <tr>
        <td id="L1027" class="blob-num js-line-number" data-line-number="1027"></td>
        <td id="LC1027" class="blob-code blob-code-inner js-file-line">
</td>
      </tr>
      <tr>
        <td id="L1028" class="blob-num js-line-number" data-line-number="1028"></td>
        <td id="LC1028" class="blob-code blob-code-inner js-file-line">\begin{Schunk}</td>
      </tr>
      <tr>
        <td id="L1029" class="blob-num js-line-number" data-line-number="1029"></td>
        <td id="LC1029" class="blob-code blob-code-inner js-file-line">\begin{Sinput}</td>
      </tr>
      <tr>
        <td id="L1030" class="blob-num js-line-number" data-line-number="1030"></td>
        <td id="LC1030" class="blob-code blob-code-inner js-file-line">pdf(&quot;manhattan.pdf&quot;, width = 12, height = 50)</td>
      </tr>
      <tr>
        <td id="L1031" class="blob-num js-line-number" data-line-number="1031"></td>
        <td id="LC1031" class="blob-code blob-code-inner js-file-line">DE.plot(mynoiseq, chromosomes = c(1,2), log.scale = TRUE,</td>
      </tr>
      <tr>
        <td id="L1032" class="blob-num js-line-number" data-line-number="1032"></td>
        <td id="LC1032" class="blob-code blob-code-inner js-file-line">        join = FALSE, q = 0.8)</td>
      </tr>
      <tr>
        <td id="L1033" class="blob-num js-line-number" data-line-number="1033"></td>
        <td id="LC1033" class="blob-code blob-code-inner js-file-line">dev.off()</td>
      </tr>
      <tr>
        <td id="L1034" class="blob-num js-line-number" data-line-number="1034"></td>
        <td id="LC1034" class="blob-code blob-code-inner js-file-line">\end{Sinput}</td>
      </tr>
      <tr>
        <td id="L1035" class="blob-num js-line-number" data-line-number="1035"></td>
        <td id="LC1035" class="blob-code blob-code-inner js-file-line">\end{Schunk}</td>
      </tr>
      <tr>
        <td id="L1036" class="blob-num js-line-number" data-line-number="1036"></td>
        <td id="LC1036" class="blob-code blob-code-inner js-file-line">
</td>
      </tr>
      <tr>
        <td id="L1037" class="blob-num js-line-number" data-line-number="1037"></td>
        <td id="LC1037" class="blob-code blob-code-inner js-file-line">
</td>
      </tr>
      <tr>
        <td id="L1038" class="blob-num js-line-number" data-line-number="1038"></td>
        <td id="LC1038" class="blob-code blob-code-inner js-file-line">
</td>
      </tr>
      <tr>
        <td id="L1039" class="blob-num js-line-number" data-line-number="1039"></td>
        <td id="LC1039" class="blob-code blob-code-inner js-file-line">
</td>
      </tr>
      <tr>
        <td id="L1040" class="blob-num js-line-number" data-line-number="1040"></td>
        <td id="LC1040" class="blob-code blob-code-inner js-file-line">\textbf{Distribution of differentially expressed features per chromosomes or biotypes}</td>
      </tr>
      <tr>
        <td id="L1041" class="blob-num js-line-number" data-line-number="1041"></td>
        <td id="LC1041" class="blob-code blob-code-inner js-file-line">
</td>
      </tr>
      <tr>
        <td id="L1042" class="blob-num js-line-number" data-line-number="1042"></td>
        <td id="LC1042" class="blob-code blob-code-inner js-file-line">This function creates a figure with two plots if both chromosomes and biotypes information is provided. </td>
      </tr>
      <tr>
        <td id="L1043" class="blob-num js-line-number" data-line-number="1043"></td>
        <td id="LC1043" class="blob-code blob-code-inner js-file-line">Otherwise, only a plot is depicted with either the chromosomes or biotypes (if information of any of them is available). The $q$ cutoff must be provided.</td>
      </tr>
      <tr>
        <td id="L1044" class="blob-num js-line-number" data-line-number="1044"></td>
        <td id="LC1044" class="blob-code blob-code-inner js-file-line">Both plots are analogous. The chromosomes plot shows the percentage of features in each chromosome, </td>
      </tr>
      <tr>
        <td id="L1045" class="blob-num js-line-number" data-line-number="1045"></td>
        <td id="LC1045" class="blob-code blob-code-inner js-file-line">the proportion of them that are differentially expressed (DEG) and the percentage of differentially expressed features in each chromosome. </td>
      </tr>
      <tr>
        <td id="L1046" class="blob-num js-line-number" data-line-number="1046"></td>
        <td id="LC1046" class="blob-code blob-code-inner js-file-line">Users may choose plotting all the chromosomes or only some of them. The chromosomes are depicted according to the number of features they contain </td>
      </tr>
      <tr>
        <td id="L1047" class="blob-num js-line-number" data-line-number="1047"></td>
        <td id="LC1047" class="blob-code blob-code-inner js-file-line">(from the greatest to the lowest). The plot for biotypes can be described similarly. The only difference is that this plot has a left axis scale</td>
      </tr>
      <tr>
        <td id="L1048" class="blob-num js-line-number" data-line-number="1048"></td>
        <td id="LC1048" class="blob-code blob-code-inner js-file-line">for the most abundant biotypes and a right axis scale for the rest of biotypes, which are separated by a green vertical line.</td>
      </tr>
      <tr>
        <td id="L1049" class="blob-num js-line-number" data-line-number="1049"></td>
        <td id="LC1049" class="blob-code blob-code-inner js-file-line">
</td>
      </tr>
      <tr>
        <td id="L1050" class="blob-num js-line-number" data-line-number="1050"></td>
        <td id="LC1050" class="blob-code blob-code-inner js-file-line">The following code shows how to draw the figure from the output object returned by \code{NOISeq} for the Marioni&#39;s example data. </td>
      </tr>
      <tr>
        <td id="L1051" class="blob-num js-line-number" data-line-number="1051"></td>
        <td id="LC1051" class="blob-code blob-code-inner js-file-line">
</td>
      </tr>
      <tr>
        <td id="L1052" class="blob-num js-line-number" data-line-number="1052"></td>
        <td id="LC1052" class="blob-code blob-code-inner js-file-line">&lt;&lt;fig_distrDEG,fig=TRUE,width=12,height=7&gt;&gt;=</td>
      </tr>
      <tr>
        <td id="L1053" class="blob-num js-line-number" data-line-number="1053"></td>
        <td id="LC1053" class="blob-code blob-code-inner js-file-line">DE.plot(mynoiseq, chromosomes = NULL, q = 0.8, graphic = &quot;distr&quot;)</td>
      </tr>
      <tr>
        <td id="L1054" class="blob-num js-line-number" data-line-number="1054"></td>
        <td id="LC1054" class="blob-code blob-code-inner js-file-line">@ </td>
      </tr>
      <tr>
        <td id="L1055" class="blob-num js-line-number" data-line-number="1055"></td>
        <td id="LC1055" class="blob-code blob-code-inner js-file-line">
</td>
      </tr>
      <tr>
        <td id="L1056" class="blob-num js-line-number" data-line-number="1056"></td>
        <td id="LC1056" class="blob-code blob-code-inner js-file-line">
</td>
      </tr>
      <tr>
        <td id="L1057" class="blob-num js-line-number" data-line-number="1057"></td>
        <td id="LC1057" class="blob-code blob-code-inner js-file-line">\begin{figure}[ht!]</td>
      </tr>
      <tr>
        <td id="L1058" class="blob-num js-line-number" data-line-number="1058"></td>
        <td id="LC1058" class="blob-code blob-code-inner js-file-line">\centering</td>
      </tr>
      <tr>
        <td id="L1059" class="blob-num js-line-number" data-line-number="1059"></td>
        <td id="LC1059" class="blob-code blob-code-inner js-file-line">\includegraphics[width=\textwidth]{NOISeq-fig_distrDEG}</td>
      </tr>
      <tr>
        <td id="L1060" class="blob-num js-line-number" data-line-number="1060"></td>
        <td id="LC1060" class="blob-code blob-code-inner js-file-line">\caption{Distribution of DEG across chromosomes and biotypes for Marioni&#39;s example dataset.}</td>
      </tr>
      <tr>
        <td id="L1061" class="blob-num js-line-number" data-line-number="1061"></td>
        <td id="LC1061" class="blob-code blob-code-inner js-file-line">\label{fig_distrDEG}</td>
      </tr>
      <tr>
        <td id="L1062" class="blob-num js-line-number" data-line-number="1062"></td>
        <td id="LC1062" class="blob-code blob-code-inner js-file-line">\end{figure}</td>
      </tr>
      <tr>
        <td id="L1063" class="blob-num js-line-number" data-line-number="1063"></td>
        <td id="LC1063" class="blob-code blob-code-inner js-file-line">
</td>
      </tr>
      <tr>
        <td id="L1064" class="blob-num js-line-number" data-line-number="1064"></td>
        <td id="LC1064" class="blob-code blob-code-inner js-file-line">
</td>
      </tr>
      <tr>
        <td id="L1065" class="blob-num js-line-number" data-line-number="1065"></td>
        <td id="LC1065" class="blob-code blob-code-inner js-file-line">
</td>
      </tr>
      <tr>
        <td id="L1066" class="blob-num js-line-number" data-line-number="1066"></td>
        <td id="LC1066" class="blob-code blob-code-inner js-file-line">
</td>
      </tr>
      <tr>
        <td id="L1067" class="blob-num js-line-number" data-line-number="1067"></td>
        <td id="LC1067" class="blob-code blob-code-inner js-file-line">\vspace{1cm}</td>
      </tr>
      <tr>
        <td id="L1068" class="blob-num js-line-number" data-line-number="1068"></td>
        <td id="LC1068" class="blob-code blob-code-inner js-file-line">
</td>
      </tr>
      <tr>
        <td id="L1069" class="blob-num js-line-number" data-line-number="1069"></td>
        <td id="LC1069" class="blob-code blob-code-inner js-file-line">%\clearpage</td>
      </tr>
      <tr>
        <td id="L1070" class="blob-num js-line-number" data-line-number="1070"></td>
        <td id="LC1070" class="blob-code blob-code-inner js-file-line">
</td>
      </tr>
      <tr>
        <td id="L1071" class="blob-num js-line-number" data-line-number="1071"></td>
        <td id="LC1071" class="blob-code blob-code-inner js-file-line">\section{Setup}</td>
      </tr>
      <tr>
        <td id="L1072" class="blob-num js-line-number" data-line-number="1072"></td>
        <td id="LC1072" class="blob-code blob-code-inner js-file-line">
</td>
      </tr>
      <tr>
        <td id="L1073" class="blob-num js-line-number" data-line-number="1073"></td>
        <td id="LC1073" class="blob-code blob-code-inner js-file-line">This vignette was built on:</td>
      </tr>
      <tr>
        <td id="L1074" class="blob-num js-line-number" data-line-number="1074"></td>
        <td id="LC1074" class="blob-code blob-code-inner js-file-line">&lt;&lt;session&gt;&gt;=</td>
      </tr>
      <tr>
        <td id="L1075" class="blob-num js-line-number" data-line-number="1075"></td>
        <td id="LC1075" class="blob-code blob-code-inner js-file-line">sessionInfo()</td>
      </tr>
      <tr>
        <td id="L1076" class="blob-num js-line-number" data-line-number="1076"></td>
        <td id="LC1076" class="blob-code blob-code-inner js-file-line">@ </td>
      </tr>
      <tr>
        <td id="L1077" class="blob-num js-line-number" data-line-number="1077"></td>
        <td id="LC1077" class="blob-code blob-code-inner js-file-line">
</td>
      </tr>
      <tr>
        <td id="L1078" class="blob-num js-line-number" data-line-number="1078"></td>
        <td id="LC1078" class="blob-code blob-code-inner js-file-line">
</td>
      </tr>
      <tr>
        <td id="L1079" class="blob-num js-line-number" data-line-number="1079"></td>
        <td id="LC1079" class="blob-code blob-code-inner js-file-line">%%%%%%%%%%%%%%%%%%%%%%%%%%%%%%%%%%%%%%%%%%%%%%%%%%%%%%%%%%%%%%%%%%%%%%%%%%%</td>
      </tr>
      <tr>
        <td id="L1080" class="blob-num js-line-number" data-line-number="1080"></td>
        <td id="LC1080" class="blob-code blob-code-inner js-file-line">\vspace{2cm}</td>
      </tr>
      <tr>
        <td id="L1081" class="blob-num js-line-number" data-line-number="1081"></td>
        <td id="LC1081" class="blob-code blob-code-inner js-file-line">
</td>
      </tr>
      <tr>
        <td id="L1082" class="blob-num js-line-number" data-line-number="1082"></td>
        <td id="LC1082" class="blob-code blob-code-inner js-file-line">
</td>
      </tr>
      <tr>
        <td id="L1083" class="blob-num js-line-number" data-line-number="1083"></td>
        <td id="LC1083" class="blob-code blob-code-inner js-file-line">\begin{thebibliography}{9}</td>
      </tr>
      <tr>
        <td id="L1084" class="blob-num js-line-number" data-line-number="1084"></td>
        <td id="LC1084" class="blob-code blob-code-inner js-file-line">% \providecommand{\natexlab}[1]{#1}</td>
      </tr>
      <tr>
        <td id="L1085" class="blob-num js-line-number" data-line-number="1085"></td>
        <td id="LC1085" class="blob-code blob-code-inner js-file-line">% \providecommand{\url}[1]{\texttt{#1}}</td>
      </tr>
      <tr>
        <td id="L1086" class="blob-num js-line-number" data-line-number="1086"></td>
        <td id="LC1086" class="blob-code blob-code-inner js-file-line">% \expandafter\ifx\csname urlstyle\endcsname\relax</td>
      </tr>
      <tr>
        <td id="L1087" class="blob-num js-line-number" data-line-number="1087"></td>
        <td id="LC1087" class="blob-code blob-code-inner js-file-line">%   \providecommand{\doi}[1]{doi: #1}\else</td>
      </tr>
      <tr>
        <td id="L1088" class="blob-num js-line-number" data-line-number="1088"></td>
        <td id="LC1088" class="blob-code blob-code-inner js-file-line">%   \providecommand{\doi}{doi: \begingroup \urlstyle{rm}\Url}\fi</td>
      </tr>
      <tr>
        <td id="L1089" class="blob-num js-line-number" data-line-number="1089"></td>
        <td id="LC1089" class="blob-code blob-code-inner js-file-line">
</td>
      </tr>
      <tr>
        <td id="L1090" class="blob-num js-line-number" data-line-number="1090"></td>
        <td id="LC1090" class="blob-code blob-code-inner js-file-line">\bibitem{tarazona2011}</td>
      </tr>
      <tr>
        <td id="L1091" class="blob-num js-line-number" data-line-number="1091"></td>
        <td id="LC1091" class="blob-code blob-code-inner js-file-line">S. Tarazona, F. Garc\&#39;{\i}a-Alcalde, J. Dopazo, A. Ferrer, and A. Conesa.</td>
      </tr>
      <tr>
        <td id="L1092" class="blob-num js-line-number" data-line-number="1092"></td>
        <td id="LC1092" class="blob-code blob-code-inner js-file-line">\newblock {Differential expression in RNA-seq: A matter of depth}.</td>
      </tr>
      <tr>
        <td id="L1093" class="blob-num js-line-number" data-line-number="1093"></td>
        <td id="LC1093" class="blob-code blob-code-inner js-file-line">\newblock \emph{Genome Research}, 21: 2213 - 2223, 2011.</td>
      </tr>
      <tr>
        <td id="L1094" class="blob-num js-line-number" data-line-number="1094"></td>
        <td id="LC1094" class="blob-code blob-code-inner js-file-line">
</td>
      </tr>
      <tr>
        <td id="L1095" class="blob-num js-line-number" data-line-number="1095"></td>
        <td id="LC1095" class="blob-code blob-code-inner js-file-line">\bibitem{tarazona2015}</td>
      </tr>
      <tr>
        <td id="L1096" class="blob-num js-line-number" data-line-number="1096"></td>
        <td id="LC1096" class="blob-code blob-code-inner js-file-line">S. Tarazona, P. Furi\&#39;{o}-Tar\&#39;{i}, D. Turr\&#39;{a}, A. Di Pietro, M.J. Nueda, A. Ferrer, and A. Conesa.</td>
      </tr>
      <tr>
        <td id="L1097" class="blob-num js-line-number" data-line-number="1097"></td>
        <td id="LC1097" class="blob-code blob-code-inner js-file-line">\newblock {Data quality aware analysis of differential expression in RNA-seq with NOISeq R/Bioc package}.</td>
      </tr>
      <tr>
        <td id="L1098" class="blob-num js-line-number" data-line-number="1098"></td>
        <td id="LC1098" class="blob-code blob-code-inner js-file-line">\newblock \emph{Nucleic Acids Research}, 43(21):e140, 2015.</td>
      </tr>
      <tr>
        <td id="L1099" class="blob-num js-line-number" data-line-number="1099"></td>
        <td id="LC1099" class="blob-code blob-code-inner js-file-line">
</td>
      </tr>
      <tr>
        <td id="L1100" class="blob-num js-line-number" data-line-number="1100"></td>
        <td id="LC1100" class="blob-code blob-code-inner js-file-line">\bibitem{marioni2008}</td>
      </tr>
      <tr>
        <td id="L1101" class="blob-num js-line-number" data-line-number="1101"></td>
        <td id="LC1101" class="blob-code blob-code-inner js-file-line">J.C. Marioni, C.E. Mason, S.M. Mane, M. Stephens, and Y. Gilad.</td>
      </tr>
      <tr>
        <td id="L1102" class="blob-num js-line-number" data-line-number="1102"></td>
        <td id="LC1102" class="blob-code blob-code-inner js-file-line">\newblock RNA-seq: an</td>
      </tr>
      <tr>
        <td id="L1103" class="blob-num js-line-number" data-line-number="1103"></td>
        <td id="LC1103" class="blob-code blob-code-inner js-file-line">assessment of technical reproducibility and comparison with gene</td>
      </tr>
      <tr>
        <td id="L1104" class="blob-num js-line-number" data-line-number="1104"></td>
        <td id="LC1104" class="blob-code blob-code-inner js-file-line">expression arrays.</td>
      </tr>
      <tr>
        <td id="L1105" class="blob-num js-line-number" data-line-number="1105"></td>
        <td id="LC1105" class="blob-code blob-code-inner js-file-line">\newblock \emph{Genome Research}, 18: 1509 - 517, 2008.</td>
      </tr>
      <tr>
        <td id="L1106" class="blob-num js-line-number" data-line-number="1106"></td>
        <td id="LC1106" class="blob-code blob-code-inner js-file-line">
</td>
      </tr>
      <tr>
        <td id="L1107" class="blob-num js-line-number" data-line-number="1107"></td>
        <td id="LC1107" class="blob-code blob-code-inner js-file-line">
</td>
      </tr>
      <tr>
        <td id="L1108" class="blob-num js-line-number" data-line-number="1108"></td>
        <td id="LC1108" class="blob-code blob-code-inner js-file-line">\bibitem{Mortazavi2008} </td>
      </tr>
      <tr>
        <td id="L1109" class="blob-num js-line-number" data-line-number="1109"></td>
        <td id="LC1109" class="blob-code blob-code-inner js-file-line">A. Mortazavi, B.A. Williams, K. McCue, L. Schaeffer, and</td>
      </tr>
      <tr>
        <td id="L1110" class="blob-num js-line-number" data-line-number="1110"></td>
        <td id="LC1110" class="blob-code blob-code-inner js-file-line">B. Wold.  \newblock {Mapping and quantifying mammalian transcriptomes by RNA-Seq}.</td>
      </tr>
      <tr>
        <td id="L1111" class="blob-num js-line-number" data-line-number="1111"></td>
        <td id="LC1111" class="blob-code blob-code-inner js-file-line">  \newblock \emph{Nature Methods}, 5: 621 - 628, 2008.</td>
      </tr>
      <tr>
        <td id="L1112" class="blob-num js-line-number" data-line-number="1112"></td>
        <td id="LC1112" class="blob-code blob-code-inner js-file-line">
</td>
      </tr>
      <tr>
        <td id="L1113" class="blob-num js-line-number" data-line-number="1113"></td>
        <td id="LC1113" class="blob-code blob-code-inner js-file-line">
</td>
      </tr>
      <tr>
        <td id="L1114" class="blob-num js-line-number" data-line-number="1114"></td>
        <td id="LC1114" class="blob-code blob-code-inner js-file-line">\bibitem{Bullard2010}</td>
      </tr>
      <tr>
        <td id="L1115" class="blob-num js-line-number" data-line-number="1115"></td>
        <td id="LC1115" class="blob-code blob-code-inner js-file-line">J.H. Bullard, E.~Purdom, K.D. Hansen, and S.~Dudoit.</td>
      </tr>
      <tr>
        <td id="L1116" class="blob-num js-line-number" data-line-number="1116"></td>
        <td id="LC1116" class="blob-code blob-code-inner js-file-line">\newblock Evaluation of statistical methods for normalization and differential</td>
      </tr>
      <tr>
        <td id="L1117" class="blob-num js-line-number" data-line-number="1117"></td>
        <td id="LC1117" class="blob-code blob-code-inner js-file-line">  expression in {mRNA-Seq} experiments.</td>
      </tr>
      <tr>
        <td id="L1118" class="blob-num js-line-number" data-line-number="1118"></td>
        <td id="LC1118" class="blob-code blob-code-inner js-file-line">\newblock \emph{BMC bioinformatics}, 11\penalty0 (1):\penalty0 94, 2010.</td>
      </tr>
      <tr>
        <td id="L1119" class="blob-num js-line-number" data-line-number="1119"></td>
        <td id="LC1119" class="blob-code blob-code-inner js-file-line">
</td>
      </tr>
      <tr>
        <td id="L1120" class="blob-num js-line-number" data-line-number="1120"></td>
        <td id="LC1120" class="blob-code blob-code-inner js-file-line">
</td>
      </tr>
      <tr>
        <td id="L1121" class="blob-num js-line-number" data-line-number="1121"></td>
        <td id="LC1121" class="blob-code blob-code-inner js-file-line">\bibitem{Robinson2010}</td>
      </tr>
      <tr>
        <td id="L1122" class="blob-num js-line-number" data-line-number="1122"></td>
        <td id="LC1122" class="blob-code blob-code-inner js-file-line">M.D. Robinson, and A. Oshlack.</td>
      </tr>
      <tr>
        <td id="L1123" class="blob-num js-line-number" data-line-number="1123"></td>
        <td id="LC1123" class="blob-code blob-code-inner js-file-line">\newblock A scaling normalization method for differential expression analysis of {RNA-Seq} data.</td>
      </tr>
      <tr>
        <td id="L1124" class="blob-num js-line-number" data-line-number="1124"></td>
        <td id="LC1124" class="blob-code blob-code-inner js-file-line">\newblock \emph{Genome Biology}, 11: R25, 2010.</td>
      </tr>
      <tr>
        <td id="L1125" class="blob-num js-line-number" data-line-number="1125"></td>
        <td id="LC1125" class="blob-code blob-code-inner js-file-line">
</td>
      </tr>
      <tr>
        <td id="L1126" class="blob-num js-line-number" data-line-number="1126"></td>
        <td id="LC1126" class="blob-code blob-code-inner js-file-line">
</td>
      </tr>
      <tr>
        <td id="L1127" class="blob-num js-line-number" data-line-number="1127"></td>
        <td id="LC1127" class="blob-code blob-code-inner js-file-line">\bibitem{nueda2012}</td>
      </tr>
      <tr>
        <td id="L1128" class="blob-num js-line-number" data-line-number="1128"></td>
        <td id="LC1128" class="blob-code blob-code-inner js-file-line">M. Nueda, A. Conesa, and A. Ferrer.</td>
      </tr>
      <tr>
        <td id="L1129" class="blob-num js-line-number" data-line-number="1129"></td>
        <td id="LC1129" class="blob-code blob-code-inner js-file-line">\newblock {ARSyN: a method for the identification and removal</td>
      </tr>
      <tr>
        <td id="L1130" class="blob-num js-line-number" data-line-number="1130"></td>
        <td id="LC1130" class="blob-code blob-code-inner js-file-line">of systematic noise in multifactorial time-course microarray experiments}.</td>
      </tr>
      <tr>
        <td id="L1131" class="blob-num js-line-number" data-line-number="1131"></td>
        <td id="LC1131" class="blob-code blob-code-inner js-file-line">\newblock \emph{Biostatistics}, 13(3):553–566, 2012.</td>
      </tr>
      <tr>
        <td id="L1132" class="blob-num js-line-number" data-line-number="1132"></td>
        <td id="LC1132" class="blob-code blob-code-inner js-file-line">
</td>
      </tr>
      <tr>
        <td id="L1133" class="blob-num js-line-number" data-line-number="1133"></td>
        <td id="LC1133" class="blob-code blob-code-inner js-file-line">\bibitem{Efron2001}</td>
      </tr>
      <tr>
        <td id="L1134" class="blob-num js-line-number" data-line-number="1134"></td>
        <td id="LC1134" class="blob-code blob-code-inner js-file-line">B. Efron, R. Tibshirani, J.D. Storey, V. Tusher.</td>
      </tr>
      <tr>
        <td id="L1135" class="blob-num js-line-number" data-line-number="1135"></td>
        <td id="LC1135" class="blob-code blob-code-inner js-file-line">\newblock {Empirical Bayes Analysis of a Microarray Experiment}.</td>
      </tr>
      <tr>
        <td id="L1136" class="blob-num js-line-number" data-line-number="1136"></td>
        <td id="LC1136" class="blob-code blob-code-inner js-file-line">\newblock \emph{Journal of the American Statistical Association}, 2001.</td>
      </tr>
      <tr>
        <td id="L1137" class="blob-num js-line-number" data-line-number="1137"></td>
        <td id="LC1137" class="blob-code blob-code-inner js-file-line">
</td>
      </tr>
      <tr>
        <td id="L1138" class="blob-num js-line-number" data-line-number="1138"></td>
        <td id="LC1138" class="blob-code blob-code-inner js-file-line">
</td>
      </tr>
      <tr>
        <td id="L1139" class="blob-num js-line-number" data-line-number="1139"></td>
        <td id="LC1139" class="blob-code blob-code-inner js-file-line">
</td>
      </tr>
      <tr>
        <td id="L1140" class="blob-num js-line-number" data-line-number="1140"></td>
        <td id="LC1140" class="blob-code blob-code-inner js-file-line">
</td>
      </tr>
      <tr>
        <td id="L1141" class="blob-num js-line-number" data-line-number="1141"></td>
        <td id="LC1141" class="blob-code blob-code-inner js-file-line">\end{thebibliography}</td>
      </tr>
      <tr>
        <td id="L1142" class="blob-num js-line-number" data-line-number="1142"></td>
        <td id="LC1142" class="blob-code blob-code-inner js-file-line">
</td>
      </tr>
      <tr>
        <td id="L1143" class="blob-num js-line-number" data-line-number="1143"></td>
        <td id="LC1143" class="blob-code blob-code-inner js-file-line">
</td>
      </tr>
      <tr>
        <td id="L1144" class="blob-num js-line-number" data-line-number="1144"></td>
        <td id="LC1144" class="blob-code blob-code-inner js-file-line">
</td>
      </tr>
      <tr>
        <td id="L1145" class="blob-num js-line-number" data-line-number="1145"></td>
        <td id="LC1145" class="blob-code blob-code-inner js-file-line">
</td>
      </tr>
      <tr>
        <td id="L1146" class="blob-num js-line-number" data-line-number="1146"></td>
        <td id="LC1146" class="blob-code blob-code-inner js-file-line">
</td>
      </tr>
      <tr>
        <td id="L1147" class="blob-num js-line-number" data-line-number="1147"></td>
        <td id="LC1147" class="blob-code blob-code-inner js-file-line">\end{document}</td>
      </tr>
</table>

  </div>

</div>

<button type="button" data-facebox="#jump-to-line" data-facebox-class="linejump" data-hotkey="l" class="d-none">Jump to Line</button>
<div id="jump-to-line" style="display:none">
  <!-- '"` --><!-- </textarea></xmp> --></option></form><form accept-charset="UTF-8" action="" class="js-jump-to-line-form" method="get"><div style="margin:0;padding:0;display:inline"><input name="utf8" type="hidden" value="&#x2713;" /></div>
    <input class="form-control linejump-input js-jump-to-line-field" type="text" placeholder="Jump to line&hellip;" aria-label="Jump to line" autofocus>
    <button type="submit" class="btn">Go</button>
</form></div>

  </div>
  <div class="modal-backdrop js-touch-events"></div>
</div>


    </div>
  </div>

    </div>

        <div class="container site-footer-container">
  <div class="site-footer" role="contentinfo">
    <ul class="site-footer-links float-right">
        <li><a href="https://github.com/contact" data-ga-click="Footer, go to contact, text:contact">Contact GitHub</a></li>
      <li><a href="https://developer.github.com" data-ga-click="Footer, go to api, text:api">API</a></li>
      <li><a href="https://training.github.com" data-ga-click="Footer, go to training, text:training">Training</a></li>
      <li><a href="https://shop.github.com" data-ga-click="Footer, go to shop, text:shop">Shop</a></li>
        <li><a href="https://github.com/blog" data-ga-click="Footer, go to blog, text:blog">Blog</a></li>
        <li><a href="https://github.com/about" data-ga-click="Footer, go to about, text:about">About</a></li>

    </ul>

    <a href="https://github.com" aria-label="Homepage" class="site-footer-mark" title="GitHub">
      <svg aria-hidden="true" class="octicon octicon-mark-github" height="24" version="1.1" viewBox="0 0 16 16" width="24"><path fill-rule="evenodd" d="M8 0C3.58 0 0 3.58 0 8c0 3.54 2.29 6.53 5.47 7.59.4.07.55-.17.55-.38 0-.19-.01-.82-.01-1.49-2.01.37-2.53-.49-2.69-.94-.09-.23-.48-.94-.82-1.13-.28-.15-.68-.52-.01-.53.63-.01 1.08.58 1.23.82.72 1.21 1.87.87 2.33.66.07-.52.28-.87.51-1.07-1.78-.2-3.64-.89-3.64-3.95 0-.87.31-1.59.82-2.15-.08-.2-.36-1.02.08-2.12 0 0 .67-.21 2.2.82.64-.18 1.32-.27 2-.27.68 0 1.36.09 2 .27 1.53-1.04 2.2-.82 2.2-.82.44 1.1.16 1.92.08 2.12.51.56.82 1.27.82 2.15 0 3.07-1.87 3.75-3.65 3.95.29.25.54.73.54 1.48 0 1.07-.01 1.93-.01 2.2 0 .21.15.46.55.38A8.013 8.013 0 0 0 16 8c0-4.42-3.58-8-8-8z"/></svg>
</a>
    <ul class="site-footer-links">
      <li>&copy; 2016 <span title="0.19471s from github-fe163-cp1-prd.iad.github.net">GitHub</span>, Inc.</li>
        <li><a href="https://github.com/site/terms" data-ga-click="Footer, go to terms, text:terms">Terms</a></li>
        <li><a href="https://github.com/site/privacy" data-ga-click="Footer, go to privacy, text:privacy">Privacy</a></li>
        <li><a href="https://github.com/security" data-ga-click="Footer, go to security, text:security">Security</a></li>
        <li><a href="https://status.github.com/" data-ga-click="Footer, go to status, text:status">Status</a></li>
        <li><a href="https://help.github.com" data-ga-click="Footer, go to help, text:help">Help</a></li>
    </ul>
  </div>
</div>



    

    <div id="ajax-error-message" class="ajax-error-message flash flash-error">
      <svg aria-hidden="true" class="octicon octicon-alert" height="16" version="1.1" viewBox="0 0 16 16" width="16"><path fill-rule="evenodd" d="M8.865 1.52c-.18-.31-.51-.5-.87-.5s-.69.19-.87.5L.275 13.5c-.18.31-.18.69 0 1 .19.31.52.5.87.5h13.7c.36 0 .69-.19.86-.5.17-.31.18-.69.01-1L8.865 1.52zM8.995 13h-2v-2h2v2zm0-3h-2V6h2v4z"/></svg>
      <button type="button" class="flash-close js-flash-close js-ajax-error-dismiss" aria-label="Dismiss error">
        <svg aria-hidden="true" class="octicon octicon-x" height="16" version="1.1" viewBox="0 0 12 16" width="12"><path fill-rule="evenodd" d="M7.48 8l3.75 3.75-1.48 1.48L6 9.48l-3.75 3.75-1.48-1.48L4.52 8 .77 4.25l1.48-1.48L6 6.52l3.75-3.75 1.48 1.48z"/></svg>
      </button>
      You can't perform that action at this time.
    </div>


      
      <script crossorigin="anonymous" integrity="sha256-eGrxxkowQBvwoW6v7VFBW5vLA/cv8xg6H6YAfXyw+Xk=" src="https://assets-cdn.github.com/assets/frameworks-786af1c64a30401bf0a16eafed51415b9bcb03f72ff3183a1fa6007d7cb0f979.js"></script>
      <script async="async" crossorigin="anonymous" integrity="sha256-rT+1ElnxP2D9CTo6KrTwb/OeuYr06jkKhbMcWYrUhew=" src="https://assets-cdn.github.com/assets/github-ad3fb51259f13f60fd093a3a2ab4f06ff39eb98af4ea390a85b31c598ad485ec.js"></script>
      
      
      
      
    <div class="js-stale-session-flash stale-session-flash flash flash-warn flash-banner d-none">
      <svg aria-hidden="true" class="octicon octicon-alert" height="16" version="1.1" viewBox="0 0 16 16" width="16"><path fill-rule="evenodd" d="M8.865 1.52c-.18-.31-.51-.5-.87-.5s-.69.19-.87.5L.275 13.5c-.18.31-.18.69 0 1 .19.31.52.5.87.5h13.7c.36 0 .69-.19.86-.5.17-.31.18-.69.01-1L8.865 1.52zM8.995 13h-2v-2h2v2zm0-3h-2V6h2v4z"/></svg>
      <span class="signed-in-tab-flash">You signed in with another tab or window. <a href="">Reload</a> to refresh your session.</span>
      <span class="signed-out-tab-flash">You signed out in another tab or window. <a href="">Reload</a> to refresh your session.</span>
    </div>
    <div class="facebox" id="facebox" style="display:none;">
  <div class="facebox-popup">
    <div class="facebox-content" role="dialog" aria-labelledby="facebox-header" aria-describedby="facebox-description">
    </div>
    <button type="button" class="facebox-close js-facebox-close" aria-label="Close modal">
      <svg aria-hidden="true" class="octicon octicon-x" height="16" version="1.1" viewBox="0 0 12 16" width="12"><path fill-rule="evenodd" d="M7.48 8l3.75 3.75-1.48 1.48L6 9.48l-3.75 3.75-1.48-1.48L4.52 8 .77 4.25l1.48-1.48L6 6.52l3.75-3.75 1.48 1.48z"/></svg>
    </button>
  </div>
</div>

  </body>
</html>

